\section{Zero-Knowledge and Proofs of Knowledge}
\label{sec:zknowledge}
Identity protocols provide proof about the ownership of a private key, without revealing that private key.
We can generalize this notion to proving knowledge of an arbitrary secret without revealing the secret as \emph{zero-knowledge (ZK) proofs}. 
The notion of zero-knowledge was first introduced by Goldwasser et al.~\cite{goldwasser1989knowledge}

For the formal definition of a ZK proof, we follow Boneh-Shoup's treatment~\cite{BonehShoupBook} and define an effective relation as a binary relation $\calR \subseteq \calX \times \calY$  where $\calR, \calX,\calY$ are sets.
(We assume the are efficiently recognizable, meaning one can determine if a value is a member of the set efficiently.)
Elements of $\calY$ are called \emph{statements} and elements of $\calX$ are called \emph{witnesses}.

We can define Sigma protocols, and thus interactive zero-knowledge proofs, relative to relations.

\begin{figure}[h]
\centering

\begin{tikzpicture}
    \node[draw, rectangle] (P) {Prover};
    \node[below=5cm of P] (P2) {};

    \node[draw, rectangle, right=5cm of P] (V) {Verifier};
    \node[below=5cm of V] (V2) {};

    \draw[-] (P) -- (P2);
    \draw[-] (V) -- (V2);

\end{tikzpicture}
\end{figure}

\subsection{Properties of Zero-Knowledge Proofs}

\subsection{Non-Interactive Zero-Knowledge Proofs}
\begin{figure}
\centering

\fpage{.25}{
\underline{$\NIZK_{\Phi,\Simu}^\advA$}\\[1pt]
$b \getsr \bits$\\
$b' \getsr \advA^{\ProofOracle}$\\
Return $(b = b')$\medskip

\underline{$\ProofOracle(x,Y)$}\\[1pt]
If $b = 1$ then\\
\myInd $\pi \getsr \Gen(x,Y)$\\
If $b = 0$ then \\
\myInd $\pi \getsr \Simu(Y)$\\
Return $\pi$
}
\fpage{.15}{
\underline{$\NIZK1_{\Phi,\Simu}^\advA$}\\[1pt]
$b \getsr \bits$\\
$b' \getsr \advA^{\ProofOracle,\Horacle}$
Ret $(b = b')$\medskip

\underline{$\ProofOracle(x,Y)$}\\[1pt]
$\pi \getsr \Gen^\Horacle(x,Y)$\\
Ret $\pi$\medskip

\underline{$\Horacle(M)$}\\[1pt]
If $\TabH[M] = \bot$ then\\
\myInd $\TabH[M] \getsr \bits^n$\\
Ret $\TabH[M]$
}
\fpage{.15}{
\underline{$\NIZK0_{\Phi,\Simu}^\advA$}\\[1pt]
$b \getsr \bits$\\
$b' \getsr \advA^{\ProofOracle,\HashSim}$\\
Return $(b = b')$\medskip

\underline{$\ProofOracle(x,Y)$}\\[1pt]
$\pi \getsr \Simu_P(Y)$\\
Return $\pi$\medskip

\underline{$\HashSim(M)$}\\[1pt]
Ret $\Simu_H(M)$
}

\caption{Definition of the Non-Interactive Zero-Knowledge Game}
    \label{fig:nizk}
\end{figure}

So far, we have shown the basic intuition behind showing zero-knowledge in an interactive setting.
Informally speaking in interactive settings the verifier can ask ``questions'' about the proof and gets answers in form of witnesses from the prover.
In an non-interactive setting, the prover has to provide answer to all possible ``questions'' a priori.


Figure \ref{fig:nizk} outlines the basic games for the non-interactive zero-knowledge proof.



