\section{Zero-Knowledge and Proofs of Knowledge}
\label{sec:zknowledge}
Identity protocols provide proof about the ownership of a private key, without revealing that private key.
We can generalize this notion to proving knowledge of an arbitrary secret without revealing the secret as \emph{zero-knowledge (ZK) proofs}. 
The notion of zero-knowledge was first introduced by Goldwasser et al.~\cite{goldwasser1989knowledge}

For the formal definition of a ZK proof, we follow Boneh-Shoup's treatment~\cite{BonehShoupBook} and define an effective relation as a binary relation $\calR \subseteq \calX \times \calY$  where $\calR, \calX,\calY$ are sets.
(We assume the are efficiently recognizable, meaning one can determine if a value is a member of the set efficiently.)
Elements of $\calY$ are called \emph{statements} and elements of $\calX$ are called \emph{witnesses}.

To demonstrate that identity protocols are zero-knowledge proofs of knowledge, we define the protocol from the previous section relative to relations.
A \emph{proof of knowledge} are particular ZK proof, where the verifier learns nothing about the witness except that the provided statement is true.
Figure \ref{fig:sigmaid} outlines the resulting protocol, which is semantically identical to the identity protocol but relies on notation based on relations.
Here, $Y$ takes the role of the public key and $x$ the role of the secret key.

\begin{figure}[h]
\centering

\begin{tikzpicture}
    \node[draw, rectangle] (P) {Prover};
    \node[below=4cm of P] (P2) {};
    \node[right=5cm of P, draw, rectangle] (V) {Verifier};
    \node[below=4cm of V] (V2) {};

    \node[below=1cm of P] (a) {};
    \node[below=1.5cm of V] (b) {};
    \node[below=2cm of V] (c) {};
    \node[below=2.5cm of P] (d) {};
    \node[below=3cm of P] (e) {};
    \node[below=3.5cm of V] (f) {};

    \node[left=0.2cm of a] (genR) { $R \getsr P.com(Y)$ };
    \node[right=0.2cm of c] (genc) { $c \getsr C$ };
    \node[left=0.2cm of e] (genz) { $z \gets P.resp(x,R,c)$ };
    \node[right=0.2cm of f] (verify) { $b \gets V.ver(Y,R,c,z)$ };

    \draw[-] (P) -- (P2);
    \draw[-] (V) -- (V2);

    \draw[->] (a.center) -- (b.center) node [midway, above] {$R$};
    \draw[->] (c.center) -- (d.center) node [midway, above] {$c$};
    \draw[->] (e.center) -- (f.center) node [midway, above] {$z$};
\end{tikzpicture}

\caption{The identity protocol adapted to the zero-knowledge terminology}
\label{fig:sigmaid}
\end{figure}

\subsection{Sigma protocol for generalized linear relationships}

So far, we showed how to provide zero-knowledge proofs for simple binary relationships.
To support more complex applications, we now generalize the Sigma protocol to prove predicates of the following form.
In particular, this scheme supports encoding any linear programming problem, which, in turn, means it can encode any problem in NP.

\begin{equation*}
    \phi(x_1, \dots, x_n) = \left\{ u_1 = \prod_{j=1}^{n} g_{1j}^{x_j} \land \dots \land u_m  = \prod_{j=1}^{n} g_{mj}^{x_j}\right\}
\end{equation*}

Here, $x_1, \dots, x_n$ are the witnesses. The statements are encoded by the $u_j$ and $g_{ij}$ values.
The values $R$, $c$, and $z$ define the communication-specific \emph{transcript}.
We can then adapt this protocol to different use-cases by adjusting $m$ and $n$. Some examples are listed in Table \ref{tab:sigmaexamples}.

\begin{table}[h]
\begin{tabular}{|c|c|c|}
\hline
    \textbf{Name} & \textbf{Description} & \textbf{Formula} \\
\hline
    Schnorr        & Digital Signatures & $n=1$, $m=1$, $\phi(x) = \left\{ u = g^x \right\}$ \\
    Chaum-Pederson & Prove that X,Y,Z is a DH triple &  $n=1$, $m=2$, $\phi(x) = \left\{ X = g^x \land Z = Y^x\right\}$ \\
    Okamoto        & Prove ``representation'' of dlog &  $n=2$, $m=1$, $\phi(x_1, x_2) = \left\{ u = g^{x_1} h^{x_2} \right\} $ \\
\hline
\end{tabular}

\caption{Examples of the applications for the generalized sigma protocol}
\label{tab:sigmaexamples}
\end{table}

Figure \ref{fig:sigmageneral} shows how this generalization applies to the communication between prover and verifier.


\begin{figure}[h]
\centering

\begin{tikzpicture}
    \node[draw, rectangle, align=center] (P) {\textbf{Prover} \\ (knows $x_1, \dots, x_n$)};
    \node[below=4cm of P] (P2) {};
    \node[right=5cm of P, draw, rectangle, align=center] (V) {\textbf{Verifier} \\ (knows $u_1, \dots, u_m$) };
    \node[below=4cm of V] (V2) {};

    \node[below=1cm of P] (a) {};
    \node[below=1.5cm of V] (b) {};
    \node[below=2cm of V] (c) {};
    \node[below=2.5cm of P] (d) {};
    \node[below=3cm of P] (e) {};
    \node[below=3.5cm of V] (f) {};

    \node[left=0.2cm of a, align=right] (genR) { $r_1, \dots, r_n \getsr \Z_q$ \\ $R_i \gets \prod_{j=1}^n g_{ij}^{r_j}$ };
    \node[right=0.2cm of c] (genc) { $c \getsr \Z_q$ };
    \node[left=0.2cm of e, align=right] (genz) { $z_j \gets r_j + x_jc$ \\ for $j=1$ to $n$ };
    \node[right=0.2cm of f, align=center] (verify) { Check that for $i=1$ to $m$\\
    $\prod_{j=1}^{n} g_{ij}^{z_j} = R_i * u_i^c $ };

    \draw[-] (P) -- (P2);
    \draw[-] (V) -- (V2);

    \draw[->] (a.center) -- (b.center) node [midway, above] {$R_1, \dots, R_m$};
    \draw[->] (c.center) -- (d.center) node [midway, label=$c$] {};
    \draw[->] (e.center) -- (f.center) node [midway, label=$z$] {};

\end{tikzpicture}

\caption{Sigma protocol for generalized interactive zero-knowledge proofs}
\label{fig:sigmageneral}
\end{figure}

In the following we show that verifying equation holds for a correct proof.

\begin{align*}
    \prod_{j=1}^{n} g_{ij}^{z_j} = R_i * u_i^c && \\
    \Leftrightarrow \prod_{j=1}^{n} g_{ij}^{r_j} * g_{ij}^{x_j*c} = R_i * u_i^c \\
    \Leftrightarrow R_i * \prod_{j=1}^{n} (g_{ij}^{x_j})^c = R_i * u_i^c \\
    \Leftrightarrow R_i * u_i^c = R_i * u_i^c  \qed \\
\end{align*}

\subsubsection{Application: Providing Encrypted Message Equality}
Suppose we want to prove that we encrypted the same message with two distinct public keys.
In particular, we show how to show message equality under ElGamal, without revealing the message itself or the randomness chosen during encryption.

\begin{align*}
{enc}({pk}_1, M) = (g^{r_1}, pk_1^{r_1} *M) \\
{enc}({pk}_2, M) = (g^{r_2}, {pk}_2^{r_2} * M) \\
\end{align*}

The proof is then generated using the following relation $\calR$, where $r_1$ and $r_2$ are the chosen randomness for ElGamal, and $pk_1$ and $pk_2$ are the different public keys.

\begin{align*}
    \calR = \left\{ ( \underbrace{(r_1, r_2, M)}_{\text{Witness}}, \underbrace{(R_1, {pk}_1, C_1, R_2, {pk}_2, C_2)}_{\text{Statement}} ) : R_1 = g^{r_1}, C_1 = {pk}_1^{r_1} *M, R_2 = g^{r_2}, C_2 = {pk}_2^{r_2} * M \right\}
\end{align*}

We can then verify the relation using the generalized Sigma equation from above.

\begin{align*}
    \phi(r_1, r_2) = \left\{ R_1 = g^{r_1} \land R_2 = g^{r_2} \land C_1 / C_2 = pk_1^{r_1} pk_2^{-r_2} \right\}
\end{align*}

\subsection{Properties of Zero-Knowledge Proofs}
All zero-knowledge proofs can be described as a proof that for some $x$ it holds that $x \in \calL$, where is a language ($\calL = \{0,1\}^{*}$~\cite{goldwasser1989knowledge}).
We further define the following properties for such proofs.

\paragraph{Completeness} For any element $x \in \calL$ we can generate a valid ZK proof.
We define a ZK proof as valid if it is accepted by a verifier with high probability in polynomial time.

\paragraph{Soundness} An adversary cannot generate a valid ZK proof for any element $x' \not \in \calL$ in polynomial time.

\paragraph{Zero-Knowledge} The proof reveals no other information about $x$.

\scribenote{Do we need to define "information" or "knowledge" here?}

There exist different variations of these properties. We will describe two of those in detail: knowledge soundness and honest-verifier zero-knowledge.

\subsubsection*{Knowledge Soundness for Sigma Protocols}
We define \emph{knowledge soundness} as the property that an entity can only generate a valid statement for a witness, if they know the witness.

For Sigma protocols in particular, we can show that a prover can derive its witness by communicating with itself.
Assume we have some extractor function that does not have direct access $w$ but can invoke the prover as a black box (including rewinding the prover's state).
We can then show that this extractor can compute the witness $w$ from two transcripts about some statement $Y$, such that $(w, Y) \in \calR$.
To do this, the extractor rewinds the communication after the first transcript $(R,c,z)$ such that for the second transcript $(R', c', z')$ it holds that $R = R'$.

\begin{align*}
    g^z = R * g^{xc} \Leftrightarrow R = g^{z-xc} \\
    g^{z'} = R' * g^{x'c'} \Leftrightarrow R' = g^{z'-xc'} \\
\end{align*}

Because $R=R'$ we can solve $z-xc = z'-xc'$ for $x$ to retrieve the witness. The same approach can also be applied to the generalized Sigma protocol.

\subsubsection{Honest-Verifier Zero-Knowledge for Sigma Protocols}
We define \emph{honest-verifier zero-knowledge (HVZK)} as the property that no information aside from $x \in \calL$ is leaked to an honest verifier.
To prove the property for Sigma protocols we construct a simulator $S$ that has black-box access to the verifier $V$.
We then show that $S$ can a transcript that is different from the previously seen transcript(s) but will be accepted by $V$.
The intuition behind this proof is that, if we can generate (computationally) valid transcript without talking to the prover directly, no other information about $w$ is revealed.

Because the verifier is honest we can assume that $c$ in the original transcript $(R,c,z)$ is chosen uniformly at random.
The goal for $S$ is then to generate a new transcript $(R', c, z')$, where $R' \neq R$ and $z \neq z'$.
To achieve this we can apply a similar approach as previously. We know $c$, $R$, and $z$. We then set $z' \getsr \calZ$ and solve for $R'$.

\subsection{Non-Interactive Zero-Knowledge Proofs}
\begin{figure}
\centering

\fpage{.25}{
\underline{$\NIZK_{\Phi,\Simu}^\advA$}\\[1pt]
$b \getsr \bits$\\
$b' \getsr \advA^{\ProofOracle}$\\
Return $(b = b')$\medskip

\underline{$\ProofOracle(x,Y)$}\\[1pt]
If $b = 1$ then\\
\myInd $\pi \getsr \Gen(x,Y)$\\
If $b = 0$ then \\
\myInd $\pi \getsr \Simu(Y)$\\
Return $\pi$
}
\fpage{.15}{
\underline{$\NIZK1_{\Phi,\Simu}^\advA$}\\[1pt]
$b \getsr \bits$\\
$b' \getsr \advA^{\ProofOracle,\Horacle}$
Ret $(b = b')$\medskip

\underline{$\ProofOracle(x,Y)$}\\[1pt]
$\pi \getsr \Gen^\Horacle(x,Y)$\\
Ret $\pi$\medskip

\underline{$\Horacle(M)$}\\[1pt]
If $\TabH[M] = \bot$ then\\
\myInd $\TabH[M] \getsr \bits^n$\\
Ret $\TabH[M]$
}
\fpage{.15}{
\underline{$\NIZK0_{\Phi,\Simu}^\advA$}\\[1pt]
$b \getsr \bits$\\
$b' \getsr \advA^{\ProofOracle,\HashSim}$\\
Return $(b = b')$\medskip

\underline{$\ProofOracle(x,Y)$}\\[1pt]
$\pi \getsr \Simu_P(Y)$\\
Return $\pi$\medskip

\underline{$\HashSim(M)$}\\[1pt]
Ret $\Simu_H(M)$
}

\caption{Definition of the Non-Interactive Zero-Knowledge Game}
    \label{fig:nizk}
\end{figure}

So far, we have shown the basic intuition behind showing zero-knowledge in an interactive setting.
Informally speaking in interactive settings the verifier can ask ``questions'' about the proof and gets answers in form of witnesses from the prover.
In an non-interactive setting, the prover has to provide answer to all possible ``questions'' a priori.

Figure \ref{fig:nizk} outlines the basic games for the \emph{non-interactive zero-knowledge (NIZK) proof}.
Here, we want to ensure that a real NIZK proof is indistinguishable from a simulated proof.


