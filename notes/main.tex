%%%%%%%%%%%%%%%%%%%%%%%%%%%%%%%%%%%%%%%%%%%%%%%%%%%%%%%%%%%%%%%%%%%%%%%%%%%%%%%%
% title:
% authors:
%%%%%%%%%%%%%%%%%%%%%%%%%%%%%%%%%%%%%%%%%%%%%%%%%%%%%%%%%%%%%%%%%%%%%%%%%%%%%%%%
\def\fullver{0}
\def\submission{1}
\documentclass[11pt]{article}

\usepackage[letterpaper,hmargin=1in,vmargin=1in]{geometry}
\usepackage[backref=true,backend=bibtex]{biblatex}

\DefineBibliographyStrings{english}{%
  backrefpage = {page},% originally "cited on page"
  backrefpages = {pages},% originally "cited on pages"
}


\usepackage{framed}
%\usepackage{cite}
\usepackage[hidelinks]{hyperref}

\usepackage{color}
\usepackage{float}
\usepackage{setspace}
\usepackage{booktabs}
\usepackage{outlines}
\usepackage{enumitem}
\usepackage{epsfig}
\usepackage{wrapfig}
\usepackage{textcomp}
\usepackage{rotating}
\usepackage{xspace}
\usepackage{tikz}
\usetikzlibrary{calc,arrows,arrows.meta,shapes,positioning,matrix,decorations.pathreplacing}
\usepackage{amsthm}
\usepackage{amssymb}
\usepackage{pifont}
\usepackage{amsfonts}
\usepackage{amsmath}
\DeclareMathOperator*{\argmax}{arg\,max}
\DeclareMathOperator*{\argmin}{arg\,min}
\usepackage{cleveref}
\usepackage{bm}
\usepackage{listings}
\usepackage{mathtools}
\allowdisplaybreaks[2]
\usepackage{latexsym}
\usepackage{graphics}
\usepackage{graphicx}
\usepackage{fancyhdr}
\usepackage{url}
%\usepackage{enumerate}
\usepackage{adjustbox}
\usepackage{multirow}
\usepackage[font=small]{caption}
\usepackage[font=small]{subcaption}
\def\authnotes{1}
\newtheorem{thm}{Theorem} %[section]
\newtheorem*{theorem*}{Theorem}
\newtheorem{lem}{Lemma}
\newtheorem*{lemma*}{Lemma}

\newtheorem{cor}[thm]{Corollary}
\newtheorem{propo}[thm]{Proposition}
\newtheorem{defn}[thm]{Definition}
\newtheorem{assm}[thm]{Assumption}
\newtheorem{clm}[thm]{Claim}
\newtheorem{rem}[thm]{Remark}
\newtheorem{exa}{Example}
\newtheorem{fact}{Fact}

\newenvironment{theorem}{\begin{thm}
    \begin{sl}
    }%
    {
    \end{sl}
  \end{thm}}

\newenvironment{lemma}{\begin{lem}\begin{sl}}%
    {\end{sl}\end{lem}}
\newenvironment{corollary}{\begin{cor}\begin{sl}}%
    {\end{sl}\end{cor}}
\newenvironment{proposition}{\begin{propo}\begin{sl}}%
    {\end{sl}\end{propo}}
\newenvironment{definition}{\begin{defn}\begin{sl}}%
    {\end{sl}\end{defn}}
\newenvironment{assumption}{\begin{assm}\begin{em}}%
    {\end{em}\end{assm}}
\newenvironment{claim}{\begin{clm}\begin{sl}}%
    {\end{sl}\end{clm}}
\newenvironment{remark}{\begin{rem}\begin{em}}%
    {\end{em}\end{rem}}
\newenvironment{example}{\begin{exa}\begin{em}}%
    {\end{em}\end{exa}}

\newcommand{\lemautorefname}{Lemma}
\newcommand{\algorithmautorefname}{Algorithm}
\renewcommand{\subsectionautorefname}{Section}


% Deluxe proof enviroment
\iffalse
    \def\qsym{\vrule width0.6ex height1em depth0ex}
    \def\qedsym{{\hspace{5pt}\rule[-1pt]{3pt}{9pt}}}
    \newcount\proofqeded
    \newcount\proofended
    \def\qed{%\qedsym
    \end{rm}\addtolength{\parskip}{-0pt}
    \setlength{\parindent}{\saveparindent}
    \global\advance\proofqeded by 1 }
    \newenvironment{proof}%
     {\proofstart}%
     {\ifnum\proofqeded=\proofended\qed\fi \global\advance\proofended by 1
      \medskip}
    \makeatletter
    \def\proofstart{\@ifnextchar[{\@oprf}{\@nprf}}
    \def\@oprf[#1]{\begin{rm}\protect\vspace{6pt}\noindent{\bf Proof of #1:\
    }%
    \addtolength{\parskip}{5pt}\setlength{\parindent}{0pt}}
    \def\@nprf{\begin{rm}\protect\vspace{6pt}\noindent{\bf Proof:\ }%
    \addtolength{\parskip}{5pt}\setlength{\parindent}{0pt}}
  \makeatother
\fi


% Extra padding for table entries
\newcommand\Tvsp{\rule{0pt}{2.6ex}}
\newcommand\Bvsp{\rule[-1.2ex]{0pt}{0pt}}
\newcommand{\TabPad}{\hspace*{5pt}}
\newcommand\TabSep{@{\hspace{5pt}}|@{\hspace{5pt}}}
\newcommand\TabSepLeft{|@{\hspace{5pt}}}
\newcommand\TabSepRight{@{\hspace{5pt}}|}



%\def\qedsym{{\ifnum\llncsclass=0  \else {\hspace{5pt}\rule[-1pt]{3pt}{9pt}} \fi}}
\def\qedsym{{\hspace{5pt}\rule[-1pt]{3pt}{9pt}} }

\def\pseudocodesize{ \footnotesize }

\renewcommand{\arraystretch}{1.3}
\DeclareMathAlphabet{\mathsl}{OT1}{cmr}{m}{sl}
\DeclareMathAlphabet{\mathsc}{OT1}{cmr}{m}{sc}


% Reference expansions
\newcommand{\secref}[1]{\mbox{Section~\ref{#1}}}
\newcommand{\subsecref}[1]{\mbox{Subsection~\ref{#1}}}
\newcommand{\apref}[1]{\mbox{Appendix~\ref{#1}}}
\newcommand{\thref}[1]{\mbox{Theorem~\ref{#1}}}
\newcommand{\thmrefshort}[1]{\mbox{\textbf{Th~\ref{#1}}}}
\newcommand{\exref}[1]{\mbox{Example~\ref{#1}}}
\newcommand{\defref}[1]{\mbox{Definition~\ref{#1}}}
\newcommand{\corref}[1]{\mbox{Corollary~\ref{#1}}}
\newcommand{\lemref}[1]{\mbox{Lemma~\ref{#1}}}
\newcommand{\clref}[1]{\mbox{Claim~\ref{#1}}}
\newcommand{\propref}[1]{\mbox{Proposition~\ref{#1}}}
\newcommand{\consref}[1]{\mbox{Construction~\ref{#1}}}
\newcommand{\figref}[1]{\mbox{Figure~\ref{#1}}}
%\newcommand{\eqref}[1]{\mbox{Equation~\ref{#1}}}

\newcommand{\enumref}[1]{\mbox{(\ref{#1})}}

\newcommand{\thmlabel}[1]{\textnormal{\textbf{[#1]}}}

\newcommand{\SetFigFont}[5]{} % This is for xfig issue
\newcommand{\SetFigFontNFSS}[5]{} % This is for xfig issue

\DeclarePairedDelimiter{\ceil}{\lceil}{\rceil}%
\DeclarePairedDelimiter{\floor}{\lfloor}{\rfloor}%
\DeclarePairedDelimiter\absv{\lvert}{\rvert}%
%\DeclarePairedDelimiter\norm{\lVert}{\rVert}%
\DeclarePairedDelimiter\prns{(}{)}%
\DeclarePairedDelimiter\braces{\{}{\}}%
\DeclarePairedDelimiter\bracks{[}{]}%
\DeclarePairedDelimiterX\condprns[2]{(}{)}{\,#1 \;\delimsize\vert\; #2\,}
\DeclarePairedDelimiterX\condbrks[2]{[}{]}{\,#1 \;\delimsize\vert\; #2\,}
\DeclarePairedDelimiterX\condbraces[2]{\{}{\}}{\,#1 \;\delimsize\vert\; #2\,}

% ========================================================================

% Lists

\newcounter{ctr}
\newcounter{savectr}
\newcounter{ectr}

\newlength{\saveparindent}
\setlength{\saveparindent}{\parindent}
\newlength{\saveparskip}
\setlength{\saveparskip}{\parskip}


\newenvironment{newitemize}{%
\begin{list}{\mbox{}\hspace{5pt}$\bullet$\hfill}{\labelwidth=15pt%
\labelsep=5pt \leftmargin=20pt \topsep=3pt%
\setlength{\listparindent}{\saveparindent}%
\setlength{\parsep}{\saveparskip}%
\setlength{\itemsep}{3pt} }}{\end{list}}


\newenvironment{newenum}{%
\begin{list}{{\rm (\arabic{ctr})}\hfill}{\usecounter{ctr} \labelwidth=17pt%
\labelsep=5pt \leftmargin=22pt \topsep=3pt%
\setlength{\listparindent}{\saveparindent}%
\setlength{\parsep}{\saveparskip}%
\setlength{\itemsep}{2pt} }}{\end{list}}

\newenvironment{tiret}{%
\begin{list}{\hspace{2pt}\rule[0.5ex]{6pt}{1pt}\hfill}{\labelwidth=15pt%
\labelsep=3pt \leftmargin=22pt \topsep=3pt%
\setlength{\listparindent}{\saveparindent}%
\setlength{\parsep}{\saveparskip}%
\setlength{\itemsep}{2pt}}}{\end{list}}


\newenvironment{blocklist}{\begin{list}{}{\labelwidth=0pt%
\labelsep=0pt \leftmargin=0pt \topsep=10pt%
\setlength{\listparindent}{\saveparindent}%
\setlength{\parsep}{\saveparskip}%
\setlength{\itemsep}{20pt}}}{\end{list}}

\newenvironment{blocklistindented}{\begin{list}{}{\labelwidth=0pt%
\labelsep=30pt \leftmargin=30pt\topsep=5pt%
\setlength{\listparindent}{\saveparindent}%
\setlength{\parsep}{\saveparskip}%
\setlength{\itemsep}{10pt}}}{\end{list}}

\newenvironment{onelist}{%
\begin{list}{{\rm (\arabic{ctr})}\hfill}{\usecounter{ctr} \labelwidth=18pt%
\labelsep=7pt \leftmargin=25pt \topsep=2pt%
\setlength{\listparindent}{\saveparindent}%
\setlength{\parsep}{\saveparskip}%
\setlength{\itemsep}{2pt} }}{\end{list}}

\newenvironment{twolist}{%
\begin{list}{{\rm (\arabic{ctr}.\arabic{ectr})}%
\hfill}{\usecounter{ectr} \labelwidth=26pt%
\labelsep=7pt \leftmargin=33pt \topsep=2pt%
\setlength{\listparindent}{\saveparindent}%
\setlength{\parsep}{\saveparskip}%
\setlength{\itemsep}{2pt} }}{\end{list}}

\newenvironment{centerlist}{%
\begin{list}{\mbox{}}{\labelwidth=0pt%
\labelsep=0pt \leftmargin=0pt \topsep=10pt%
\setlength{\listparindent}{\saveparindent}%
\setlength{\parsep}{\saveparskip}%
\setlength{\itemsep}{10pt} }}{\end{list}}

\newenvironment{newcenter}[1]{\begin{centerlist}\centering%
\item #1}{\end{centerlist}}

\newenvironment{codecenter}[1]{\begin{small}\begin{centerlist}\centering%
\item #1}{\end{centerlist}\end{small}}

% =========================================================================

% Math

\newlength{\savejot}
\setlength{\jot}{3pt}
\setlength{\savejot}{\jot}

\newenvironment{newmath}{\begin{displaymath}%
\setlength{\abovedisplayskip}{4pt}%
\setlength{\belowdisplayskip}{4pt}%
\setlength{\abovedisplayshortskip}{6pt}%
\setlength{\belowdisplayshortskip}{6pt} }{\end{displaymath}}

\newenvironment{newequation}{\begin{equation}%
\setlength{\abovedisplayskip}{4pt}%
\setlength{\belowdisplayskip}{4pt}%
\setlength{\abovedisplayshortskip}{6pt}%
\setlength{\belowdisplayshortskip}{6pt} }{\end{equation}}

%\newcommand{\headingg}[1]{{\textbf{#1}}}
%\newcommand{\heading}[1]{{\vspace{6pt}\noindent\textbf{#1}}}
\newcommand{\ind}{\hspace*{1.5em}}
\newcommand{\indsm}{\hspace*{.75em}}
\newcommand{\indeqn}{\;\;\;\;\;\;\;}
\newcommand{\bits}{\{0,1\}}
\newcommand{\zon}[1]{\bits^{#1}}
\newcommand{\emptystring}{\varepsilon}
\newcommand{\xor}{{\:\oplus\:}}
%\newcommand{\concat}{\,,\,}
\newcommand{\smidge}{{\kern .05em}}
\newcommand{\Colon}{{\smidge\colon\smidge}}
\newcommand{\norm}[1]{\|#1\|}
\def\poly{\mathop{\rm poly}\nolimits}
\def\div{\mathop{\rm div}\nolimits}
%\newcommand{\ro}{RO}
\newcommand{\advA}{{\mathcal A}}
\newcommand{\advB}{{\mathcal B}}
\newcommand{\advC}{{\mathcal C}}
\newcommand{\advD}{{\mathcal D}}
\newcommand{\advE}{{\mathcal E}}

\newcommand{\calA}{{\cal A}}
\newcommand{\calB}{{\cal B}}
\newcommand{\calC}{{\cal C}}
\newcommand{\calE}{{\cal E}}
\newcommand{\calF}{{\cal F}}
\newcommand{\calG}{{\cal G}}
\newcommand{\calH}{{\cal H}}
\newcommand{\calI}{{\cal I}}
\newcommand{\calJ}{{\cal J}}
\newcommand{\calO}{{\cal O}}
\newcommand{\calR}{{\cal R}}
\newcommand{\calS}{{\cal S}}
\newcommand{\calD}{{\cal D}}
\newcommand{\calK}{{\cal K}}
\newcommand{\calL}{{\cal L}}
\newcommand{\calM}{{\cal M}}
\newcommand{\calN}{{\cal N}}
\newcommand{\calP}{{\cal P}}
\newcommand{\calQ}{{\cal Q}}
\newcommand{\calT}{{\cal T}}
\newcommand{\calU}{{\cal U}}
\newcommand{\calV}{{\cal V}}
\newcommand{\calW}{{\cal W}}
\newcommand{\calX}{{\cal X}}
\newcommand{\calY}{{\cal Y}}

\newcommand{\N}{{{\mathbb N}}}
\newcommand{\Z}{{{\mathbb Z}}}
\newcommand{\G}{{{\textnormal G}}}
\newcommand{\Hgame}{{{\textnormal H}}}
\newcommand{\bigO}{\calO}
\newcommand{\goesto}{{\rightarrow}}
\newcommand{\eqdef}{\stackrel{\rm def}{=}}
\newcommand{\negsmidge}{{\hspace{-0.1ex}}}
\newcommand{\cdotsm}{\negsmidge\negsmidge\negsmidge\cdot\negsmidge\negsmidge\negsmidge}
\def\union{\cup}
\def\sep{\,}
\def\bigunion{\bigcup}
\def\suchthatt{\: :\:}
%\def\next{\hspace{12pt};\hspace{12pt}}
\def\nextt{\hspace{3pt};\hspace{6pt}}
\newcommand{\set}[2]{\{\:#1 \suchthatt #2\:\}}
\newcommand{\card}[1]{|#1|}
\def\leqq{\;\leq\;}
\def\eqq{\;=\;}
\def\geqq{\;\geq\;}
\def\lst{\;<\;}
\def\gst{\;>\;}
\def\prn#1{\left(#1\right)}

% \newcolumntype{y}[1]{%
% >{\hspace{0pt}}p{#1}}%

% \newcolumntype{x}[1]{%
% >{\centering\hspace{0pt}}p{#1}}%




\newcommand{\verylongleftarrow}[1]
      {\setlength{\unitlength}{.01in}
      \begin{picture}(#1,1) \put(#1,0){\vector(-1,0){#1}} \end{picture}}
\newcommand{\verylongrightarrow}[1]             %longleft and rightgoing arrows
      {\setlength{\unitlength}{.01in}           %for protocols
      \begin{picture}(#1,1) \put(0,0){\vector(1,0){#1}} \end{picture}}
\newcommand{\verylongbotharrow}[2]             %longleft and rightgoing arrows
      {\setlength{\unitlength}{.01in}           %for protocols
      \begin{picture}(#2,1) \put(#1,0){\vector(1,0){#1}}
                            \put(#1,0){\vector(-1,0){#1}} \end{picture}}
\newcommand{\leftgoing}[2]{{\stackrel{{\displaystyle #2}} {\verylongleftarrow{#1}}}}
\newcommand{\rightgoing}[2]{{\stackrel{{\displaystyle #2}} {\verylongrightarrow{#1}}}}
\newcommand{\bothgoing}[3]{{\stackrel{{\displaystyle #3}} {\verylongbotharrow{#1}{#2}}}}


\newcommand{\leftgoinga}[1]{\leftgoing{230}{#1} }
\newcommand{\rightgoinga}[1]{\rightgoing{230}{#1} }

\newcommand{\leftgoingb}[1]{\leftgoing{300}{#1} }
\newcommand{\rightgoingb}[1]{\rightgoing{300}{#1} }




\newcommand{\veryshortleftarrow}[1]
      {\setlength{\unitlength}{.01in}
      \begin{picture}(#1,1) \put(#1,0){\vector(-1,0){#1}} \end{picture}}


\newcommand{\getparse}[1]{{\:\stackrel{\raisebox{-0.5em}{{\hspace{0.1em}\mbox{\boldmath$\scriptscriptstyle
            #1$}}}}{\leftarrow}\:}}
\newcommand{\getu}{{\:\stackrel{\raisebox{-0.5em}{{\hspace{0.1em}\mbox{\boldmath$\scriptscriptstyle \cup$}}}}{\leftarrow}\:}}
\newcommand{\getdiff}{{\:\stackrel{{\scriptscriptstyle\hspace{0.2em} /}}{\leftarrow}\:}}
\newcommand{\getsr}{{\:{\leftarrow{\hspace*{-3pt}\raisebox{.75pt}{$\scriptscriptstyle\$$}}}\:}}
\newcommand{\get}[1]{{\:\leftarrow_{#1}\:}}
%\newcommand{\getsr}{{\:{\raisebox{3pt}{\veryshortleftarrow{15}}{\raisebox{1pt}{$\scriptscriptstyle\$$}}}\:}}
%\newcommand{\getsr}{{\:{\xleftarrow{\scriptscriptstyle\$}}\:}}
%\newcommand{\getsr}{{\:\stackrel{\raisebox{-0.5em}{$\scriptscriptstyle \hspace{0.2em}\$$}}{\leftarrow}\:}}
\newcommand{\sendsr}{{\:\stackrel{\scriptscriptstyle \hspace{0.2em}\$}{\rightarrow}\:}}
\renewcommand{\choose}[2]{{{#1}\atopwithdelims(){#2}}}
\newcommand{\abs}[1]{{\displaystyle \left| {#1} \right| }}
\newcommand{\E}{{\mbox{\bf E}}}
\newcommand{\EE}[1]{{\E\left[{#1}\right]}}
\newcommand{\EEE}[2]{{\E_{#1}\left[{#2}\right]}}
\newcommand{\Ex}{{\textbf E}}
\newcommand{\Exx}[1]{{\Ex\left[{#1}\right]}}
\newcommand{\Exxx}[2]{{\Ex_{#1}\left[{#2}\right]}}
\newcommand{\Var}{{\textnormal{Var}}}
\newcommand{\Varr}[1]{{\Var\left[{#1}\right]}}
\newcommand{\Varrr}[2]{{\Var_{#1}\left[{#2}\right]}}
\newcommand{\Prob}[1]{\Pr\left[\: #1 \:\right]}
\newcommand{\CondProb}[2]{{\Pr}\left[\: #1\:\left|\right.\:#2\:\right]}
\newcommand{\CondProbb}[3]{{\Pr}_{#1}\left[\: #2\:\left|\right.\:#3\:\right]}
\newcommand{\ProbExp}[2]{\Pr\left[\: #1 \: : \: #2\: \right]}
\newcommand{\Probb}[2]{{\Pr}_{#1}\left[\: #2 \:\right]}
\newcommand{\Probc}[2]{\Pr\left[\: #1 \:{\left|\right.}\:#2\:\right]}
\newcommand{\Probcc}[3]{{\Pr}_{#1}\left[\: #2 \:\left|\right.\:#3\:\right]}
\newcommand{\suchthat}{{\mbox{s.t.\ }}}
\newcommand{\qquadd}{{\quad}}
\def\d{{\delta}}
\def\e{{\epsilon}}
\newcommand{\ceiling}[1]{\lceil #1\rceil}
\newcommand{\sfrac}[2]{{\textstyle \frac{#1}{#2}}}
\newcommand{\ssum}[2]{{\textstyle \sum_{\,#1}^{\,#2}\,}}
\newcommand{\sprod}[2]{{\textstyle \prod_{\,#1}^{\,#2}\,}}
\def\smax{{\textstyle \max}}
%\def\N{{\sf N}}
\def\R{{\sf R}}
%\def\getsr{\stackrel{\$}{\leftarrow}}
\newcommand{\blockindex}[2]{{\langle#1\rangle}_{#2}}
\def\chv{\raisebox{2pt}{$\chi$}}

\newcommand{\cclass}[1]{{\rm #1}}
\def\P{\cclass{P}}
\def\NP{\cclass{NP}}
\def\BPP{\cclass{BPP}}
\def\coRP{\cclass{coRP}}
\def\NEXP{\cclass{NEXP}}
\def\DES{\mbox{\rm DES}}
\newcommand{\md}{\textsf{md5}}
\newcommand{\MD}{\textnormal{MD}}
\newcommand{\MDb}{\textbf{MD}}
\newcommand{\sha}{\textsf{sha-1}}
\newcommand{\ripemd}{\textsf{ripemd-160}}

\newcommand{\badSet}{\calS}
\newcommand{\bad}{\ensuremath{\mathsf{bad}}}
\newcommand{\cbad}{\bad}
\newcommand{\ctrue}{\true}
\newcommand{\notbad}{\overline{\bad}}
\newcommand{\win}{\ensuremath{\mathsf{win}}}
\newcommand{\outputs}{\:{\Rightarrow}\:}
\newcommand{\cheat}{\mathsf{cheat}}
\newcommand{\cheattrue}{\cheat\gets\true}
\newcommand{\notcheated}{\mathsf{Honest}}
\newcommand{\badtrue}{\bad\gets\true}
\newcommand{\Good}{\mathsf{Good}}
\newcommand{\good}{\mathsf{good}}
\newcommand{\EvNbadA}{\overline{\mathsf{Bad}}_1}
\newcommand{\EvbadA}{\mathsf{Bad}_1}
\newcommand{\EvNbadB}{\overline{\mathsf{Bad}}_2}
\newcommand{\EvbadB}{\mathsf{Bad}_2}
\newcommand{\badA}{\bad_1}
\newcommand{\badB}{\bad_2}
\newcommand{\badC}{\bad_3}
\newcommand{\badAtrue}{\badA\gets\true}
\newcommand{\badBtrue}{\badB\gets\true}
\newcommand{\badCtrue}{\badC\gets\true}
\newcommand{\setsbad}{\mbox{~\textup{sets} $\bad$}\,}
\newcommand{\setsbadA}{\mbox{~\textup{sets} $\badA$}\,}
\newcommand{\setsbadB}{\mbox{~\textup{sets} $\badB$}\,}
\newcommand{\setsbadzero}{\mbox{~\textup{sets} $\badzero$}\,}
\newcommand{\setsbadone}{\mbox{~\textup{sets} $\badone$}\,}
\newcommand{\setsbadtwo}{\mbox{~\textup{sets} $\badtwo$}\,}
\newcommand{\setsbadthree}{\mbox{~\textup{sets} $\badthree$}\,}
\newcommand{\setsbadfour}{\mbox{~\textup{sets} $\badfour$}\,}
\newcommand{\badzero}{\ensuremath{\mathsf{bad}_0}}
\newcommand{\badone}{\ensuremath{\mathsf{bad}_1}}
\newcommand{\badtwo}{\ensuremath{\mathsf{bad}_2}}
\newcommand{\badthree}{\ensuremath{\mathsf{bad}_3}}
\newcommand{\badfour}{\ensuremath{\mathsf{bad}_4}}
\newcommand{\defined}{\ensuremath{\mathsf{defined}}}
%\newcommand{\undefined}{\ensuremath{\mathsf{undefined}}}
\newcommand{\true}{\ensuremath{\mathsf{true}}}
\newcommand{\false}{\ensuremath{\mathsf{false}}}
\newcommand{\zero}{\ensuremath{\mathsf{zer}}}
\newcommand{\one}{\ensuremath{\mathsf{one}}}
\newcommand{\Initialize}{{\textbf{Initialize}}}
\newcommand{\Finalize}  {{\textbf{Finalize}}}
\newcommand{\Onquery}   {{\textbf{procedure~}}}
\newcommand{\onquery}   {{\textbf{query~}}}
\newcommand{\Update}{{\textnormal{update}}}
%\newcommand{\algorithm}[1]{{\textbf{algorithm~}{#1}}}
\newcommand{\fn}{\footnotesize}

\newcommand{\hash}{\mathcal{H}}

\newcommand{\Hash}{\textbf{Hash}}
\newcommand{\fEval}{\textbf{f-Eval}}
\newcommand{\fReveal}{\textbf{f-Reveal}}

\newcommand{\xRV}{X_{t,i}}
\newcommand{\XRV}{X_t}
\newcommand{\zRV}{Z_{t\concat r}}
\newcommand{\ZRV}{Z}
\newcommand{\const}{\mathsf{c}}

% from /usr/local/lib/TeX+MF/tex/macros/art11.sty

\makeatletter

\def\subsubsection{\@startsection{subsubsection}{3}{\z@}{-2.25ex plus
 -1ex minus -.2ex}{1.5ex plus .2ex}{\sf}}


% =========================================================================


% Definitions for this paper
\newcommand{\secparam}{\kappa}
\newcommand{\ctr}{\mathit{ctr}}
\newcommand{\MAC}{\mathsf{MAC}}
\newcommand{\kwfont}[1]{\textrm{#1}}
\newcommand{\procfont}[1]{\textbf{#1}}
\newcommand{\nfont}[1]{{\footnotesize #1}}
\newcommand{\lnum}[1]{{\footnotesize #1}\;\;}

\newcommand{\ETS}[2]{\cal{ETS}_{#1}^{#2}}

\newcommand{\Exp}{\mathbf{Exp}}
\newcommand{\MUExp}[4]{\mathbf{Exp}^{\mathrm{n \mbox{-}mu\mbox{-}#4\mbox{-}#1}}_{#2}(#3)}
%\newcommand{\SUExp}[2]{\mathbf{Exp}^{\mathrm{su\mbox{-}cpa\mbox{-}#2}}_{#1}}
\newcommand{\SUExp}[3]{\mathbf{Exp}^{\mathrm{ind\mbox{-}{#3}}}_{#1}(#2)}
\newcommand{\ForgeExp}[2]{\mathbf{Exp}^{\mathrm{uf\mbox{-}cma}}_{#1}(#2)}
\newcommand{\LeakExp}[3]{\mathbf{Exp}^{\mathrm{pred\mbox{-}ct\mbox{-}#3}}_{#1}(#2)}
\newcommand{\CompExp}[3]{\mathbf{Exp}^{\mathrm{pred\mbox{-}pt\mbox{-}#3}}_{#1}(#2)}
\newcommand{\FuncExp}[2]{\mathbf{Exp}^{\mathrm{ss\mbox{-}real}}_{#1}(#2)}
\newcommand{\FuncExpReal}[2]{\mathbf{Exp}^{\mathrm{ror\mbox{-}det\mbox{-}real}}_{#1}(#2)}
\newcommand{\FuncExpRand}[2]{\mathbf{Exp}^{\mathrm{ror\mbox{-}det\mbox{-}rand}}_{#1}(#2)}
\newcommand{\FuncExpRealH}[2]{\mathbf{Exp}^{\mathrm{hyb\mbox{-}det\mbox{-}real}}_{#1}(#2)}
\newcommand{\FuncExpRandH}[2]{\mathbf{Exp}^{\mathrm{hyb\mbox{-}det\mbox{-}rand}}_{#1}(#2)}
\newcommand{\FuncExpD}[2]{\mathbf{Exp}^{\mathrm{det\mbox{-}1}}_{#1}(#2)}
\newcommand{\FuncExpS}[3]{\mathbf{Exp}^{\mathrm{priv\mathrm{\mbox{-}#3}\mbox{-}1}}_{#1}(#2)}
\newcommand{\FuncExpDR}[2]{\mathbf{Exp}^{\mathrm{ss\mbox{-}det\mbox{-}rand}}_{#1}(#2)}
\newcommand{\FuncPTExp}[2]{\mathbf{Exp}^{\mathrm{func\mbox{-}pt}}_{#1}(#2)}
\newcommand{\PlainFuncExp}[3]{\mathbf{Exp}^{\mathrm{func\mbox{-}nil\mbox{-}#3}}_{#1}(#2)}
\newcommand{\FuncPredExp}[2]{\mathbf{Exp}^{\mathrm{ss\mbox{-}sim}}_{#1}(#2)}
\newcommand{\FuncPredExpD}[2]{\mathbf{Exp}^{\mathrm{det\mbox{-}0}}_{#1}(#2)}
\newcommand{\FuncPredExpS}[3]{\mathbf{Exp}^{\mathrm{priv\mathrm{\mbox{-}#3}\mbox{-}0}}_{#1}(#2)}
\newcommand{\SSExp}[2]{\mathbf{Exp}^{\mathrm{ss}\mbox{-}\mathrm{real}}_{#1}(#2)}
\newcommand{\SSExpS}[2]{\mathbf{Exp}^{\mathrm{ss\mbox{-}sim}}_{#1}(#2)}
\newcommand{\InvertExp}[2]{\mathbf{Exp}^{\mathrm{pt\mbox{-}pred}}_{#1}(#2)}
\newcommand{\IndOracle}[2]{\mathbf{I}_{#1}(#2)}
\newcommand{\BOracle}{\mathsf{B}}
\newcommand{\EqOracle}{\mathsf{Eq}}
\newcommand{\REC}{\mathsf{REC}}
\newcommand{\notask}{\overline{\mathsc{Ask}}}
\newcommand{\ask}{\mathsc{Ask}}
\newcommand{\MQ}{\mathsf{MsgQueried}}
\newcommand{\MG}{\mathsf{QueryOfMsgGuessed}}
\newcommand{\MC}{\mathsf{MsgChosenAgain}}
\newcommand{\MT}{\mathsf{MsgTargeted}}
\newcommand{\AT}{\mathsf{AdvFindsTarget}}
\newcommand{\ME}{\mathsf{MsgsAreEqual}}
\newcommand{\QM}{\mathsf{AskM}}
\newcommand{\QR}{\mathsf{AskM_r}}
\newcommand{\TG}{\mathsf{CorrGuess}}
\newcommand{\flips}{n_{\text{f}}}
\newcommand{\mprime}{m^{\prime}}

\newcommand{\queries}{q}

\newcommand{\Aguess}{A_{\mathrm{g}}}
\newcommand{\Ad}{A_{\mathrm{d}}}
\newcommand{\Ap}{A_{\mathrm{p}}}
\newcommand{\Af}{A_{\mathrm{f}}}
\newcommand{\Ag}{A_{\mathrm{g}}}
\newcommand{\Am}{A_{\mathrm{m}}}
\newcommand{\Ac}{A_{\mathrm{c}}}
\newcommand{\Ai}{A_{\mathrm{i}}}



\newcommand{\Aone}{A_1}
\newcommand{\Atwo}{A_2}
\newcommand{\Done}{D_1}
\newcommand{\Dtwo}{D_2}

\newcommand{\Agone}{A^*_{\mathrm{g}}}
\newcommand{\Amone}{A^*_{\mathrm{m}}}
\newcommand{\Acone}{A^*_{\mathrm{c}}}

\newcommand{\Astar}{A^*}
\newcommand{\Agstar}{A^*_{\mathrm{g}}}
\newcommand{\Amstar}{A^*_{\mathrm{m}}}
\newcommand{\Acstar}{A^*_{\mathrm{c}}}

\newcommand{\Ig}{I_{\mathrm{g}}}
\newcommand{\Iguess}{\Ig}
\newcommand{\Imsg}{I_{\mathrm{m}}}
\newcommand{\Ic}{I_{\mathrm{c}}}
\newcommand{\Ip}{I_{\mathrm{p}}}
\newcommand{\Is}{I_{\mathrm{s}}}

\newcommand{\Istar}{I^*}
\newcommand{\Igstar}{I^*_{\mathrm{g}}}
\newcommand{\Imstar}{I^*_{\mathrm{m}}}
\newcommand{\Icstar}{I^*_{\mathrm{c}}}


\newcommand{\Jm}{J_{\mathrm{m}}}
\newcommand{\Jg}{J_{\mathrm{g}}}
\newcommand{\Jc}{J_{\mathrm{c}}}
\newcommand{\Jp}{J_{\mathrm{p}}}
\newcommand{\Js}{J_{\mathrm{s}}}

\newcommand{\Dm}{D_{\mathrm{m}}}
\newcommand{\Dc}{D_{\mathrm{c}}}
\newcommand{\Dp}{D_{\mathrm{p}}}
\newcommand{\Dg}{D_{\mathrm{g}}}

\newcommand{\Pg}{P_{\mathrm{g}}}
\newcommand{\Pp}{P_{\mathrm{p}}}
\newcommand{\Pt}{P_{\mathrm{t}}}
\newcommand{\Pguess}{\Pg}
\newcommand{\Pmsg}{P_{\mathrm{m}}}
\newcommand{\Pm}{\Pmsg}
\newcommand{\Pc}{P_{\mathrm{c}}}
\newcommand{\Pd}{P_{\mathrm{d}}}


\newcommand{\Bc}{B_{\mathrm{c}}}
\newcommand{\Bf}{B_{\mathrm{f}}}
\newcommand{\Bg}{B_{\mathrm{g}}}
\newcommand{\Bm}{B_{\mathrm{m}}}
\newcommand{\Bi}{B_{\mathrm{i}}}

\newcommand{\SSAdv}{\calA_{\mathrm{SS}}}
\newcommand{\TrivAdv}{\calA_\emptystr}
%\newcommand{\LegAdv}{\calA_L}
%\newcommand{\EQAdv}{\calA_{\mathrm{E}}}
\newcommand{\BoolAdv}{\calA_{\mathrm{B}}}
\newcommand{\BBAdv}{\calA_{\mathrm{BB}}^\delta}
\newcommand{\MEAdv}{\calA_{\mathrm{ME}}^\mu}
\newcommand{\UMEAdv}{\calA_{\mathrm{UN}}}
\newcommand{\IMAdv}{\calA_{\times}}
\newcommand{\AdvSep}{\calA_{\mathrm{sep}}}

\newcommand{\AdvJell}{\calJ_{\ell}}
\newcommand{\AdvJzero}{\calJ_{0}}
\newcommand{\AdvIell}{\calJ_{\ell}}
\newcommand{\AdvDell}{\calD_{\ell}}

\newcommand{\AdvJme}{\calJ_{\mathrm{ME}}^\mu}
\newcommand{\AdvDme}{\calD_{\mathrm{ME}}^\mu}

\newcommand{\MEAdvv}[1]{\calA_{\mathrm{E}}^{#1}}
\newcommand{\HEAdv}{\calA_{\mathrm{HE}}}
\newcommand{\SMEAdv}{\calA_{\mathrm{SME}}^\mu}

\newcommand{\INDCPA}{\textnormal{IND-CPA}\xspace}
\newcommand{\INDCCA}{\textnormal{IND-CCA}\xspace}
\newcommand{\INDRAND}{\textnormal{IND\$}\xspace}
\newcommand{\INDSIM}{\textnormal{IND-SIM}\xspace}

\newcommand{\INDAdv}{\calI}
\newcommand{\TrivAdvI}{\calI_\emptystr}
\newcommand{\MEAdvI}{\calI_{\mathrm{ME}}^\mu}
\newcommand{\MEAdvvI}[1]{\calI_{\mathrm{ME}}^{#1}}
\newcommand{\HEAdvI}{\calI_{\mathrm{HE}}}
\newcommand{\SMEAdvI}{\calI_{\mathrm{SME}}^\mu}

\newcommand{\setZero}{\mathtt{I}_0}

%\newcommand{\MUExpcca}[2]{\mathbf{Exp}^{\mathrm{n \mbox{-}mu\mbox{-}cca\mbox{-}#2}}_{#1}}
%\newcommand{\SUExpcca}[2]{\mathbf{Exp}^{\mathrm{su\mbox{-}cca}\mbox{-}{#2}}_{#1}}
\newcommand{\DDHRealexp}[2]{\mathbf{Exp}^{\mathrm{ddh\mbox{-}real}}_{#1}(#2)}
\newcommand{\DDHRandexp}[2]{\mathbf{Exp}^{\mathrm{ddh\mbox{-}rand}}_{#1}(#2)}

\newcommand{\HyExp}{\mathbf{ExpH}}
\newcommand{\find}{\mathsf{find}}
\newcommand{\guess}{g}

\newcommand{\wPRF}{\mathsf{wPRF}}

\newcommand{\Succ}{\mathsf{Succ}}
\newcommand{\Adv}{\mathbf{Adv}}
\newcommand{\Advcpa}[1]{\Adv^{\mathrm{ind\mbox{-}cpa}}_{#1}}
\newcommand{\Advcca}[1]{\Adv^{\mathrm{ind\mbox{-}cca}}_{#1}}
\newcommand{\AdvBIND}[2]{\Adv^{\mathrm{r\dash bind}}_{#1}(#2)}
\newcommand{\AdvRBIND}[2]{\Adv^{\mathrm{r\dash bind}}_{#1}(#2)}
\newcommand{\AdvSRBIND}[2]{\Adv^{\mathrm{sr\dash bind}}_{#1}(#2)}

\newcommand{\AdvVFROB}[2]{\Adv^{\mathrm{efrob}}_{#1}(#2)}
\newcommand{\AdvSBIND}[2]{\Adv^{\mathrm{s\dash bind}}_{#1}(#2)}
\newcommand{\AdvDBIND}[2]{\Adv^{\mathrm{t\dash bind}}_{#1}(#2)}
\newcommand{\AdvMRBIND}[2]{\Adv^{\mathrm{mr\dash bind}}_{#1}(#2)}
\newcommand{\AdvVBIND}[2]{\Adv^{\mathrm{v\dash bind}}_{#1}(#2)}
\newcommand{\AdvNAE}[1]{\Adv^{\mathrm{dae}}_{#1}}
\newcommand{\AdvFROB}[2]{\Adv^{\mathrm{frob}}_{#1}(#2)}
\newcommand{\AdvOTROR}[2]{\Adv^{\mathrm{ot \dash ror}}_{#1}(#2)}
\newcommand{\AdvOTCTXT}[2]{\Adv^{\mathrm{scu}}_{#1}(#2)}
\newcommand{\AdvotCTXT}[2]{\Adv^{\mathrm{ot-ctxt}}_{#1}(#2)}
\newcommand{\AdvTBC}[2]{\Adv^{\mathrm{t\dash sprp}}_{#1}(#2)}

\newcommand{\AdvMEANBIND}[1]{\Adv^{\mathrm{mbind}}_{#1}}
\newcommand{\Coins}{\mathsf{Coins}}
\newcommand{\inv}{^{-1}}
 \newcommand{\AdvPOW}[2]{\Adv^{\mathrm{pow}}_{#1}(#2)}
 \newcommand{\ExpPOW}[2]{\mathbf{Exp}^{\mathrm{pow}}_{#1}(#2)}
 \newcommand{\ExpPOWW}[3]{\mathbf{Exp}^{\mathrm{pow\mbox{-}#2}}_{#1}(#3)}

 \newcommand{\AdvPRIV}[2]{\Adv^{\mathrm{priv}}_{#1}(#2)}
 \newcommand{\ExpPRIV}[2]{\mathbf{Exp}^{\mathrm{priv}}_{#1}(#2)}
 \newcommand{\ExpPRIVV}[3]{\mathbf{Exp}^{\mathrm{priv\mbox{-}#2}}_{#1}(#3)}

 \newcommand{\XCSS}{\textnormal{X-CSS}\xspace}
 \newcommand{\XSSS}{\textnormal{X-SSS}\xspace}
 \newcommand{\CSS}{\textnormal{CSS}\xspace}
 \newcommand{\FCSS}{\textnormal{A-CSS}\xspace}
 \newcommand{\BCSS}{\textnormal{B-CSS}\xspace}
 \newcommand{\BBCSS}{\textnormal{BB-CSS}\xspace}
 \newcommand{\BBCSSbf}{\textbf{BBCSS}\xspace}
 \newcommand{\CSSbf}{\textbf{CSS}\xspace}
 \newcommand{\AdvCSS}[2]{\Adv^{\mathrm{css}}_{#1}(#2)}
 \newcommand{\AdvBBCSS}[2]{\Adv^{\mathrm{bbcss}}_{#1}(#2)}
 \newcommand{\ExpCSS}[2]{\mathbf{Exp}^{\mathrm{css}}_{#1}(#2)}
 \newcommand{\ExpCSSS}[3]{\mathbf{Exp}^{\mathrm{css\mbox{-}#2}}_{#1}(#3)}

 \newcommand{\SSS}{\textnormal{SSS}\xspace}
 \newcommand{\SSSbf}{\textbf{SSS}\xspace}
 \newcommand{\FSSS}{\textnormal{A-SSS}\xspace}
 \newcommand{\BSSS}{\textnormal{B-SSS}\xspace}
 \newcommand{\BBSSS}{\textnormal{BB-SSS}\xspace}
 \newcommand{\BBSSSbf}{\textbf{BBSSS}\xspace}
 \newcommand{\AdvSSS}[2]{\Adv^{\mathrm{sss}}_{#1}(#2)}
 \newcommand{\ExpSSS}[2]{\mathbf{Exp}^{\mathrm{sss}}_{#1}(#2)}
 \newcommand{\ExpSSSS}[3]{\mathbf{Exp}^{\mathrm{sss\mbox{-}#2}}_{#1}(#3)}
 \newcommand{\prf}{F}
 \newcommand{\dash}{\mbox{-}}
 \newcommand{\IND}{\textnormal{IND}\xspace}
 \newcommand{\INDbf}{\textbf{IND}\xspace}
 \newcommand{\ExpSPRP}[2]{\mathbf{Exp}^{\mathrm{sprp}}_{#1}(#2)}
\newcommand{\AdvPRPprime}[2]{\Adv^{\mathrm{prp'}}_{#1}(#2)}
 \newcommand{\ExpPRPprime}[2]{\mathbf{Exp}^{\mathrm{prp'}}_{#1}(#2)}
 \newcommand{\AdvNPRG}[2]{\Adv^{\mathrm{prg}}_{#1}(#2)}
 %\newcommand{\AdvOTPRF}[2]{\Adv^{\mathrm{ot\dash prf}}_{#1}(#2)}
 \newcommand{\AdvMUPRF}[2]{\Adv^{\mathrm{mu\dash prf}}_{#1}(#2)}
 \newcommand{\ExpPRF}[2]{\mathbf{Exp}^{\mathrm{prf}}_{#1}(#2)}
 \newcommand{\AdvIND}[2]{\Adv^{\mathrm{ind}}_{#1}(#2)}
 \newcommand{\AdvINDCCA}[2]{\Adv^{\mathrm{ind\dash cca}}_{#1}(#2)}
 \newcommand{\ExpIND}[2]{\mathbf{Exp}^{\mathrm{ind}}_{#1}(#2)}
 \newcommand{\ExpINDD}[3]{\mathbf{Exp}^{\mathrm{ind\dash#2}}_{#1}(#3)}
 \newcommand{\ExpINDCCA}[2]{\mathbf{Exp}^{\mathrm{ind\dash cca}}_{#1}(#2)}
 \newcommand{\ExpINDCCAD}[3]{\mathbf{Exp}^{\mathrm{ind\dash cca\dash#2}}_{#1}(#3)}

 \newcommand{\AdvKEMCCA}[2]{\Adv^{\mathrm{kem\dash cca}}_{#1}(#2)}
 \newcommand{\AdvKEMCPA}[2]{\Adv^{\mathrm{kem}}_{#1}(#2)}
 \newcommand{\ExpKEMCCA}[2]{\mathbf{Exp}^{\mathrm{kem\dash cca}}_{#1}(#2)}
 \newcommand{\ExpKEMCPA}[2]{\mathbf{Exp}^{\mathrm{kem}}_{#1}(#2)}

 \newcommand{\AdvLOR}[2]{\Adv^{\mathrm{lor}}_{#1}(#2)}
 \newcommand{\AdvLORCCA}[2]{\Adv^{\mathrm{lor\dash cca}}_{#1}(#2)}
 \newcommand{\ExpLOR}[2]{\mathbf{Exp}^{\mathrm{lor}}_{#1}(#2)}
 \newcommand{\ExpLORCCA}[2]{\mathbf{Exp}^{\mathrm{lor\dash cca}}_{#1}(#2)}
 \newcommand{\Enc}{\procfont{Enc}}

 \newcommand{\Ectxtspace}{\calC}
% \newcommand{\coinspace}{\calR}
 \newcommand{\ctxtspacebar}{\overline{\calC}}
 \newcommand{\Frank}{\calF}
 \newcommand{\metadata}{\text{MD}}
 \newcommand{\Dec}{\procfont{Dec}}
 \newcommand{\Tag}{\procfont{Tag}}
 \newcommand{\Report}{\procfont{Report}}
 \newcommand{\Chk}{\procfont{Check}}
 \newcommand{\lrproc}{\procfont{Enc}}
 \newcommand{\decproc}{\procfont{Dec}}
 \newcommand{\conceal}{\textsf{conceal}}
 \newcommand{\conopen}{\textsf{open}}
 \newcommand{\CON}{\mathcal{C}}
 \newcommand{\hider}{h}
 \newcommand{\binder}{b}
% \newcommand{\CKgtrans}{\CKg_{\EC, \AEAD}}
% \newcommand{\CEnctrans}{\CEnc_{\EC, \AEAD}}
%  \newcommand{\CDectrans}{\CDec_{\EC, \AEAD}}
%  \newcommand{\CVertrans}{\CVer_{\EC, \AEAD}}
\newcommand{\CKgtrans}{\CKg}
\newcommand{\CEnctrans}{\CEnc}
\newcommand{\CDectrans}{\CDec}
\newcommand{\CEnccmpr}{\CEnc}
\newcommand{\CDeccmpr}{\CDec}
\newcommand{\CEnctbc}{\CEnc}
\newcommand{\CDectbc}{\CDec}
\newcommand{\CVertrans}{\CVer}
\newcommand{\afenc}{\text{FAF\dash Enc}}
\newcommand{\afdec}{\text{FAF\dash Dec}}
\newcommand{\afver}{\text{FAF\dash Ver}}
\newcommand{\GetID}{\text{GetID}}
\newcommand{\AdvWPRF}[2]{\Adv^{\mathrm{wPRF}}_{#1}(#2)}
\newcommand{\AdvINDrCPA}[2]{\Adv^{\mathrm{ind\$\dash cpa}}_{#1}(#2)}
\newcommand{\ExpINDrCPA}[2]{\mathbf{Exp}^{\mathrm{ind\$\dash cpa}}_{#1}(#2)}
\newcommand{\AdvLHINDrCPA}[2]{\Adv^{\mathrm{lh\dash ind\$\dash cpa}}_{#1}(#2)}
\newcommand{\ExpLHINDrCPA}[2]{\mathbf{Exp}^{\mathrm{lh\dash ind\$\dash cpa}}_{#1}(#2)}

\newcommand{\AdvINDCPA}[2]{\Adv^{\mathrm{ind\dash cpa}}_{#1}(#2)}
\newcommand{\ExpINDCPA}[2]{\mathbf{Exp}^{\mathrm{ind\dash cpa}}_{#1}(#2)}
\newcommand{\AdvLHINDCPA}[2]{\Adv^{\mathrm{lh\dash ind\dash cpa}}_{#1}(#2)}
\newcommand{\ExpLHINDCPA}[2]{\mathbf{Exp}^{\mathrm{lh\dash ind\dash cpa}}_{#1}(#2)}

\newcommand{\AdvAE}[2]{\Adv^{\mathrm{ror}}_{#1}(#2)}
\newcommand{\AdvMOROR}[2]{\Adv^{\mathrm{mo\dash ror}}_{#1}(#2)}
\newcommand{\AdvNMOROR}[2]{\Adv^{\mathrm{mo\dash nror}}_{#1}(#2)}
\newcommand{\AdvMORORctxt}[2]{\Adv^{\mathrm{mo\dash ror\dash ctxt}}_{#1}(#2)}
\newcommand{\AdvNMORORctxt}[2]{\Adv^{\mathrm{mo\dash nror\dash ctxt}}_{#1}(#2)}
\newcommand{\AdvSOROR}[2]{\Adv^{\mathrm{ror}}_{#1}(#2)}
\newcommand{\AdvSORORctxt}[2]{\Adv^{\mathrm{ror\dash ctxt}}_{#1}(#2)}
\newcommand{\AdvMOCTXT}[2]{\Adv^{\mathrm{mo\dash ctxt}}_{#1}(#2)}
\newcommand{\AdvNMOCTXT}[2]{\Adv^{\mathrm{mo\dash nctxt}}_{#1}(#2)}
\newcommand{\AdvSOCTXT}[2]{\Adv^{\mathrm{ctxt}}_{#1}(#2)}
\newcommand{\AdvCSCTXT}[2]{\Adv^{\mathrm{cs\dash nm}}_{#1}(#2)}
\newcommand{\AdvCSROR}[2]{\Adv^{\mathrm{cs\dash ror}}_{#1}(#2)}
\newcommand{\AdvAElor}[2]{\Adv^{\mathrm{ae\dash lor}}_{#1}(#2)}
\newcommand{\AdvLHAE}[2]{\Adv^{\mathrm{lh\dash ae}}_{#1}(#2)}
\newcommand{\AdvLHAElor}[2]{\Adv^{\mathrm{lh \dash ae\dash lor}}_{#1}(#2)}

\newcommand{\OTRORReal}{\text{otROR0}}
\newcommand{\OTRORRand}{\text{otROR1}}
\newcommand{\OTCTXT}{\text{SCU}}
\newcommand{\OTROR}{\text{otROR}}
\newcommand{\lens}{\textsf{lens}}
\newcommand{\eccomlen}{\textsf{btlen}}

\newcommand{\LRKAPRF}{\text{RKA-PRF}}
\newcommand{\LRKAPRFReal}{\text{RKA-PRF0}}
\newcommand{\LRKAPRFIdeal}{\text{RKA-PRF1}}
\newcommand{\AdvLRKAPRF}[2]{\Adv^{\mathrm{\oplus \dash prf}}_{#1}(#2)}

\newcommand{\LRKAPRP}{\text{RKA-PRP}}
\newcommand{\LRKAPRPReal}{\text{RKA-PRP0}}
\newcommand{\LRKAPRPIdeal}{\text{RKA-PRP1}}
\newcommand{\AdvLRKAPRP}[2]{\Adv^{\mathrm{\oplus \dash prp}}_{#1}(#2)}

\newcommand{\AEAD}{\textnormal{\textsf{AE}}}
%% \newcommand{\AEADone}{\textnormal{LHAE1}}
%% \newcommand{\AEADzero}{\textnormal{LHAE0}}
%% \newcommand{\AEADstate}{\textnormal{sLHAE}}
%% \newcommand{\AEADstateOne}{\textnormal{sLHAE1}}
%% \newcommand{\AEADstateZero}{\textnormal{sLHAE0}}
%% \newcommand{\AdvLHSTAE}[2]{\Adv^{\mathrm{lh\dash st\dash ae}}_{#1}(#2)}


\newcommand{\ExpAE}[2]{\mathbf{Exp}^{\mathrm{ae}}_{#1}(#2)}
\newcommand{\AdvCRD}[2]{\Adv^{\mathrm{crd}}_{#1}(#2)}
\newcommand{\ExpCRD}[2]{\mathbf{Exp}^{\mathrm{crd}}_{#1}(#2)}
\newcommand{\GameCRD}{\textnormal{CRD}}
\newcommand{\CRD}{\textnormal{CRD}}
\newcommand{\AdvRD}[2]{\Adv^{\mathrm{rp}}_{#1}(#2)}
\newcommand{\AdvRDIND}[2]{\Adv^{\mathrm{rp\dash ind}}_{#1}(#2)}
\newcommand{\ExpRD}[2]{\mathbf{Exp}^{\mathrm{rp}}_{#1}(#2)}
\newcommand{\ExpRDreal}[2]{\mathbf{Exp}^{\mathrm{rp\dash 1}}_{#1}(#2)}
\newcommand{\ExpRDideal}[2]{\mathbf{Exp}^{\mathrm{rp\dash 0}}_{#1}(#2)}
\newcommand{\RP}{\textnormal{RD}}


 \newcommand{\AdvKI}[2]{\Adv^{\mathrm{ki}}_{#1}(#2)}
\newcommand{\AdvINTPTXT}[2]{\Adv^{\mathrm{int\dash ptxt}}_{#1}(#2)}
\newcommand{\AdvPTXT}[2]{\Adv^{\mathrm{ptxt}}_{#1}(#2)}
\newcommand{\ExpINTPTXT}[2]{\mathbf{Exp}^{\mathrm{int\dash ptxt}}_{#1}(#2)}
\newcommand{\AdvINTCTXT}[2]{\Adv^{\mathrm{int\dash ctxt}}_{#1}(#2)}
\newcommand{\AdvCTXT}[2]{\Adv^{\mathrm{ctxt}}_{#1}(#2)}
\newcommand{\ExpINTCTXT}[2]{\mathbf{Exp}^{\mathrm{int\dash ctxt}}_{#1}(#2)}

\newcommand{\CTXT}{\textnormal{CTXT}}
\newcommand{\CTXTce}{\textnormal{CTXT}}
\newcommand{\PTXT}{\textnormal{PTXT}}

\newcommand{\UFCMA}{\textnormal{UF-CMA}}
\newcommand{\MUUFCMA}{\textnormal{MU-UF-CMA}}
\newcommand{\SUFCMA}{\textnormal{SUF-CMA}}

 \newcommand{\AdvMUUFCMA}[2]{\Adv^{\mathrm{mu\dash uf\dash cma}}_{#1}(#2)}
 \newcommand{\AdvUFCMA}[2]{\Adv^{\mathrm{uf\dash cma}}_{#1}(#2)}
 \newcommand{\AdvSUFCMA}[2]{\Adv^{\mathrm{suf\dash cma}}_{#1}(#2)}
 \newcommand{\AdvPRO}[2]{\Adv^{\mathrm{pro}}_{#1}(#2)}
 \newcommand{\AdvPROG}[2]{\Adv^{\mathrm{prog}}_{#1}(#2)}
 \newcommand{\AdvPUBPRO}[2]{\Adv^{\mathrm{pub\dash pro}}_{#1}(#2)}
 \newcommand{\AdvPUBPICF}[2]{\Adv^{\mathrm{pub\dash gpro}}_{#1}(#2)}
 \newcommand{\AdvWPA}[2]{\Adv^{\mathrm{wpra}}_{#1}(#2)}
 \newcommand{\AdvWPAone}[2]{\Adv^{\mathrm{wpra\dash 1}}_{#1}(#2)}
 \newcommand{\AdvINV}[2]{\Adv^{\mathrm{inv}}_{#1}(#2)}
 \newcommand{\AdvPA}[2]{\Adv^{\mathrm{pra}}_{#1}(#2)}
 \newcommand{\AdvVPA}[2]{\Adv^{\mathrm{v\dash pra}}_{#1}(#2)}
 \newcommand{\AdvOnePA}[2]{\Adv^{\mathrm{1\dash pra}}_{#1}(#2)}
 \newcommand{\AdvPAone}[2]{\Adv^{\mathrm{pa\dash 1}}_{#1}(#2)}
 \newcommand{\AdvEXT}[2]{\Adv^{\mathrm{ext}}_{#1}(#2)}
 \newcommand{\AdvEXTone}[2]{\Adv^{\mathrm{pra}}_{#1}(#2)}
 \newcommand{\AdvEXTtwo}[2]{\Adv^{\mathrm{ext}}_{#1}(#2)}
 \newcommand{\AdvCR}[2]{\Adv^{\mathrm{cr}}_{#1}(#2)}
 \newcommand{\AdvePRE}[2]{\Adv^{\mathrm{ePre}}_{#1}(#2)}
 \newcommand{\AdvrOWF}[2]{\Adv^{\mathrm{r\dash owf}}_{#1}(#2)}


\newcommand{\query}[1]{\procfont{query} {#1}:}
\newcommand{\queryl}[1]{\underline{\procfont{query} {#1}:}}
\newcommand{\oracle}[1]{\underline{\procfont{oracle} {#1}:}}
\newcommand{\oraclev}[1]{\underline{\procfont{oracle} {#1}:}\smallskip}
\newcommand{\procedure}[1]{\underline{\procfont{procedure} {#1}:}}
%\newcommand{\procedurev}[1]{\underline{\procfont{procedure} {#1}:}\smallskip}
\newcommand{\procedurev}[1]{\underline{{#1}:}\smallskip}
\newcommand{\subroutine}[1]{\underline{\procfont{subroutine} {#1}:}}
\newcommand{\subroutinev}[1]{\underline{\procfont{subroutine} {#1}:}\smallskip}
\newcommand{\subroutinenl}[1]{{\procfont{subroutine} {#1}:}}
\newcommand{\subroutinenlv}[1]{{\procfont{subroutine} {#1}:}\smallskip}
\newcommand{\adversary}[1]{\underline{\procfont{adversary} {#1}:}}
\newcommand{\adversaryv}[1]{\underline{\procfont{adversary} {#1}:}\smallskip}
\newcommand{\experiment}[1]{\underline{{#1}}}
\newcommand{\experimentv}[1]{\underline{{#1}}\smallskip}

\newcommand{\algorithm}[1]{\underline{\procfont{algorithm} {#1}:}}
\newcommand{\algorithmv}[1]{\underline{\procfont{algorithm} {#1}:}\smallskip}
%\newcommand{\experiment}[1]{\underline{\procfont{Experiment} {#1}}}
%\newcommand{\experimentv}[1]{\underline{\procfont{Experiment} {#1}}\smallskip}

\newcommand{\AdvMU}[3]{\Adv^{\mathrm{n \mbox{-}mu\mbox{-}#3}}_{#1}(#2)}
%\newcommand{\AdvSU}[2]{\Adv^{\mathrm{su\mbox{-}cpa}}_{#1}(#2)}
\newcommand{\AdvSU}[3]{\Adv^{\mathrm{ ind\mbox{-}#3}}_{#1}(#2)}
\newcommand{\AdvSUnok}[2]{\Adv^{\mathrm{ ind\mbox{-}#2}}_{#1}}
\newcommand{\AdvForge}[2]{\Adv^{\mathrm{uf\mbox{-}cma}}_{#1}(#2)}
\newcommand{\AdvPred}[2]{\Adv^{\mathrm{pred\mbox{-}ct}}_{#1}(#2)}
\newcommand{\AdvInvert}[2]{\Adv^{\mathrm{pt\mbox{-}pred}}_{#1}(#2)}
\newcommand{\AdvCompPred}[2]{\Adv^{\mathrm{pred\mbox{-}pt}}_{#1}(#2)}
\newcommand{\AdvFuncPred}[2]{\Adv^{\mathrm{func\mbox{-}pred}}_{#1}(#2)}
\newcommand{\AdvFunc}[2]{\Adv^{\mathrm{ss}\mbox{-}\mathrm{det}}_{#1}(#2)}
\newcommand{\AdvSS}[2]{\Adv^{\mathrm{ss}}_{#1}(#2)}
\newcommand{\AdvFuncD}[2]{\Adv^{\mathrm{det}}_{#1}(#2)}
\newcommand{\AdvFuncSnok}[2]{\Adv^{\mathrm{priv\mathrm{\mbox{-}#2}}}_{#1}}
\newcommand{\AdvFuncS}[3]{\Adv^{\mathrm{priv\mathrm{\mbox{-}#3}}}_{#1}(#2)}
\newcommand{\AdvFuncSS}[2]{\Adv^{\mathrm{priv\mathrm{\mbox{-}#2}}}_{#1}}
\newcommand{\AdvFuncH}[2]{\Adv^{\mathrm{hyb\mbox{-}det}}_{#1}(#2)}
\newcommand{\AdvFuncROR}[2]{\Adv^{\mathrm{ror\mbox{-}det}}_{#1}(#2)}
\newcommand{\AdvPlainFunc}[2]{\Adv^{\mathrm{func\mbox{-}nil}}_{#1}(#2)}
%\newcommand{\AdvMUcca}[2]{\Adv^{\mathrm{n \mbox{-}mu\mbox{-}cca}}_{#1}(#2)}
%\newcommand{\AdvSUcca}[2]{\Adv^{\mathrm{su\mbox{-}cca}}_{#1}(#2)}
\newcommand{\AdvOWF}[2]{\Adv^{\mathrm{owf}}_{{#1}}(#2)}
\newcommand{\AdvOWFnok}[1]{\Adv^{\mathrm{owf}}_{{#1}}}
\newcommand{\ExpOWF}[2]{\mathbf{Exp}^{\mathrm{owf}}_{#1}(#2)}
\newcommand{\AdvPOWF}[2]{\Adv^{\mathrm{powf}}_{{#1}}(#2)}
\newcommand{\AdvDDH}[2]{\Adv^{\mathrm{ddh}}_{#1}(#2)}
\newcommand{\AdvH}[2]{\Adv^{cr}_{\cal H, #1}(#2)}
\newcommand{\Hcollexp}[2]{\mathbf{Exp}^{\mathrm{cr}}_{#1}(#2)}
\newcommand{\Hcolladv}[2]{\Adv^{\mathrm{cr}}_{#1}(#2)}

\newcommand{\EG}{\text{EG}}


\newcommand{\st}{st}
\newcommand{\seqnum}{c}

\newcommand{\init}{{\sf Init}}
\newcommand{\LR}{\mathrm{LR}}
\newcommand{\SEscheme}{\textnormal{\textsf{SE}}}
\newcommand{\SEschemebar}{\overline{\textnormal{\textsf{SE}}}}
\newcommand{\MACscheme}{\mbox{\textsf{MAC}}}
\newcommand{\enc}{{\sf enc}}
\newcommand{\encr}{{\enc}^{\$}}
\newcommand{\dec}{{\sf dec}}
\newcommand{\encsign}{{\cal ES}}
%\newcommand{\sign}{{\sf S}}
\newcommand{\mac}{{\sf Mac}}
\newcommand{\verdecc}{{\cal VD}}
\newcommand{\henc}{{\cal HE}}
\newcommand{\hkg}{{\cal HK}}
\newcommand{\hf}{{\cal HF}}
\newcommand{\h}{{\cal H}}
\newcommand{\detenc}{{\cal DE}}
\newcommand{\denc}{{\cal DE}}
\newcommand{\senc}{{\cal SE}}
\newcommand{\sdecc}{{\cal SD}}
\newcommand{\decc}{{\cal D}}
\newcommand{\detdecc}{{\cal DD}}
\newcommand{\hdecc}{{\cal HD}}

\newcommand{\EN}{{\sf CODE}}
\newcommand{\encode}{{\sf encode}}
\newcommand{\decode}{{\sf decode}}
\newcommand{\AEnc}{\textsf{AE}}
\newcommand{\goodcode}{{\sf GENcode}}
\newcommand{\goodencode}{{\sf GENencode}}
\newcommand{\gooddecode}{{\sf GENdecode}}

\newcommand{\kg}{{\sf kg}}
\newcommand{\kgse}{\kg_{\mathrm{se}}}
\newcommand{\kgma}{\kg_{\mathrm{ma}}}
\newcommand{\kgc}{{\cal G}}


\newcommand{\kgSym}{\mathsf{kg_s}}
\newcommand{\encSym}{\mathsf{enc_s}}
\newcommand{\decSym}{\mathsf{dec_s}}

\newcommand{\msgsp}{\mathrm{MsgSp}}
\newcommand{\PKscheme}{\Pi} %{{\cal AE}} %replaced it according to other notation
\newcommand{\DSscheme}{{\cal DS}}
\newcommand{\hPKscheme}{{\cal HPE}}
\newcommand{\dPKscheme}{{\cal DPE}}
\newcommand{\sPKscheme}{{\cal SAE}}
\newcommand{\EGscheme}{{\cal EG}}
\newcommand{\CSscheme}{{\cal CS}}
\newcommand{\OAEPD}{\mathsf{DOAEP}}
\newcommand{\RSAOAEPD}{\mathsf{RSA\mbox{-}DOAEP}}
\newcommand{\cert}{\mathrm{cert}}
\newcommand{\pk}{\mathsl{pk}}
\newcommand{\pkbar}{\overline{\pk}}
\newcommand{\sk}{\mathsl{sk}}
\newcommand{\skbar}{\overline{\sk}}
\newcommand{\ke}{\mathsl{k_{e}}}
\newcommand{\kd}{\mathsl{k_{d}}}
\newcommand{\lr}{\mathrm{LR}}

\newcommand{\SE}{{\mathsf{SE}}}
\newcommand{\Skg}{{\mathsf{kg}}}
\newcommand{\Senc}{{\mathsf{enc}}}
\newcommand{\Sdec}{{\mathsf{dec}}}

\newcommand{\KEM}{\Psi}
\newcommand{\KEMkg}{{\cal KK}}
\newcommand{\KEMenc}{{\cal KE}}
\newcommand{\KEMdec}{{\cal KD}}

\def\next{\:;\:}
\newcommand{\concat}{\: \| \:}
\newcommand{\texp}{T_{\kgc}^{\mathrm{exp}}(k)}
\newcommand{\tgr}{T_{\kgc}^{\mathrm{gr-op}}(k)}
\newcommand{\tmem}{T_{\kgc}^{\mathrm{gr-mem}}(k)}
\newcommand{\tran}{T_{\kgc}^{\mathrm{rand}}(k)}
\newcommand{\qu}{q_{\mathrm{e}}}
\newcommand{\qd}{q_{\mathrm{d}}}
\newcommand{\qr}{q_{\mathrm{r}}}
\newcommand{\qh}{q_{\mathrm{h}}}
\newcommand{\qho}{q_{\mathrm{h_1}}}
\newcommand{\qht}{q_{\mathrm{h_2}}}
\newcommand{\qhi}{q_{\mathrm{h_i}}}
\newcommand{\lh}{l_{\mathrm{h}}}

\newcommand{\Bcpa}[1]{B_{\mathrm{cpa}}^{#1}}
\newcommand{\Bcca}[1]{B_{\mathrm{cca}}^{#1}}
\newcommand{\Dcca}[1]{D_{\mathrm{cca}}^{#1}}
\newcommand{\Acpa}[1]{A_{\mathrm{cpa}}^{#1}}
\newcommand{\Alg}[2]{A^{{#1}{#2}}}
\newcommand{\atk}{\mathrm{atk}}
\newcommand{\cpa}{\mathrm{cpa}}
\newcommand{\cca}{\mathrm{cca}}
\newcommand{\Aatk}[1]{A_{\mathrm{atk}}^{#1}}
\newcommand{\Batk}[1]{B_{\mathrm{atk}}^{#1}}
\newcommand{\Datk}[1]{D_{\mathrm{atk}}^{#1}}
\newcommand{\sasha}[1]{\textit{Sasha: #1}}
\newcommand{\GH}{\mathcal{GH}}
\newcommand{\EH}{\mathcal{EH}}
\newcommand{\GF}{\mathcal{GF}}
\newcommand{\gftwo}[1]{\text{GF}(2^{#1})}
\newcommand{\EF}{\mathcal{EF}}
\newcommand{\tmf}{\overline{\calG}}
\newcommand{\gh}{{\mathit{g}}}
\newcommand{\uh}{{\mathit{u}}}
\newcommand{\NR}{\mathsf{NR}}
\newcommand{\mc}{\text{mc}}
\newcommand{\Inv}{\sf{Inv}}
\newcommand{\rview}{\mathsf{View}}
\newcommand{\Acca}[1]{A^{#1}}

\newcommand{\adam}[1]{\textit{Adam: #1}}

\newcommand{\comment}[1]{\hspace{5pt}\textbf{[}\hspace{2pt}#1\textbf{]}}
% \ \ /$\!$/ {\small\textsl #1}}

\newcommand{\veca}{{{\bf a}}}
\newcommand{\vecb}{{{\bf b}}}
\newcommand{\vect}{{{\bf t}}}
\newcommand{\vecx}{{{\bf x}}}
\newcommand{\vecc}{{{\bf c}}}
\newcommand{\vece}{{{\bf e}}}
\newcommand{\vecK}{{{\bf K}}}
\newcommand{\vecv}{{{\bf v}}}
\newcommand{\vecy}{{{\bf y}}}
\newcommand{\vecz}{{{\bf z}}}
\newcommand{\vech}{{{\bf h}}}
\newcommand{\vecomega}{\boldsymbol{\omega}}
\newcommand{\test}{t}
\newcommand{\oguess}{g}
\newcommand{\me}[2]{\mathrm{me}_{#1}(#2)}
\newcommand{\menok}[1]{\mathrm{me}_{#1}}
%\newcommand{\inst}[1]{{}$^{#1}$}
%\newcommand{\email}[1]{\texttt{#1}}

\newcommand{\dqed}{{\hspace{5pt}\mbox{$\square$}}}
\newcommand{\SD}[1]{{\textnormal{SD}\left(#1\right)}}
\newcommand{\thh}{^{\textit{th}}} % th
\newcommand{\emptystr}{\varepsilon}
\newcommand{\emptyalg}{\Lambda}
\newcommand{\dotdot}{{\,..\,}}

\newcommand{\nk}{n}
\newcommand{\nt}{n}
\newcommand{\nm}{v}
\newcommand{\ml}{w}
\newcommand{\skl}{s}
\newcommand{\indless}{\hspace*{1em}}

%---Marc
\newcommand{\AdvPRGo}[2]{\Adv^{\mathrm{prg}}_{{#1}}({#2})}
\newcommand{\AdvPRG}[3]{\Adv^{\mathrm{prg}, #3}_{{#1}}({#2})}
\newcommand{\ExpPRGo}[2]{\mathbf{Exp}^{\mathrm{prg}}_{{#1}}({#2})}
\newcommand{\ExpPRG}[3]{\mathbf{Exp}^{\mathrm{prg},#3}_{{#1}}({#2})}
\newcommand{\AdvPRGv}[3]{\Adv^{\mathrm{prg}\textnormal{-}{#1}}_{#2}(#3)}
\newcommand{\ExpPRGv}[3]{\mathbf{Exp}^{\mathrm{prg}\textnormal{-}{#1}}_{#2}(#3)}
\newcommand{\AdvPRE}[2]{\Adv^{\mathrm{pre}}_{#1}(#2)}
\newcommand{\ExpPRE}[2]{\mathbf{Exp}^{\mathrm{pre}}_{#1}(#2)}
\newcommand{\prg}{\ensuremath{\mathcal{PRG}}}
\newcommand{\prgkeygen}{\ensuremath{\mathcal{GK}}}
\newcommand{\prgeval}{\ensuremath{\mathcal{G}}}
\newcommand{\vecu}{\ensuremath{\mathbf{u}}}
\newcommand{\vecs}{\ensuremath{\mathbf{s}}}
\newcommand{\vecr}{\ensuremath{\mathbf{r}}}
\newcommand{\hb}{\ensuremath{\mathsl{hb}}}
\newcommand{\Angle}[1]{\ensuremath{\left\langle #1\right\rangle}}
\newcommand{\Anglefix}[1]{\ensuremath{\langle #1\rangle}}
\newcommand{\game}[1]{\text{Game #1}}
\newcommand{\recover}{\ensuremath{\textsc{Recover}}}
\newcommand{\marc}[1]{\textbf{mf:} \textit{#1}}
\newcommand{\iter}{n}
%----
\newcommand{\rtab}{\mathrm{R}}
\newcommand{\dtab}{\mathrm{D}}
\newcommand{\khm}{[K,H,M]}
\newcommand{\khc}{[K,H,C]}
\newcommand{\kh}{[K,H]}
\newcommand{\rng}{\mathrm{Rng}}
\newcommand{\dom}{\mathrm{Dom}}

\newcommand{\Dplus}{D^+}
\newcommand{\fiter}{\textsf{Itr}}   %{f_+}
\newcommand{\Itr}{\textsf{Itr}}   %{f_+}
\newcommand{\Itrbf}{\textbf{Itr}}   %{f_+}

\newcommand{\kset}{\mathtt{K}}
\newcommand{\pkdet}{\pk_{\text{d}}}
\newcommand{\skdet}{\sk_{\text{d}}}

\newcommand{\stretchval}{1.2}

\newcommand{\mpage}[2]{\begin{minipage}{#1\textwidth} #2 \end{minipage}}
\newcommand{\fpage}[2]{\framebox{\begin{minipage}{#1\textwidth}\setstretch{\stretchval}\gamesfontsize #2 \end{minipage}}}

\newcommand{\hfpages}[3]{\hfpagess{#1}{#1}{#2}{#3}}
\newcommand{\hfpagess}[4]{
		\begin{tabular}{c@{\hspace*{.5em}}c}
		\framebox{\begin{minipage}[t]{#1\textwidth}\setstretch{\stretchval}\gamesfontsize #3 \end{minipage}}
		&
		\framebox{\begin{minipage}[t]{#2\textwidth}\setstretch{\stretchval}\gamesfontsize #4 \end{minipage}}
		\end{tabular}
	}
\newcommand{\hfpagesss}[6]{
		\begin{tabular}{c@{\hspace*{.5em}}c@{\hspace*{.5em}}c}
		\framebox{\begin{minipage}[t]{#1\textwidth}\setstretch{\stretchval}\gamesfontsize #4 \end{minipage}}
		&
		\framebox{\begin{minipage}[t]{#2\textwidth}\setstretch{\stretchval}\gamesfontsize #5 \end{minipage}}
		&
		\framebox{\begin{minipage}[t]{#3\textwidth}\setstretch{\stretchval}\gamesfontsize #6 \end{minipage}}
		\end{tabular}
	}
\newcommand{\hfpagessss}[8]{
		\begin{tabular}{c@{\hspace*{.5em}}c@{\hspace*{.5em}}c@{\hspace*{.5em}}c}
		\framebox{\begin{minipage}[t]{#1\textwidth}\setstretch{\stretchval}\gamesfontsize #5 \end{minipage}}
		&
		\framebox{\begin{minipage}[t]{#2\textwidth}\setstretch{\stretchval}\gamesfontsize #6 \end{minipage}}
		&
		\framebox{\begin{minipage}[t]{#3\textwidth}\setstretch{\stretchval}\gamesfontsize #7 \end{minipage}}
		&
		\framebox{\begin{minipage}[t]{#4\textwidth}\setstretch{\stretchval}\gamesfontsize #8 \end{minipage}}
		\end{tabular}
	}


\def\codestretch{\stretchval}

\newcommand{\hpagesl}[3]{
	\begin{tabular}{c|c}
	  \begin{minipage}{#1\textwidth}\setstretch{\codestretch} #2 \end{minipage}
	  &
	  \begin{minipage}{#1\textwidth} #3 \end{minipage}
	\end{tabular}
	}

\newcommand{\hpagessl}[4]{
	\begin{tabular}{c|@{\hspace*{.5em}}c}
	   \begin{minipage}[t]{#1\textwidth}\setstretch{\codestretch} #3 \end{minipage}
	   &
	   \begin{minipage}[t]{#2\textwidth}\setstretch{\codestretch} #4 \end{minipage}
	\end{tabular}
	}

\newcommand{\hpages}[3]{
	\begin{tabular}{cc}
	   \begin{minipage}[t]{#1\textwidth}\setstretch{\codestretch} #2 \end{minipage}
	   &
	   \begin{minipage}[t]{#1\textwidth}\setstretch{\codestretch} #3 \end{minipage}
	\end{tabular}
	}

\newcommand{\hpagess}[4]{
    \begin{tabular}{c@{\hspace*{1.5em}}c}
	   \begin{minipage}[t]{#1\textwidth}\setstretch{\codestretch} #3 \end{minipage}
	   &
	   \begin{minipage}[t]{#2\textwidth}\setstretch{\codestretch} #4 \end{minipage}
    \end{tabular}
	}

\newcommand{\hpagesss}[6]{
	\begin{tabular}{ccc}
	\begin{minipage}[t]{#1\textwidth}\setstretch{\codestretch}\gamesfontsize #4 \end{minipage} &
	\begin{minipage}[t]{#2\textwidth}\setstretch{\codestretch}\gamesfontsize #5 \end{minipage} &
	\begin{minipage}[t]{#3\textwidth}\setstretch{\codestretch}\gamesfontsize #6 \end{minipage}
	\end{tabular}}
\newcommand{\hpagesssl}[6]{
	\begin{tabular}{c|c|c}
	\begin{minipage}[t]{#1\textwidth}\setstretch{\codestretch} #4 \end{minipage} &
	\begin{minipage}[t]{#2\textwidth}\setstretch{\codestretch} #5 \end{minipage} &
	\begin{minipage}[t]{#3\textwidth}\setstretch{\codestretch} #6 \end{minipage}
	\end{tabular}}
\newcommand{\hpagessss}[8]{
	\begin{tabular}{cccc}
	\begin{minipage}[t]{#1\textwidth}\setstretch{\codestretch} #5 \end{minipage} &
	\begin{minipage}[t]{#2\textwidth}\setstretch{\codestretch} #6 \end{minipage} &
	\begin{minipage}[t]{#3\textwidth}\setstretch{\codestretch} #7 \end{minipage}
	\begin{minipage}[t]{#4\textwidth}\setstretch{\codestretch} #8 \end{minipage}
	\end{tabular}}



\newcommand{\authnote}[2]{\ifnum\authnotes=1\begin{quote}\textbf{#1 says:} #2\end{quote}\fi}
\newcommand{\authfnote}[2]{\ifnum\authnotes=1\footnote{\textbf{#1 says:} #2}\fi}
\newcommand{\myemph}[1]{\textsl{\textbf{#1}}}
\renewcommand{\paragraph}[1]{\vspace{.6em}\noindent\textbf{#1}\hspace*{.5em}}

\newcounter{mynote}[section]
\newcommand{\notecolor}{blue}
\newcommand{\scribenotecolor}{purple}
\newcommand{\proofreadernotecolor}{green}

\newcommand{\thenote}{\thesection.\arabic{mynote}}
\newcommand{\tnote}[1]{\ifnum\authnotes=1\refstepcounter{mynote}{\bf \textcolor{\notecolor}{$\ll$TCR~\thenote: {\sf #1}$\gg$}}\fi}
\newcommand{\scribenote}[1]{\ifnum\authnotes=1\refstepcounter{mynote}{\bf \textcolor{\scribenotecolor}{$\ll$Scribe~\thenote: {\sf #1}$\gg$}}\fi}
\newcommand{\pfreadernote}[1]{\ifnum\authnotes=1\refstepcounter{mynote}{\bf \textcolor{\proofreadernotecolor}{$\ll$Proofreader~\thenote: {\sf #1}$\gg$}}\fi}
\newcommand{\fixme}[1]{\ifnum\authnotes=1\textbf{\textcolor{red}{[FIXME: #1]}}\fi}



\newcommand{\semi}{\:;\:}
\newcommand{\bitset}{B}
\newcommand{\eq}{\mathrm{eq}}
\newcommand{\TrapPerm}{\mathcal{TP}}
\newcommand{\TPgen}{G}
\newcommand{\TPf}{F}
\newcommand{\TPinv}{\overline{F}}
%\newcommand{\Gen}{\mathcal{G}}

\newcommand{\Pibar}{{\overline{\Pi}}}
\newcommand{\kgbar}{{\overline{\kg}}}
\newcommand{\encbar}{{\overline{\enc\raisebox{2.5mm}{}}}}
\newcommand{\decbar}{{\overline{\dec}}}
%\newcommand{\kgbar}{{\overline{\calK}}}
%\newcommand{\encbar}{{\overline{\calE}}}
%\newcommand{\decbar}{{\overline{\calD}}}

\newcommand{\DEone}{\textsf{DE1}}
\newcommand{\DEtwo}{\textsf{DE2}}

\newcommand{\Znstar}{\Z_n^*}

\newcommand{\caseeqndef}[4]{
    \left\{
	\begin{array}{ll}
	    #1 & \textnormal{#2}	\\
        #3 & \textnormal{#4}
	\end{array}\right.}

% ========================================================================

\newcommand{\pubsim}{\calS}
\newcommand{\blenn}{\textsf{blen}}
\newcommand{\rhotable}{\mathtt{P}}
\newcommand{\TabIdx}{\mathtt{Idx}}
\newcommand{\TabB}{\mathtt{B}}
%\newcommand{\TabC}{\mathtt{C}}
\newcommand{\TabC}{\mathtt{ctr}}
\newcommand{\Tabc}{\mathtt{c}}
\newcommand{\TabD}{\mathtt{D}}
\newcommand{\TabE}{\mathtt{E}}
\newcommand{\TabH}{\mathtt{H}}
\newcommand{\TabYtoX}{\mathtt{YtoX}}
\newcommand{\TabX}{\mathtt{X}}
\newcommand{\TabY}{\mathtt{Y}}
\newcommand{\TabP}{\mathtt{P}}
\newcommand{\TabK}{\mathtt{K}}
\newcommand{\TabT}{\mathtt{T}}
\newcommand{\TabR}{\mathtt{R}}
\newcommand{\TabF}{\mathtt{F}}
\newcommand{\Tabf}{\mathtt{f}}
\newcommand{\TabFx}{\mathtt{F_x}}
\newcommand{\TabFy}{\mathtt{F_y}}
\newcommand{\TabV}{\mathtt{V}}
\newcommand{\TabZ}{\mathtt{T}}
\newcommand{\TabQ}{\mathtt{Q}}
\newcommand{\TabValue}{\mathtt{V}}
\newcommand{\TabIV}{\mathtt{IV}}

\newcommand{\xstar}{x^*}
\newcommand{\cstar}{c^*}
\newcommand{\mstar}{m^*}
\newcommand{\vstar}{v^*}

\newcommand{\satisfies}{\textsf{ satisfies}}

\newcommand{\advice}{\alpha}
\newcommand{\advicem}{\alpha_\textrm{m}}
\newcommand{\adviceg}{\alpha_\textrm{g}}
\newcommand{\wext}{\calE^+}
%\newcommand{\ext}{\calE}
\newcommand{\extenh}{\ext^+}
\newcommand{\extA}{\calE_A}
\newcommand{\extB}{\calE_B}
\newcommand{\extC}{\calE_C}
\newcommand{\Signoracle}{\mathsf{Sign}}
\newcommand{\environ}{{Env}}
\newcommand{\Poracle}{P}
\newcommand{\Pobject}{P}
\newcommand{\Horacle}{\mathsf{H}}
\newcommand{\Ooracle}{\mathsf{O}}
\newcommand{\pOracle}{\textsf{P}}
\newcommand{\chalOracle}{\mathbb{D}}
\newcommand{\encOracle}{\mathbb{E}}
\newcommand{\decOracle}{\mathbb{D}}
\newcommand{\extOracle}{\textsf{Ex}}
\newcommand{\extOracleA}{\textsf{SimEx}}
\newcommand{\extOracleB}{{\extOracle}}
\newcommand{\extkea}{\calE_{\textrm{kea}}}
\newcommand{\Akea}{A_{\textrm{kea}}}
\newcommand{\reconstruct}{\textnormal{Rec}}
\newcommand{\recons}{\textnormal{Rec}}
\newcommand{\Ext}{\textnormal{Ext}}
\newcommand{\pra}{\textnormal{pra}}
\newcommand{\wpra}{\textnormal{wpra}}
\newcommand{\extone}{\textnormal{pra}}
\newcommand{\exttwo}{\textnormal{ext}}
\newcommand{\extthree}{\textnormal{ext3}}
\newcommand{\ExtOne}[1]{\Exp^{\extone}_{#1}}
%\newcommand{\eExtOne}[1]{\Exp^{e\textnormal{-ext}}_{#1}}
%\newcommand{\ExtOneStar}[1]{\Exp^{\textnormal{ext*}}_{#1}}
\newcommand{\WPAExp}[1]{\Exp^{\textnormal{wpra}}_{#1}}
\newcommand{\WPAExpone}[1]{\Exp^{\textnormal{wpra}\dash1}_{#1}}
\newcommand{\InvExp}[1]{\Exp^{\textnormal{inv}}_{#1}}
\newcommand{\PAExp}[1]{\Exp^{\extone}_{#1}}
\newcommand{\PAExpone}[1]{\Exp^{\extone\dash1}_{#1}}
\newcommand{\CRExp}[1]{\Exp^{\mathrm{cr}}_{#1}}
\newcommand{\ExtExp}[2]{\Exp^{\exttwo\dash#1}_{#2}}
\newcommand{\ExtTwo}[2]{\Exp^{\exttwo\dash#1}_{#2}}
\newcommand{\ExtThree}[1]{\Exp^{\extthree}_{#1}}
%\newcommand{\ExtLR}[2]{\Exp^{\textnormal{ext-ind-}#2}_{#1}}
\newcommand{\orac}{{(\cdot)}}
\newcommand{\oracs}{{(\cdots)}}
\newcommand{\exorac}{\ext}
\newcommand{\Rorac}{\$}
\newcommand{\XSet}{\calX}
\newcommand{\flag}{\textsf{flag}}
\newcommand{\csROR}{\textrm{ROR}}

\newcommand{\ComO}{\textbf{Com}}



\newcommand{\AdvOr}{\calA_\textrm{OR}}
\newcommand{\AdvCoins}{\calA_\textrm{\$\$}}
\newcommand{\Zp}{\mathbb{Z}_p}
\newcommand{\bbG}{\mathbb{G}}
\newcommand{\group}{\bbG}

\newcommand{\rOWF}{\textnormal{rOWF}}
\newcommand{\ePre}{\textnormal{ePre}}
\newcommand{\CR}{\textnormal{CR}}
\newcommand{\CRbf}{\textbf{CR}}
%\newcommand{\EXT}{\textnormal{EXTow}}
\newcommand{\EXTbf}{\textbf{EXT}}
\newcommand{\EXT}{\textnormal{EXT}}
\newcommand{\PROp}{\textnormal{PRO-Pr}}

\newcommand{\WPA}{\textnormal{WPrA}}
\newcommand{\WPAbf}{\textbf{WPrA}}

\newcommand{\vPA}{v\textnormal{-PrA}}
\newcommand{\onePA}{\textnormal{1-PrA}}
\newcommand{\onedashpa}{\mathrm{1\dash pra}}
\newcommand{\vdashpa}{\mathrm{v\dash pra}}
\newcommand{\onePAExp}[1]{\Exp^{\onedashpa}_{#1}}
\newcommand{\vPAExp}[1]{\Exp^{\vdashpa}_{#1}}

\newcommand{\PA}{\textnormal{PrA}}
\newcommand{\PAbf}{\textbf{PrA}}
\newcommand{\PAp}{\textnormal{PrA-Pr}}
\newcommand{\PApbf}{\textbf{PrA-Pr}}

\newcommand{\PAgame}{\textnormal{PrA}}

\newcommand{\EXTone}{\textnormal{EXT1}}
\newcommand{\EXTonebf}{\textbf{EXT1}}
%\newcommand{\EXTcbf}{\textbf{EXT1}}
\newcommand{\EXTlr}{\textnormal{EXT-IND}}
%\newcommand{\EXTp}{\textnormal{EXT-$\$$}}
\newcommand{\EXTtwo}{\textnormal{EXT2}}
\newcommand{\EXTtwobf}{\textbf{EXT2}}
%\newcommand{\eEXT}{e\textnormal{-EXT}}

\newcommand{\pro}{\textnormal{pro}}
\newcommand{\PRO}{\textnormal{PRO}}
\newcommand{\PRObf}{\textbf{PRO}}
\newcommand{\PUBRO}{\textnormal{pub-RO}}
\newcommand{\PUBROsc}{\textsc{pub-RO}}
\newcommand{\pubpro}{\textnormal{pub-pro}}
\newcommand{\PUBPRO}{\textnormal{pub-PRO}}
\newcommand{\PUBPROsc}{\textsc{pub-PRO}}
\newcommand{\PUBPRObf}{\textbf{pub-PRO}}

\newcommand{\PUBICF}{\textnormal{pub-GRO}}
\newcommand{\PUBPICF}{\textnormal{pub-GPRO}}
\newcommand{\PUBPICFbf}{\textbf{pub-GPRO}}
\newcommand{\pubgpro}{\textnormal{pub-gpro}}
\newcommand{\icf}{f}

\newcommand{\PROG}{\textnormal{PROG}}
\newcommand{\PROGbf}{\textbf{PROG}}


\newcommand{\SimF}{\textnormal{Sim-$F$}}
\newcommand{\Choosef}{\textnormal{Choose-$f$}}
\newcommand{\ChooseE}{\textnormal{Choose-$E$}}
\newcommand{\ChooseD}{\textnormal{Choose-$D$}}
\newcommand{\RngSet}{\mathcal{R}}
\newcommand{\DomSet}{\mathcal{D}}

\newcommand{\eval}{\mathit{eval}}
\newcommand{\geval}{\mathit{geval}}
\newcommand{\reveal}{\mathit{reveal}}

\newcommand{\cheatflag}{\mathsf{flag}}
\newcommand{\notcheatflag}{\overline{\mathsf{flag}}}

\newcommand{\IV}{IV}
\newcommand{\sfpad}{\textsf{sfpad}}
\newcommand{\pfpad}{\textsf{pfpad}}
\newcommand{\strippad}{\textsf{unpad}}
\newcommand{\SMD}{\textnormal{SMD}}
\newcommand{\SMDb}{\textbf{SMD}}
\newcommand{\EXTP}{\textnormal{EXT-Pr}}
\newcommand{\EXTPbf}{\textbf{EXT-Pr}}
\newcommand{\EXTPone}{\textnormal{EXTow-Pr}}
\newcommand{\EXTPonebf}{\textbf{EXTow-Pr}}
\newcommand{\EXTPtwo}{\textnormal{EXT1-Pr}}
\newcommand{\EXTPtwobf}{\textbf{EXT1-Pr}}



\newcommand{\AdvLeg}{\calA}
\newcommand{\AdvSingle}{\calA_1}
\newcommand{\ADmodel}{\textrm{AD}}
\newcommand{\ORmodel}{\textrm{OR}}
\newcommand{\Cmodel}{\textrm{\$\$}}
\newcommand{\AdvAdvice}{\calA_{\ADmodel}}
\newcommand{\ExtSingle}{\calE_{\textrm{SI}}}

\newcommand{\simu}{\calS}

\newcommand{\PreIm}{\textsf{PreIm}}

\newcommand{\IC}[1]{\mathsf{IC}_{#1}}

\newcommand{\RF}[1]{\mathsf{RF}_{#1}}
\newcommand{\finv}{f^{-1}}
\newcommand{\TDPgen}{\calF}

\newcommand{\RSAkem}{\mathsf{KEM}}

\newcommand{\intspy}{\mathcal{I}}

\newcommand{\RO}{\textnormal{RO}}

\newcommand{\FDH}{\textsf{FDH}}
\newcommand{\Kg}{\textsf{Kg}}
\newcommand{\Sign}{\textsf{Sign}}
\newcommand{\Ver}{\textsf{Ver}}

\newcommand{\dropin}[1]{\mathsf{#1}}
\newcommand{\Func}{\mathrm{Func}}
\newcommand{\Perm}{\mathrm{Perm}}

\newcommand{\Ddom}{D^+}
\newcommand{\state}{st}

\newcommand{\extfails}{\mathsf{\ext\ fails}}
\newcommand{\oracleP}{P}

\newcommand{\Coll}{\mathsf{Coll}}
\newcommand{\Einv}{D}
\newcommand{\cbar}{\overline{c}}
\newcommand{\mbar}{\overline{m}}
\newcommand{\ybar}{\overline{y}}

\newcommand{\cnext}{w}

\newcommand{\calSA}{{\cal S}_A}
\newcommand{\calSB}{{\cal S}_B}

\newcommand{\calOA}{\calO'}
\newcommand{\MMO}{\textsf{MMO}}
\newcommand{\MMOguarded}{\overline{\textsf{MMO}}}
\newcommand{\DM}{\textsf{DM}}
\newcommand{\DMrest}{\overline{\textsf{DM}}}
\newcommand{\DMguarded}{\overline{\textsf{DM}}}
\newcommand{\ellmax}{\ell_{max}}

\newcommand{\adv}{\textnormal{adv}}
\newcommand{\priv}{\textnormal{priv}}
\newcommand{\pub}{\textnormal{pub}}
\newcommand{\hguarded}{\bar{h}}

\newcommand{\pubro}{\calF}
\newcommand{\ro}{\calF}
\newcommand{\rostr}{\calF}
\newcommand{\idcipher}{\calC}

\newcommand{\eqnand}{\hspace{1em}\textnormal{ and }\hspace{1em}}

\newcommand{\buf}{\mathrm{Buf}}
\newcommand{\bufpub}{\mathrm{PubBuf}}
\newcommand{\bufpriv}{\mathrm{PrivBuf}}

\newcommand{\numfont}[1]{{\footnotesize \texttt{#1}}\,}
\newcommand{\numfontbf}[1]{{\footnotesize \textbf{\texttt{#1}}}\,}
\newcommand{\Finish}{\textnormal{Finish()}}

\newcommand{\cpre}{C^{\mathsc{pre}}}
\newcommand{\cpreinv}{C^{\mathsc{-pre}}}
\newcommand{\cpost}{C^{\mathsc{post}}}
\newcommand{\caux}{C^{\mathsc{aux}}}
\newcommand{\cpostinv}{C^{\mathsc{-post}}}
\newcommand{\cauxinv}{C^{\mathsc{-aux}}}

\newcommand{\given}{\; | \;}
\newcommand{\indiff}{\textrm{indiff}}
\newcommand{\ExpIndiffCons}[1]{\Exp^{\indiff\dash1}_{#1}}
\newcommand{\ExpIndiffSim}[1]{\Exp^{\indiff\dash0}_{#1}}

% ========================================================================


\newcommand{\eve}{\advA}
\newcommand{\keyse}{\mathcal{K_{\mathrm{se}}}}
\newcommand{\keyma}{\mathcal{K_{\mathrm{ma}}}}

\newcommand{\then}{\,;\,}
\newcommand{\andthen}{\thinspace : \thinspace}
\newcommand{\bitsorc}{\$}
\newcommand{\newmsg}{\mathrm{new\mbox{-}msg}}

%\newcommand{\botse}{\bot_e}
%\newcommand{\botma}{\bot_m}
%\newcommand{\botcode}{\bot_c}
\newcommand{\botse}{\bot}
\newcommand{\botma}{\bot}
\newcommand{\botcode}{\bot}

\newcommand{\pforge}{\mathrm{pforge}}
\newcommand{\cforge}{\mathrm{cforge}}
\newcommand{\invalid}{\mathrm{invalid}}
\newcommand{\valid}{f}
\newcommand{\dectest}{\mathsf{Test}}
\newcommand{\testproc}{\procfont{Test}}
\newcommand{\lrorc}{\mathsf{LR}}

\newcommand{\MEE}{\mathsf{MEE}}
\newcommand{\MEEK}{{K}}
\newcommand{\ma}{\mathrm{ma}}
\newcommand{\se}{\mathrm{se}}
\newcommand{\MEEKma}{{K_{\ma}}}
\newcommand{\MEEKse}{{K_{\se}}}
\newcommand{\TLSEN}{\textsf{TLScode}}
\newcommand{\PadTLS}{\textsf{PadTLS}}
\newcommand{\UnpadTLS}{\textsf{UnpadTLS}}
\newcommand{\TLSencode}{\textsf{TLSencode}}
\newcommand{\TLSdecode}{\textsf{TLSdecode}}


\newcommand{\meetlscbc}{\mathsf{MEE\dash TLS\dash CBC}}
\newcommand{\meetlsctr}{\mathsf{MEE\dash TLS\dash CTR}}
\newcommand{\meegencbc}{\mathsf{MEE\dash GEN\dash CBC}}
\newcommand{\meegenctr}{\mathsf{MEE\dash GEN\dash CTR}}

\newcommand{\prfkeyspace}{\calK_{\ma}}
\newcommand{\prfkey}{K}
\newcommand{\PaddingScheme}{\textsf{PadS}}

\newcommand{\collevent}{\mathsc{DColl}}

\newcommand{\ith}{{i^{\mathrm{th}}}}
\newcommand{\jth}{{j^{\mathrm{th}}}}
\newcommand{\newXflag}{{\mathsf{NewX}}}
\newcommand{\repeatXflag}{{\mathsf{RepeatX}}}
\newcommand{\notrepeatXflag}{{\mathsf{NoRepeatX}}}

\newcommand{\bnm}{\begin{newmath}}
\newcommand{\enm}{\end{newmath}}
\newcommand{\bne}{\begin{newequation}}
\newcommand{\ene}{\end{newequation}}
\newcommand{\sets}{\textnormal{ sets }}
\newcommand{\CTR}{\textnormal{CTR}}
\newcommand{\CBC}{\textnormal{CBC}}
\newcommand{\PCBC}{\textnormal{P-CBC}}
\newcommand{\tilC}{\tilde{C}}
\newcommand{\gamesfontsize}{\scriptsize}

\newcommand{\mgood}{\mathsf{Mgood}}
\newcommand{\tagused}{\mathsf{TagUsed}}
\newcommand{\mblock}{\mathsf{Mblock}}
\newcommand{\newcblock}{\mathsf{Cnew}}

\newcommand{\reqlen}{\ell}
\newcommand{\reqlenmax}{\reqlen_{\mathrm{max}}}
\newcommand{\padlen}{\psi}
\newcommand{\padlenmax}{\psi}
\newcommand{\padblen}{p}
\newcommand{\tagidx}{t}
\newcommand{\numpadblocks}{\gamma}
\newcommand{\header}{H}

\newcommand{\taglen}{\tau}
\newcommand{\lastbyte}{\textnormal{lastbyte}}
\newcommand{\inttobyte}{\textnormal{int2byte}}
\newcommand{\bytetoint}{\textnormal{byte2int}}

\newcommand{\bitsplit}{\textnormal{split}}
\newcommand{\last}{\textnormal{last}}
\newcommand{\first}{\textnormal{first}}
\newcommand{\chop}{\textnormal{chop}}

\newcommand{\modF}{\tilde{F}}
\newcommand{\phase}{\textsf{phase}}


\newcommand{\main}{{\procfont{main}}}
\newcommand{\mainn}[1]{{\underline{\main\ {#1}}:}}
\newcommand{\mainnv}[1]{\setstretch{1.1}{\underline{\main\ {#1}}:}\smallskip}
\newcommand{\bzo}{\bracketize{1}}
\newcommand{\bzt}{\bracketize{2}}
\newcommand{\koe}{k_1^e}
\newcommand{\kie}[1]{k_{#1}^e}
\newcommand{\kim}[1]{k_{#1}^m}
\newcommand{\tbc}{\mathsf{TBC}}
\newcommand{\tbce}{\widetilde{E}}
\newcommand{\tbcd}{\widetilde{D}}
\newcommand{\kte}{k_2^e}
\newcommand{\kom}{k_1^m}
\newcommand{\ktm}{k_2^m}
\newcommand{\eecl}{t}
\newcommand{\headerspace}{\calH}
\newcommand{\reqlenspace}{\calL}

\newcommand{\spacecomma}{\;,\;}
\newcommand{\calZ}{\mathcal{Z}}
\newcommand{\trunc}{\mathrm{Trunc}}
\newcommand{\truncleft}{\overline{\mathrm{Trunc}}}

\newcommand{\Pad}{\mathrm{Pad}}
\newcommand{\PadH}{\mathrm{PadH}}
\newcommand{\PadM}{\mathrm{PadM}}
%%Alternate padding

\newcommand{\AltPaddingScheme}{\textsf{AltPadD}}

\newcommand{\AltPad}{\mathrm{AltPad}}
\newcommand{\AltPadH}{\mathrm{AltPadH}}
\newcommand{\AltPadM}{\mathrm{AltPadM}}
\newcommand{\AltPadSuffix}{\mathrm{AltPadSuf}}
%%

\newcommand{\Unpad}{\mathrm{Unpad}}
\newcommand{\UnpadH}{\mathrm{UnpadH}}
\newcommand{\UnpadM}{\mathrm{UnpadM}}
\newcommand{\Parse}{\mathrm{Parse}}
\newcommand{\PadSuffix}{\mathrm{PadSuf}}
\newcommand{\Padcheck}{\mathrm{PadCheck}}
\newcommand{\tagbit}{\mathtt{t}}
\newcommand{\TabS}{\mathtt{S}}
\newcommand{\ellc}{{\ell_c}}
\newcommand{\ellj}{{\ell_j}}
\newcommand{\teec}{{t_c}}
\newcommand{\teej}{{t_j}}
\newcommand{\essj}{{s_j}}
\newcommand{\essc}{{s_c}}
\newcommand{\tpad}{\tilde{P}}
\newcommand{\taglenn}{{n}}
\newcommand{\bmin}{{b_{\mathrm{min}}}}
\newcommand{\headerstar}{\header^*}
\newcommand{\gamenote}[1]{{\tiny #1}}
%\newcommand{\ellmax}{{\ell_{\mathrm{max}}}}

\newcommand{\highlight}[1]{\mbox{\colorbox{mygrey}{$#1$}}}
\newcommand{\calYbad}{\calB}%{\calY_{bad}}
\newcommand{\PadPrefix}{\textnormal{PadPrefix}}
\newcommand{\testcount}{\mathrm{cnt}}


\newcommand{\CN}{{CN}}
\newcommand{\CRan}{{R}}
\newcommand{\SN}{{SN}}
\newcommand{\orbit}{{orb}}


\newcommand{\KDF}{\mathsf{KDF}}

\newcommand{\skserver}{\sk_s}
\newcommand{\skclient}{\sk_c}
\newcommand{\Nonce}{N}
\newcommand{\Nclient}{N_c}
\newcommand{\Nserver}{N_s}
\newcommand{\csuites}{\mathcal{CS}}
\newcommand{\certclient}{\cert_c}
\newcommand{\certserver}{\cert_s}


\newcommand{\servername}{\mathrm{SN}}

\newcommand{\confserver}{\mathrm{conf}_s}
\newcommand{\confclient}{\mathrm{conf}_c}
\newcommand{\confsn}{\mathrm{conf}_{\servername}}
\newcommand{\bracketize}[1]{\lbrack #1 \rbrack}
\newcommand{\bktz}[1]{\bracketize{#1}}
\newcommand{\dbktz}[2]{\bracketize{#1}\bracketize{#2}}
\newcommand{\sid}{\mathrm{sid}}
\newcommand{\kex}{\mathrm{kex}}
\newcommand{\kexguess}{\mathrm{kex}^*}
\newcommand{\hello}{\mathrm{hel}}
\newcommand{\fin}{\mathrm{fin}}
\newcommand{\chc}{\mathrm{chc}}
\newcommand{\ext}{\mathrm{ext}}
\newcommand{\appdata}{\mathrm{app data}}
\newcommand{\encrypted}[1]{\{#1\}}
\newcommand{\initencrypted}[1]{\{#1\}^{\mathrm{init}}}
\newcommand{\signed}[1]{\llbracket#1\rrbracket}
%\newcommand{\optional}[1]{\textcolor{blue}{[} #1 \textcolor{blue}{]}}
\newcommand{\optional}[1]{\textcolor{Gray}{ #1 }}
\newcommand{\sidclient}{\sid_c}
\newcommand{\sidserver}{\sid_s}
\newcommand{\verclient}{\ver_c}
\newcommand{\verserver}{\ver_s}
\newcommand{\kexclient}{\kex_c}
\newcommand{\kexclientguess}{\kexguess_c}
\newcommand{\kexserver}{\kex_s}
\newcommand{\helloclient}{\hello_c}
\newcommand{\helloserver}{\hello_s}
\newcommand{\finclient}{\fin_c}
\newcommand{\finserver}{\fin_s}
\newcommand{\chcclient}{\chc_c}
\newcommand{\chcserver}{\chc_s}
\newcommand{\extclient}{\ext_c}
\newcommand{\extserver}{\ext_s}




\newcommand{\HIDE}{\textnormal{HIDE}}
\newcommand{\FROB}{\textnormal{FROB}}
\newcommand{\CROB}{\textnormal{CROB}}
\newcommand{\BIND}{\textnormal{r-BIND}}
\newcommand{\SRBIND}{\textnormal{sr-BIND}} %%% strong binding
\newcommand{\TRBIND}{\textnormal{tr-BIND}} %%% targeted receiver binding
\newcommand{\RBIND}{\BIND}
\newcommand{\VFROB}{\textnormal{eFROB}}
\newcommand{\SBIND}{\textnormal{s-BIND}}
\newcommand{\MRBIND}{\textnormal{mrBIND}}
\newcommand{\VBIND}{\textnormal{vBIND}}
\newcommand{\DBIND}{\textnormal{t-BIND}}
\newcommand{\EBIND}{\textnormal{EBIND}}
\newcommand{\MBIND}{\textnormal{mBIND}}
\newcommand{\TBIND}{\textnormal{TBIND}}
\newcommand{\CS}{{\textnormal{\textsf{CS}}}}
\newcommand{\Commit}{\textsf{Com}}
\newcommand{\Open}{\textsf{Open}}
\newcommand{\Verify}{\textsf{VerC}}
\newcommand{\openspace}{\keyspace_f}
\newcommand{\openingspace}{\keyspace_{\EC}}
\newcommand{\commitspace}{\calC}
\newcommand{\frankspace}{\calT}
\newcommand{\Efrankspace}{\calT}
\newcommand{\HMAC}{\textnormal{\textsf{HMAC}}}
\newcommand{\FSE}{\textsf{FSE}}
\newcommand{\ftag}{T}
\newcommand{\fopen}{K_{f}}
\newcommand{\open}{K_{c}}
\newcommand{\kimg}{K_{\text{a}}}
\newcommand{\nimg}{N_{\text{a}}}
\newcommand{\gcmenc}{\mathsf{GCM\dash Enc}}
\newcommand{\gcmdec}{\mathsf{GCM\dash Dec}}
\newcommand{\fbid}{\textsf{id}}
\newcommand{\fbtag}{\mathsf{FBTag}}
\newcommand{\fbauthtag}{\textsf{t}_{\text{FB}}}
\newcommand{\imct}{C_{\text{a}}}
\newcommand{\imgd}{D_{\text{a}}}
\newcommand{\shatfs}{\mathsf{SHA\dash 256}}
\newcommand{\collidegcm}{\mathsf{Collide\dash GCM}}
\newcommand{\AD}{\text{AD}}
\newcommand{\mlen}{\text{mlen}}
\newcommand{\adlen}{\text{adlen}}
\newcommand{\junk}{\mathsf{M}_{\text{j}}}
\newcommand{\CE}{\textnormal{{\textsf{CE}}}}
\newcommand{\CEH}{\textsf{CEH}}
\newcommand{\CKg}{{\textsf{Kg}}}
\newcommand{\CEnc}{{\textsf{Enc}}}
\newcommand{\CDec}{{\textsf{Dec}}}
\newcommand{\CVer}{{\textsf{Ver}}}
\newcommand{\cmpr}{\mathsf{f}}
\newcommand{\modu}{\,\mathrm{mod}\,}
\newcommand{\PadScheme}{\mathbb{P}}
\newcommand{\UpdatePad}{\mathrm{UpdatePad}}
\newcommand{\FinalizePad}{\mathrm{FinalizePad}}
\newcommand{\kprf}{K_{\text{prf}}}
\newcommand{\attch}{M_{\text{a}}}
\newcommand{\attchct}{C_{\text{a}}}
\newcommand{\attchab}{\attch^{\text{ab}}}
\newcommand{\AESCTREnc}{\text{CTR-Enc}}
\newcommand{\AESCTRDec}{\text{CTR-Dec}}
\newcommand{\AMerge}{\text{Att-Merge}}
\newcommand{\gpad}{P}
\newcommand{\kimgab}{K_{1}}
\newcommand{\kjunk}{K_{2}}
\newcommand{\abus}{_{1}}
\newcommand{\juhk}{_{2}}
\newcommand{\accum}{\mathsf{acc}}
\newcommand{\blen}{\mathsf{blen}}
\newcommand{\ghash}{\mathsf{GHASH}}
\newcommand{\sshare}{\mathsf{SShare}}

% randomized symmetric encryption defs
\newcommand{\SEenc}{\textsf{Enc}}
\newcommand{\SEencsim}{\textsf{EncSim}}
\newcommand{\RORreal}{\textnormal{REAL}}
\newcommand{\RORrand}{\textnormal{RAND}}


%newcommand{\ctxtlen}{\ell}
\newcommand{\EC}{\textnormal{{\textsf{EC}}}}
\newcommand{\EKg}{{\textsf{EKg}}}
\newcommand{\EEnc}{{\textsf{EC}}}
\newcommand{\EDec}{{\textsf{DO}}}
\newcommand{\EVer}{{\textsf{EVer}}}


\newcommand{\MDE}{\HFC}
\newcommand{\MDKg}{{\textsf{\HFC Kg}}}
\newcommand{\MDEnc}{{\textsf{\HFC Enc}}}
\newcommand{\MDDec}{{\textsf{\HFC Dec}}}
\newcommand{\MDVer}{{\textsf{\HFC Ver}}}

\newcommand{\SPE}{\textnormal{{\textsf{SPE}}}}
\newcommand{\SPKg}{{\textsf{SPKg}}}
\newcommand{\SPEnc}{{\textsf{SPEnc}}}
\newcommand{\SPDec}{{\textsf{SPDec}}}
\newcommand{\SPVer}{{\textsf{SPVer}}}

\newcommand{\HFC}{\textsf{HFC}}
\newcommand{\EtM}{\mathsf{EtM}}
\newcommand{\MtE}{\mathsf{MtE}}
\newcommand{\GCM}{\mathsf{GCM}}

\newcommand{\ctxt}{C}
\newcommand{\ctxtA}{C}
\newcommand{\ctxtB}{C_B}
\newcommand{\ctxtAE}{C_{\textnormal{\textsf{AE}}}}
\newcommand{\EctxtA}{\ctxtAEC}
\newcommand{\EctxtB}{\ctxtBEC}

\newcommand{\ctxtlen}{\textsf{clen}}

\newcommand{\CEbad}{\textnormal{\textsf{CEBad}}}
\newcommand{\CEncBad}{\textsf{EncBad}}
\newcommand{\CDecBad}{\textsf{DecBad}}
\newcommand{\CVerBad}{\textsf{VerBad}}

\newcommand{\CSbad}{\textnormal{\textsf{CSBad}}}
\newcommand{\CommitBad}{\textsf{ComBad}}
\newcommand{\VerifyBad}{\textsf{VerBad}}



\newcommand{\FBfrank}{\textsf{FB}}
\newcommand{\FBenc}{\textsf{FBEnc}}
\newcommand{\FBdec}{\textsf{FBDec}}
\newcommand{\FBver}{\textsf{FBVer}}

\newcommand{\AEADror}{\textnormal{ROR}} % nae = nonce-based auth enc
\newcommand{\AEADreal}{\textnormal{REAL}}
\newcommand{\AEADrand}{\textnormal{RAND}}
\newcommand{\AEADctxt}{\textnormal{\CTXT}}
\newcommand{\COREC}{\textnormal{COR}} % nae = nonce-based auth enc
\newcommand{\SCOREC}{\textnormal{S-COR}} % nae = nonce-based auth enc

\newcommand{\MORORctxt}{\textnormal{MO-ROR-C}} % nae = nonce-based auth enc
\newcommand{\MOREALctxt}{\textnormal{MO-REAL-C}} % nae = nonce-based auth enc
\newcommand{\MORANDctxt}{\textnormal{MO-RAND-C}} % nae = nonce-based auth enc
\newcommand{\SORORctxt}{\textnormal{ROR-C}} % nae = nonce-based auth enc
\newcommand{\SOREALctxt}{\textnormal{SO-REAL-C}} % nae = nonce-based auth enc
\newcommand{\SORANDctxt}{\textnormal{SO-RAND-C}} % nae = nonce-based auth enc
\newcommand{\MOROR}{\textnormal{MO-ROR}} % nae = nonce-based auth enc
\newcommand{\MOREAL}{\textnormal{MO-REAL}} % nae = nonce-based auth enc
\newcommand{\MORAND}{\textnormal{MO-RAND}} % nae = nonce-based auth enc
\newcommand{\SOROR}{\textnormal{ROR}} % nae = nonce-based auth enc
\newcommand{\SOREAL}{\textnormal{REAL}} % nae = nonce-based auth enc
\newcommand{\SORAND}{\textnormal{RAND}} % nae = nonce-based auth enc
%\newcommand{\REALce}{\textnormal{REAL}} % nae = nonce-based auth enc
%\newcommand{\IDEALce}{\textnormal{RAND}} % nae = nonce-based auth enc

\newcommand{\NMOROR}{\textnormal{MO-nROR}} % nae = nonce-based auth enc
\newcommand{\NMOREAL}{\textnormal{MO-nREAL}} % nae = nonce-based auth enc
\newcommand{\NMORAND}{\textnormal{MO-nRAND}} % nae = nonce-based auth enc

\newcommand{\NMORORctxt}{\textnormal{MO-nROR-C}} % nae = nonce-based auth enc
\newcommand{\NMOREALctxt}{\textnormal{MO-nREAL-C}} % nae = nonce-based auth enc
\newcommand{\NMORANDctxt}{\textnormal{MO-nRAND-C}} % nae = nonce-based auth enc

\newcommand{\NSOREAL}{\textnormal{nREAL}} % nae = nonce-based auth enc
\newcommand{\NSORAND}{\textnormal{nRAND}} % nae = nonce-based auth enc
\newcommand{\NSOCTXT}{\textnormal{nCTXT}} % nae = nonce-based auth enc

\newcommand{\EtE}{\textnormal{EtE}} % nae = nonce-based auth enc

\newcommand{\CtE}{\textnormal{CtE1}}
\newcommand{\CtEEnc}{\textnormal{CtE1-Enc}}
\newcommand{\CtEDec}{\textnormal{CtE1-Dec}}
\newcommand{\CtEVer}{\textnormal{CtE1-Ver}}

\newcommand{\FCtE}{\textnormal{CtE2}}
\newcommand{\FCtEEnc}{\textnormal{CtE2-Enc}}
\newcommand{\FCtEDec}{\textnormal{CtE2-Dec}}
\newcommand{\FCtEVer}{\textnormal{CtE2-Ver}}

\newcommand{\NCE}{\textnormal{nCE}}
\newcommand{\NCEKg}{\textnormal{Kg}}
\newcommand{\NCEEnc}{\textnormal{Enc}}
\newcommand{\NCEDec}{\textnormal{Dec}}
\newcommand{\NCEVer}{\textnormal{Ver}}

\newcommand{\FastNCE}{\textnormal{CEP}}
\newcommand{\FastKg}{\textnormal{CEP-Kg}}
\newcommand{\FastEnc}{\textnormal{CEP-Enc}}
\newcommand{\FastDec}{\textnormal{CEP-Dec}}
\newcommand{\FastVer}{\textnormal{CEP-Ver}}
\newcommand{\LCP}{\text{LCP}}
\newcommand{\OCB}{\textnormal{OCB}}
\newcommand{\OCBKg}{\textnormal{OCB-Kg}}
\newcommand{\OCBEnc}{\textnormal{OCB-Enc}}
\newcommand{\OCBDec}{\textnormal{OCB-Dec}}
\newcommand{\OCBVer}{\textnormal{OCB-Ver}}
\newcommand{\querya}{a}
\newcommand{\BC}{\textnormal{BC}}
\newcommand{\BCKg}{\textnormal{BC-Kg}}
\newcommand{\BCEnc}{\textnormal{BC-Enc}}
\newcommand{\BCDec}{\textnormal{BC-Dec}}
\newcommand{\BCVer}{\textnormal{BC-Ver}}

\newcommand{\tweakCipher}{\tilde{\calE}}
\newcommand{\tweakE}{\tilde{E}}
\newcommand{\tweakD}{\tilde{D}}

\newcommand{\tlk}{\tilde{K}}
\newcommand{\GSub}{\textnormal{G}}
\newcommand{\twc}{\textnormal{TC}}
\newcommand{\tweakspace}{\mathcal{T}}
\newcommand{\CSCTXT}{\textnormal{NM}} % nae = nonce-based auth enc
\newcommand{\OCommit}{\textbf{Com}} % nae = nonce-based auth enc
\newcommand{\OChalVerify}{\textbf{ChalVerC}} % nae = nonce-based auth enc

\newcommand{\MOCTXT}{\textnormal{MO-CTXT}} % nae = nonce-based auth enc
\newcommand{\SOCTXT}{\textnormal{CTXT}} % nae = nonce-based auth enc
\newcommand{\NMOCTXT}{\textnormal{MO-nCTXT}} % nae = nonce-based auth enc
\newcommand{\REALnae}{\textnormal{REAL}} % nae = nonce-based auth enc
\newcommand{\IDEALnae}{\textnormal{RAND}} % nae = nonce-based auth enc

\newcommand{\OEnc}{\textbf{Enc}} % nae = nonce-based auth enc
\newcommand{\ODec}{\textbf{Dec}} % nae = nonce-based auth enc
\newcommand{\OChalEnc}{\textbf{ChalEnc}} % nae = nonce-based auth enc
\newcommand{\OChalDec}{\textbf{ChalDec}} % nae = nonce-based auth enc

\newcommand{\TagAlg}{\textsf{tag}} % nae = nonce-based auth enc
\newcommand{\VerAlg}{\textsf{ver}} % nae = nonce-based auth enc

\newcommand{\OTag}{\textbf{Tag}} % nae = nonce-based auth enc
\newcommand{\OVer}{\textbf{Ver}} % nae = nonce-based auth enc
\newcommand{\OKey}{\textbf{KeyCom}} % nae = nonce-based auth enc

\newcommand{\regF}{F}
\newcommand{\crF}{F^{\mathrm{cr}}}

\newcommand{\Valid}{\mathbb{V}}
\newcommand{\Zos}{B}
\newcommand{\badtruezo}{\bad_{01}\gets\true}
\newcommand{\badtrueot}{\bad_{12}\gets\true}
\newcommand{\badzo}{\bad_{01}}
\newcommand{\badot}{\bad_{12}}
\newcommand{\prim}{f}
\newcommand{\func}{f}
\newcommand{\triphash}{\mathcal{H}}
\newcommand{\LP}{LP_{231}}
\newcommand{\LPPRF}{LP^{\text{prf}}_{231}}
\newcommand{\chain}{V}
\newcommand{\two}{\textbf{2}}
\newcommand{\perm}{\pi}
\newcommand{\TC}{TC}
\newcommand{\moENC}{\textnormal{\textsf{mo\textnormal{-}\CEnc}}}
\newcommand{\moDEC}{\textnormal{\textsf{mo\textnormal{-}\CDec}}}
\newcommand{\moVER}{\textnormal{\textsf{mo\textnormal{-}\CVer}}}
\newcommand{\moCE}{\textnormal{\textsf{mo\textnormal{-}\CE}}}
\newcommand{\moKG}{\textnormal{\textsf{mo\textnormal{-}\CKg}}}
\newcommand{\aeadlen}{\textsf{aead.len}}
\newcommand{\tagT}{T}
\newcommand{\encrypt}{encryptment }%%%Macros for names of encryptment algorithsm
\newcommand{\opening}{opening }
\newcommand{\ctxtAEC}{C_{\EC}}
\newcommand{\ctxtBEC}{B_{\EC}}
\newcommand{\KEC}{K_{\EC}}
\newcommand{\queryflag}{\textsf{query\textnormal{-}made}}
\newcommand{\kblen}{\kappa}
\newcommand{\gstate}{S}
\newcommand{\padded}{P}
\newcommand{\squeeze}{X}
\newcommand{\taglength}{t}

\newcommand{\otCTXT}{\text{otCTXT}}
\newcommand{\headlen}{\ell_H}
\newcommand{\msglen}{\ell_M}
\newcommand{\padindex}{n}
\newcommand{\supp}{\textnormal{Supp}}

\newcommand{\runtime}{t}
\newcommand{\numqueries}{q}


%%% Ciphers
\newcommand{\cipher}{\calE}
\newcommand{\cipherK}{\calK}
\newcommand{\cipherE}{E}
\newcommand{\cipherD}{D}
\newcommand{\keyspace}{\calK}
\newcommand{\msgspace}{\calM}
\newcommand{\ctxtspace}{\calC}
\newcommand{\msg}{M}
\newcommand{\ct}{C}
\newcommand{\randfn}{\rho}

\newcommand{\blockciphers}{\textnormal{BC}}
\newcommand{\ic}{E}
\newcommand{\icInv}{D}
\newcommand{\Domain}{\texttt{Dom}}
\newcommand{\Range}{\texttt{Ran}}

\newcommand{\TKR}{\textnormal{TKR}} % security game for key recovery
\newcommand{\Fn}{\textnormal{Fn}}
\newcommand{\FnInv}{\textnormal{FnInv}}
\newcommand{\FnSim}{\textnormal{FnSim}}
\newcommand{\AdvTKR}[2]{\Adv^{\mathrm{tkr}}_{#1}(#2)}

\newcommand{\eks}{\textnormal{eks}}

\newcommand{\KR}{\textnormal{KR}} % security game for key recovery
\newcommand{\AdvKR}[2]{\Adv^{\mathrm{kr}}_{#1}(#2)}

\newcommand{\OTIND}{\textnormal{otIND}} % security game for key recovery
\newcommand{\AdvOTIND}[2]{\Adv^{\mathrm{ot\dash ind}}_{#1}(#2)}


\newcommand{\PRP}{\textnormal{PRP}}
\newcommand{\AdvPRP}[2]{\Adv^{\mathrm{prp}}_{#1}(#2)}

\newcommand{\TPRP}{\textnormal{TPRP}}
\newcommand{\AdvTPRP}[2]{\Adv^{\mathrm{tprp}}_{#1}(#2)}
\newcommand{\tweakpi}{\tilde{\pi}}

\newcommand{\STPRP}{\textnormal{STPRP}}
\newcommand{\AdvSTPRP}[2]{\Adv^{\mathrm{\pm tprp}}_{#1}(#2)}



\newcommand{\SPRP}{\textnormal{SPRP}}
\newcommand{\AdvSPRP}[2]{\Adv^{\mathrm{\pm prp}}_{#1}(#2)}



% \newcommand{\inv}{^{-1}}
% \newcommand{\Adv}{\mathbf{Adv}}
% \newcommand{\keyspace}{\calK}
\newcommand{\PRFReal}{\text{PRF-Real}}
\newcommand{\PRFIdeal}{\text{PRF-Ideal}}
\newcommand{\RFRange}{\calR}
\newcommand{\RFun}{\mathbf{R}}
\newcommand{\Fun}{\mathsf{Fn}}
\newcommand{\SPRPIdeal}{\text{SPRP-Ideal}}
\newcommand{\SPRPReal}{\text{SPRP-Real}}
\newcommand{\PRPIdeal}{\text{PRP-Ideal}}
\newcommand{\PRPReal}{\text{PRP-Real}}
\newcommand{\TSPRPIdeal}{\text{TSPRP-Ideal}}
\newcommand{\TSPRPReal}{\text{TSPRP-Real}}
\newcommand{\TPRPIdeal}{\text{TPRP-Ideal}}
\newcommand{\TPRPReal}{\text{TPRP-Real}}
\newcommand{\AdvTSPRP}[2]{\AdvSTPRP{#1}{#2}}
%\newcommand{\AdvTPRP}[2]{\Adv^{\mathrm{tprp}}_{#1}(#2)}
\newcommand{\RPerm}{\Pi}
\newcommand{\RPermInv}{\Pi\inv}
%\newcommand{\tbce}{\widetilde{E}}
%\newcommand{\tbcd}{\widetilde{D}}
%\newcommand{\tbc}{\text{TBC}}
%\newcommand{\tweakspace}{\calT}
\newcommand{\noncespace}{\calN}
%\newcommand{\ctxtspace}{\calC}
\newcommand{\gf}[1]{\text{GF}(#1)}
%\newcommand{\SE}{\mathsf{SE}}
\newcommand{\xe}{\mathsf{XE}}
\newcommand{\lrw}{\mathsf{LRW}}
\newcommand{\xeenc}{\xe.\enc}
\newcommand{\xedec}{\xe.\dec}
\newcommand{\Perms}{\text{Perms}}
\newcommand{\Funs}{\text{Funs}}
\newcommand{\ocbcore}{\mathsf{OCB\dash Core}}
\newcommand{\ocbcoreenc}{\mathsf{OCB\dash Core}.\enc}
\newcommand{\ocbcoredec}{\mathsf{OCB\dash Core}.\dec}
\newcommand{\clen}{\ctxtlen}
\newcommand{\rorcpareal}{\text{RoR-CPA-Real}}
\newcommand{\rorcpaideal}{\text{RoR-CPA-Ideal}}
\newcommand{\integers}{\mathbb{Z}}

\newcommand{\dlog}{\mathsf{Dlog}_{G,g}}
\newcommand{\AdvDL}[2]{\Adv^{\mathrm{dl}}_{#1}(#2)}
\newcommand{\DL}{\textnormal{DL}}
\newcommand{\AdvCDH}[2]{\Adv^{\mathrm{cdh}}_{#1}(#2)}
\newcommand{\CDH}{\textnormal{CDH}}

\newcommand{\PRF}{\textnormal{PRF}}
\newcommand{\AdvPRF}[2]{\Adv^{\mathrm{prf}}_{#1}(#2)}


\newcommand{\Feistel}{\cipher}
\newcommand{\Maj}{\textnormal{Maj}}

\newcommand{\MRUMA}{\textnormal{MR-UMA}}
\newcommand{\MRUMAideal}{\textnormal{MR-UMA-IDEAL}}
\newcommand{\msgsampler}{\calP}
\newcommand{\mdist}{p_m}
\newcommand{\AdvMRUMA}[2]{\Adv^{\mathrm{mr\dash uma}}_{#1}(#2)}
\newcommand{\hatE}{\hat{E}}

\newcommand{\REAL}{\textnormal{REAL}}
\newcommand{\RAND}{\textnormal{RAND}}
\newcommand{\ROR}{\textnormal{ROR}}
\newcommand{\AdvROR}[2]{\Adv^{\mathrm{ror}}_{#1}(#2)}
\newcommand{\AdvRAND}[2]{\Adv^{\mathrm{\INDRAND}}_{#1}(#2)}
\newcommand{\RORCCA}{\textnormal{ROR-CCA}}
\newcommand{\AdvRORCCA}[2]{\Adv^{\mathrm{ror\dash cca}}_{#1}(#2)}

\newcommand{\LORCCA}{\textnormal{LOR-CCA}}

% \newcommand{\INDSIM}{\textnormal{INDSIM}}
\newcommand{\AdvINDSIM}[2]{\Adv^{\mathrm{ind\dash sim}}_{#1}(#2)}


\newcommand{\KgOracle}{\textnormal{Kg}}
\newcommand{\EncOracle}{\textnormal{Enc}}
\newcommand{\DecOracle}{\textnormal{Dec}}
\newcommand{\EncSim}{\textnormal{EncSim}}
\newcommand{\DecSim}{\textnormal{DecSim}}
\newcommand{\Time}{\textsf{T}}

\newcommand{\tildeC}{\tilde{C}}
\newcommand{\EM}{\textnormal{EM}}
\newcommand{\Sbox}{\textnormal{Sbox}}
\newcommand{\myInd}{\hspace*{1em}}

\newcommand{\MA}{\mathsf{MA}}
\newcommand{\mtag}{\mathsf{tag}}
\newcommand{\ver}{\mathsf{ver}}

\newcommand{\msgset}{\textsf{MsgSet}}
\newcommand{\pairset}{\textsf{PairSet}}
\newcommand{\TagOracle}{\textnormal{Tag}}
\newcommand{\VerOracle}{\textnormal{Ver}}
\newcommand{\TagSim}{\textnormal{TagSim}}

\newcommand{\cAU}{\textnormal{cAU}}
\newcommand{\AdvcAU}[2]{\Adv^{\mathrm{cau}}_{#1}(#2)}

\newcommand{\specialcell}[2][c]{%
  \begin{tabular}[#1]{@{}l@{}}#2\end{tabular}}

\newcommand{\OWF}{\textnormal{OWF}}

\newcommand{\SPR}{\textnormal{SPR}}
\newcommand{\AdvSPR}[2]{\Adv^{\mathrm{spr}}_{#1}(#2)}

\newcommand{\PWR}{\textnormal{PWR}}
\newcommand{\AdvPWR}[2]{\Adv^{\mathrm{pwr}}_{#1}(#2)}
\newcommand{\pw}{pw}
\newcommand{\salt}{sa}

\newcommand{\Hinfty}{H^{\infty}}
\newcommand{\Hshan}{H}

\newcommand{\simoracle}{\textsf{Sim}}
\newcommand{\foracle}{\textsf{f}}
\newcommand{\FnOracle}{\textnormal{Fn}}
\newcommand{\INDIFF}{\textnormal{Indiff}}
\newcommand{\AdvINDIFF}[2]{\Adv^{\mathrm{indiff}}_{#1}(#2)}


\newcommand{\RKAPRF}{\textnormal{RKA-PRF}}
\newcommand{\AdvRKAPRF}[2]{\Adv^{\mathrm{rka\dash prf}}_{#1}(#2)}

\newcommand{\ipad}{\mathrm{ipad}}
\newcommand{\opad}{\mathrm{opad}}
\newcommand{\modify}[1]{{\color{blue} #1 }}
\newcommand{\ftable}{\mathtt{f}}
\newcommand{\Htable}{\mathtt{H}}

\newenvironment{proofsketch}{%
  \renewcommand{\proofname}{Proof Sketch}\proof}{\endproof}


\newcommand{\DDH}{\textnormal{DDH}}

\newcommand{\ICDH}{\textnormal{ICDH}}
\newcommand{\DDHoracle}{\textnormal{DHCheck}}


\newcommand{\RSAk}{\textnormal{RSA}\dash k}
\newcommand{\AdvOWFRSA}[2]{\Adv^{\mathrm{owf}}_{#1}(#2)}


\newcommand{\DS}{{\mathsf{DS}}}
\newcommand{\sign}{{\mathsf{sign}}}

\newcommand{\SignOracle}{\textnormal{Sign}}
\newcommand{\HashSim}{\textnormal{HSim}}
\newcommand{\SignSim}{\textnormal{SignSim}}


\newcommand{\IDscheme}{\textsf{ID}}
\newcommand{\IDkeygen}{\textsf{kg}}
\newcommand{\IDkg}{\textsf{IDkg}}
\newcommand{\IDcom}{P.com}
\newcommand{\IDresp}{P.resp}
\newcommand{\IDver}{V.ver}
\newcommand{\IDchalset}{\calC}


\newcommand{\IDPASS}{\textnormal{IDPASS}}

\newcommand{\queried}{\textsf{queried}}

\newcommand{\AdvIDPASS}[2]{\Adv^{\mathrm{idpass}}_{#1}(#2)}
%\newcommand{\ver}{{\mathsf{ver}}}

\newcommand{\coins}{\omega}
\newcommand{\CoinSpace}{\Omega}


\newcommand{\NIZK}{\textnormal{NIZK}}
\newcommand{\Gen}{\textsf{Gen}}
\newcommand{\Simu}{\textsf{Sim}}
\newcommand{\ProofOracle}{\textnormal{Prove}}

\newcommand{\pair}{\textbf{e}}
\newcommand{\mpk}{mpk}
\newcommand{\msk}{msk}
\newcommand{\IBEpg}{\textsf{pg}}
\newcommand{\IBEkg}{\textsf{kg}}
\newcommand{\IBEenc}{\textsf{enc}}
\newcommand{\IBEdec}{\textsf{dec}}
\newcommand{\id}{id}


\newcommand{\IBE}{\textsf{IBE}}
\newcommand{\INDIDCPA}{\textnormal{IND-ID-CPA}}
%%% Local Variables:
%%% mode: latex
%%% TeX-master: "main"
%%% End:


\usepackage{pgfplots}
    \pgfplotsset{
        compat=1.12,
    }
\usepackage{amsmath}

\usepackage{algorithm}
\usepackage{algpseudocode}
\usepackage[n,advantage,operators,sets,adversary,landau,probability,notions,logic,
 ff,mm,primitives,events,complexity,asymptotics,keys]{cryptocode}

\addbibresource{notes.bib}


\title{\textbf{Designing Secure Cryptography}}

\author{
  Cornell CS 6831
}


\ifnum\submission=1
\date{}
\fi


\begin{document}

\maketitle

\tableofcontents


\section*{Notes}

\tnote{A couple notes for myself, will remove later.}
\begin{itemize}
  \item Fix a pseudocode language, including randomness, and an associated model
    of computation. Treat memory usage as well as run times of algorithms. We
    will most often treat memory as subservient to run-time, bounding the former
    by the latter. But special treatments are needed.
    The goal is to have a simplified abstraction, but which
    provides good predictions of running code.  Probably look at ``Careful with
    Composition'' paper as starting point.

 \item Algorithms are simply pseudocode that take input and produce an output.
    Define what it means for algorithm to be runnable. Define adversaries
    as algorithms and require they be runnable. Comment on how this interacts
    with (and rules out) non-uniform reductions, other such things.

  \item  Algorithms can be parameterized, but this is just notational sugar. The result
    must still be runnable. They may have oracles, and this defines some
    interface.  Give conventions on resource usage of algorithms: number of
    oracle queries, run time (with oracle queries unit cost), etc. Discuss how
    runnable algorithms can then be composed.

  \item Associate probability space to any algorithm, be pedantic here since we
    may want to prove some basic stuff via direct manipulation of probability
    space (i.e., coin-counting arguments).


  \item Introduce security definitions as special algorithms called games. The
    game is just another algorithm parameterized by some adversary, scheme, etc.
    Thus typically what we define as a game is actually a family of algorithms,
    one for each instantiation of things. (How much notion of a template do we
    need? Is this confusing?) It is used to measure adversarial advantage, a
    which is a measure of success for an adversary.

  \item Discuss the viewpoint underlying all the above. Security models are
    coarse abstractions of real cryptographic systems. They must be coarse for
    us to do certain kinds of analysis. This means that what we prove in our
    formalizations do not typically apply to real systems. But they are a good
    heuristic: schemes for which we have good analyses tend to provide better
    security. We will therefore judge the value of models by their utility in
    helping us build cryptographic schemes that resist attackers in practice.
    We will include examples and discussions reinforcing this viewpoint along
    the way. (One example to treat right away is asymptotics. Could be good
    heuristic, but is often quite poor, and gives little advice about how to set security
    parameters meaningfully.)

  \item Discuss assumptions. These are formalized as games. What separates them
    from security definitions is contextual. They are ``lower level'' and
    less related to the goals of cryptographic protocols.

  \item Reductions are runnable algorithms.

  \item Using reductions to set security parameters. Maybe interleave some of
    this discussion with first couple of chapters
\end{itemize}


\newpage
\input{intro}
\newpage
\input{notation}
%\newpage
%\input{ciphers}
\newpage
\input{ciphers-draft}
\newpage
\input{prf}
\input{prp/all}
\newpage
\input{cryptanalysis}
\newpage
%%%%%%%%%%%%%%%%%%%%%%%%%%%%%%%%%%%%%%%%%%%%%%%%%%%%%%%%%%%%%%%%%%%%%%%%%%%%%%%%
\section{Deterministic Encryption and Frequency Analysis}
\label{sec:freqanalysis}

We have seen how cryptanalysts design block ciphers for practical use, such as AES and DES, with the hope that enough study of such ciphers will allow us to create stronger ciphers before these are broken.
 Such ciphers are used both as a building block towards randomized and authenticating encryption but also as applications themselves.

\paragraph{Length preserving within encryption.} 
Length preserving encryption (LPE) is simply encryption where a message and its ciphertext have the same length. The need for such encryption originated with hard disks. If private information is stored on some unencrypted physical disk, stealing the information is just a matter of stealing the disk itself. However, if the encryption is expanding then we could use much more (valuable) disk space. If we can use length preserving encryption on the disk layer, than deleting the secret key effectively erases the hard drive. There are many implementations of LPE including ECB-Mask-ECB (EME) \cite{Halevi2004EME}, CBC-Mask-CBC (CMC) \cite{Halevi2003CMC}, and linear-Transformation; linear-Transformation (TET) \cite{Halevi2007TET}.

\paragraph{Format preserving encryption.}
Format preserving encryption (FPE) is a superset of LPE. In FPE, we may want a ciphertext to be the same length as the message, but we may also want more features of the message to be preserved. This notion has existed for decades informally \cite{commerce1981FPE} \cite{Brightwell1997FPE}, but the motivation for the first more formal presentation of FPE was so that credit card companies could transition databases that held credit card numbers in plaintext into encrypted databases \cite{Rogaway2002FPE}. Software using the numbers assumed a certain syntactic format, so it was useful to have the encryption preserve such a format in addition to length.

It is natural that encryption for sensitive databases would be deterministic. As the main function of databases is to actually use the data, it would be ideal if database queries were supported on the encrypted database. For example, it would be useful for the encrypted database to support equality search (find all rows where the name is ``Alice"), range queries (find all rows where the age is between 20 and 30), and to return sorted lists (return the records ordered by salary). Whether or not the database supports such operations depends on the method of encryption. 

\paragraph{Proxy-Based encrypted database} One simple approach is proxy-based. That is, there is some proxy between the client and the server. The client only sees plaintext and the server holds the entire database but with every entry encrypted. The client sends its queries to the proxy which encrypts information from the client entry by entry. The proxy then gives the server the same query with all the data encrypted. The server returns back encrypted data which is decrypted by the proxy and sent back to the client. 

In this case deterministic encryption is useful. Given, say, a search query for all rows with Name=``Alice", the proxy can simply ask the server for all rows where the value in the corresponding column is the encryption of ``Alice" and return to the client the decryption of the rows returned by the server. Because encryption is deterministic, ``Alice" encrypts to the same value every time, so the proxy only needs to search the one value.

This approach, however has certain drawbacks in that it reveals information about plaintexts. The server can see every time a plaintext is encrypted more than once with the same key.
Thus the server has access to the plaintext frequency information of the database. As each column uses one key throughout the column, the server can see when plaintexts are repeated in a column.
%In real databases, columns hold names, short numbers, or genders so plaintexts are often repeated.
This will allow for what we will call a \textit{frequency analysis attack}.
    
One possible defense might be to use a different key for every entry so that messages do not encrypt to the same ciphertext every time. However, the proxy must then encrypt every entry in a column in order to do a search query, which  is detrimental to the efficiency of the proxy. However, solutions which try to hide frequency information with less computational overhead, such as frequency smoothing \cite{Lacharite2018FSE}, have been proposed.

\paragraph{Frequency analysis attacks.}
    Why is the knowledge of a repeated plaintext sensitive information? Suppose the most common name is ``Bob", and in the US 20\% of people are named ``Bob". If I have a database which is sampled fairly uniformly among people in the US and 20\% of the ciphertexts in the name column are $0x011010$, I can be fairly confident that the decryption of this name is in fact ``Bob". This is already more information than we would like to share with an untrusted server, but if we can easily recover a high percentage of names, and we can in a similar fashion recover a high percentage of information in other columns, we reveal information throughout specific rows and possibly of specific users.
    
    More generally, a frequency analysis attack uses the adversary's knowledge of the distribution of messages to match ciphertexts to plaintexts. That is, given a list of ciphertexts which were encrypted from messages sampled from a known distribution, the adversary attempts to correctly guess which message encrypts to which ciphertext. Consider the following game $\MRUMA$:
    
\begin{figure}[H]
\centering
\fpage{.22}{
		\underline{$\MRUMA^{\advA}_{\cipher,\mdist,q}$}\\[1pt]
		$K \getsr \keyspace$\\
    For $i = 1$ to $q$\\
    \myInd $M_i \get{\mdist} \msgspace$\\
    \myInd $C_i \gets E_K(M_i)$\\
		$\hatE \getsr \advA(C_1,\ldots,C_q)$\\
    Ret $\forall_{i=1}^q \left(\hatE(M_i) = C_i\right)$
	}
\end{figure}

Here we assume that the adversary $\advA$ has access to the probability distribution $\mdist$, and this distribution can be sampled efficiently. The notation $\get{\mdist}$ means to sample according to $\mdist$. Thus the adversary wins if it can find which message produced which ciphertext and return the resulting map (encryption scheme).

The advantage that an adversary would like to maximize is the following:
\bnm
   \AdvMRUMA{\cipher,\mdist,q}{\calA} = \Prob{\MRUMA^\advA_{\cipher,\mdist,q}\Rightarrow\true}
\enm

We claim that, for a ``good" encryption scheme, the best adversary for $\MRUMA$ simply matches the $i$-th most common ciphertext with the $i$-th most probable message. Formally, let the adversary $\advA^*$ be the following:

%Such databases need to support certain database operations, such as search  ideally while encrypted.
\begin{figure}[H]
 \centering
\fpage{.40}{

		\underline{$\advA^*(C_1,\ldots,C_q)$}\\[1pt]
    Let $c$ be number of unique ciphertexts in $C_1,\ldots,C_q$\\
    Let $\tildeC_1,\ldots,\tildeC_c$ be unique ciphertexts\\
    Let $N_{\tildeC_i}$ be number of occurences of $\tildeC_i$\\
    $\hatE \gets \argmax_f \prod_{i=1}^{c} \mdist\left(f^{-1}(\tildeC_i)\right)^{N_{\tildeC_i}}$\\
    Ret $\hatE$
	}  
\end{figure}

	
 Note that this adversary literally picks the most likely map. For every map of messages to ciphertexts, $\advA^*$ chooses the one that was most likely to result in $\tildeC_i$ occurring $N_{\tildeC_i}$ times. Because multiplication is an increasing function, this happens to be the map which matches the most common ciphertexts with the most probable messages.
 
Theorem \ref{freq-opt} shows that any adversary which does significantly better that $\advA^*$ ``breaks" the cipher, in the sense of distinguishing it from a PRP.
\begin{theorem}[Optimality Frequency Analysis Attack]
	\label{freq-opt}
Let $\cipher$ be a cipher, $\mdist$ a message distribution, and $q>0$. Let
$\advA^*$ be the frequency analysis $\MRUMA_{\cipher,q}$-adversary and $\advA$
be some $\MRUMA_{\cipher,q}$-adversary. Then we give a
$\PRP_\cipher$-adversary $\advB$ such that
\bnm
  \AdvMRUMA{\cipher,\mdist,q}{\advA} \le 
        \AdvMRUMA{\cipher,\mdist,q}{\advA^*} + \AdvPRP{\cipher}{\advB}
\enm
$\advB$ makes $q$ queries and runs in time 
$\Time(\advA) + 2q+q\cdotsm\Time(\mdist)$.
\end{theorem}

 \begin{proof}[\thref{freq-opt}]We begin by considering the following game which is the same as $\MRUMA$ except instead of a cipher, we use a random permutation\footnote{We use a permutation to avoid considering noise in the frequencies caused by function collisions.}.

\begin{figure}[H]
\centering
\fpage{.22}{
		\underline{$\G1$}\\[1pt]
		$\pi \getsr \Perm(\msgspace)$\\
     For $i = 1$ to $q$\\
    \myInd $M_i \get{\mdist} \msgspace$\\
    \myInd $C_i \gets \pi(M_i)$\\
    $\hatE \getsr \advA(C_1,\ldots,C_q)$\\
    Ret $\forall_{i=1}^q \left(\hatE(M_i) = C_i\right)$
	}
\end{figure}

% Let the advantage for this game be the following:
% $$\Adv^{\textrm{G1}}_{\mdist,q}(\advA)=\Prob{\textrm{G1}^\advA_{\mdist,q}\Rightarrow\true}.$$

A random permutation can be considered a perfect cipher in that it is distinguishable from a PRP with probability 0 because it is one. If the theorem holds then certainly so must the following lemma.

\begin{lemma}
\label{freqsidelem}
Let $\mdist$ be a message distribution, and $q>0$. For any G1-adversary $\advA$, $\Adv^{\textrm{G1}}_{\mdist,q}(\advA)\leq\Adv^{\textrm{G1}}_{\mdist,q}(\advA^*)$.
\end{lemma}

\begin{proof}[\lemref{freqsidelem}]

The lemma asserts that $A^*$ is the adversary with the highest advantage for G1.  This is because we are looking to maximize 
$\Prob{\textrm{G1}^\advA_{\mdist,q}\Rightarrow\true}$
and so we want $\advA$ to output the function $\hatE$ which maximizes $\Prob{\forall_{i=1}^q \left(\hatE(M_i) = C_i\right)}$. In other words, the best we can do is 

$$\hatE=\argmax_f \textrm{ } \Prob{\forall_{i=1}^q \left(f(M_i) = C_i\right)}.$$

This probability is taken over all messages and permutations, so we have:

$$\hatE= \argmax_f \sum_{\pi\in\Perm(\msgspace)}\CondProb{\pi=\rho}{\rho\getsr\Perm(\msgspace)} \sum_{m_1, \ldots,m_q\in \msgspace} \mdist(m_1)\ldots\mdist(m_q)\Prob{\forall_{i=1}^q (f(m_i)=\pi(m_i))}.$$

Because $\CondProb{\pi=\rho}{\rho\getsr\Perm(\msgspace)} $ is a positive constant, we can pull it out of the sum and then the $\argmax$ to get

$$\hatE=\argmax_f \sum_{\pi\in\Perm(\msgspace)}\sum_{m_1, \ldots,m_q\in \msgspace} \mdist(m_1)\ldots\mdist(m_q)\Prob{\forall_{i=1}^q (f(m_i)=\pi(m_i))}.$$

 Now, if $\advA$ is deterministic, then $\Prob{\forall_{i=1}^q (f(m_i)=\pi(m_i))}$ is 1 or 0 because $\advA$ outputs the same $f$ whenever $\pi$ and $m_1,\ldots,m_q$ are the same. In this sense, $\Prob{\forall_{i=1}^q (f(m_i)=\pi(m_i))}$ is the same as 1 if $\forall_{i=1}^q f(m_i)=\pi(m_i)$ and 0 otherwise. Thus we have

$$\hatE=\argmax_f \sum_{m_1, \ldots,m_q\in \msgspace }\sum_{\substack{\pi\in\Perm \\ \forall_{i=1}^q f(m_i)=\pi(m_i)}} \mdist(m_1)\ldots\mdist(m_q).$$

Becuase $\pi$ is a permutation and thus a bijection, we can replace $m_i$ in our bounds with $c_i$ such that $\pi(m_i)=c_i$ to get 

$$\hatE=\argmax_f \sum_{c_1, \ldots,c_q\in \msgspace }\sum_{\substack{\pi\in\Perm \\\forall_{i=1}^q f(\pi^{-1}(c_i))=c_i}} \mdist(\pi^{-1}(c_1))\ldots\mdist(\pi^{-1}(c_q)).$$

We can disregard $f$ which is not 1-1 or onto because for all such $f$,  $\forall_{i=1}^q f(m_i)=\pi(m_i)$ will not hold. This is useful so that $f^{-1}$ is well defined. This means the condition $\forall_{i=1}^q f(\pi^{-1}(m_i))=c_i$ is equivalent to $\forall_{i=1}^q \pi^{-1}(m_i)=f^{-1}(c_i)$. Now we have

\begin{align*}
   \hatE&=\argmax_f \sum_{c_1, \ldots,c_q\in \msgspace }\sum_{\substack{\pi\in\Perm \\\forall_{i=1}^q \pi^{-1}(c_i)=f^{-1}(c_i)}} \mdist(f^{-1}(c_1))\ldots\mdist(f^{-1}(c_q))\\
    &=\argmax_f \textrm{ }  \mdist(f^{-1}(c_1))\ldots\mdist(f^{-1}(c_q)).
\end{align*}


The last equality holds because we can mazimize a sum by maximizing the summand. Thus the best function we can hope for is the one returned by $\advA^*$, so long as the adversary must be deterministic.

If $\advA$ can be nondeterministic, it still cannot beat $\advA^*$. This is because the probability of success is now over all sets of random coins that $\advA^*$ receives. The coins that it receives are based only on $q$, so if for any $q$, the probability of success for $\advA$ is greater than the probability of success of $\advA^*$, then there is one set of random coins for which the probability of success for $\advA$ is greater than the probability of success of $\advA^*$. This implies that for each $q$, there is a deterministic machine which beats $\advA^*$, which we have proven in the previous lemma is false. Thus $\advA^*$ is actually the best deterministic or nondeterministic machine for G1.
\end{proof}

To prove the theorem, we will define $\advB$ based on $\advA$. Remember that because $\advB$ is a $\PRP_\cipher$-adversary, it is given access to an oracle and is successful if it can decide if the oracle is a random permutation or not.
\begin{figure}[H]
\centering
\fpage{.22}{
		\underline{$\advB^\textrm{Fn}$}\\[1pt]
	%$\rho \getsr \Perm(\msgspace)$\\
    For $i = 1$ to $q$\\
    \myInd $M_i \get{\mdist} \msgspace$\\
    \myInd $C_i \gets \textrm{Fn}(M_i)$\\
    If $\exists i, j$, $M_i\neq M_j$, $C_i=C_j$\\ \myInd return 0\\
    $\hatE \getsr \advA(C_1,\ldots,C_q)$\\
    If $\forall_{i=1}^q \left(\hatE(M_i) = C_i\right)$\\ \myInd return 1\\ else\\ \myInd return 0
	}
\end{figure}

Here, $\textrm{F}_0$ is a random That is, $\advB$ returns 1 if and only if $\advA$ ``wins" its $\MRUMA$ game. This means that $\Prob{\PRP1_\cipher^\advA\Rightarrow 1}= \AdvMRUMA{\cipher,\mdist,q}{\advA}$ and
$\Prob{\PRP0_\cipher^\advA\Rightarrow1}=\Adv^{\textrm{G1}}_{\cipher,\mdist,q}(\advA)$
In our analysis of G1, we have shown that $\Adv^{\textrm{G1}}_{\cipher,\mdist,q}(\advA)\leq\Adv^{\textrm{G1}}_{\cipher,\mdist,q}(\advA^*)$. Thus we have

\begin{align*}
    \AdvPRP{\cipher}{\advB}&=\left| \AdvMRUMA{\cipher,\mdist,q}{\advA}-\Adv^{\textrm{G1}}_{\mdist,q}(\advA) \right|\\
    \AdvMRUMA{\cipher,\mdist,q}{\advA}&\leq\AdvPRP{\cipher}{\advB}+\Adv^{\textrm{G1}}_{\mdist,q}(\advA)\\
    \AdvMRUMA{\cipher,\mdist,q}{\advA}&\leq\AdvPRP{\cipher}{\advB}+\Adv^{\textrm{G1}}_{\mdist,q}(\advA^*).
\end{align*}

However, because of the nature of $\advA^*$, it will be correct with the same probability given the same messages no matter what encryption function is used, so long as the ciphertexts of different messages are different. If the ciphertexts of different messages are the same, we automatically return 0, as the function is clearly not a random permutation. This only increases the advantage of $\advB$. Thus 
$$ \AdvMRUMA{\cipher,\mdist,q}{\advA} \le 
        \AdvMRUMA{\cipher,\mdist,q}{\advA^*} + \AdvPRP{\cipher}{\advB}.$$

\end{proof}

\paragraph{Frequency analysis in practice}
In actuality, the assumptions we have made for \thref{freq-opt} do not necessarily hold. For example the adversary likely does not have access to the exact probability distribution of any column in any database. Thus it is unclear whether this attack should work in practice. However, there have been simulated case studies which have shown the effectiveness of $\advA^*$ despite practical issues.

Generally, these case studies use machine learning techniques on some representative dataset, such as healthcare data or even a breach dataset, to create a message probability distribution $\hat p_m$ which estimates $\mdist$. Then they run $\advA^*$ with $\hat p_m$ many times on some independent dataset to find its the success rate.

The success of such attacks has so far depended on factors such as the size and uniformity of the message space. If a message space is small and very nonuniform then the attack should work better. For example, if the message is 0 with 99\% probability and 1 with 1\% probability, it should be obvious with enough samples which ciphertext decrypts to which bit. The closer the probabilities move to 50\% each, or the more messages in the message space, the less clear it will be.

In actual experiments, this is exactly what happened. In a hospital dataset in which most columns had few possibilities and were nonuniform, 100\% of the deterministically encrypted values were recovered \cite{Bindschaelder2018tao}. Similarly, \cite{Naveed2015inference} saw that attributes like sex and whether a patient died during their stay were much more easily recoverable than age or admission month, which have more values and are more uniform.

\paragraph{Problem 5.1 (From Boneh-Shoup book \cite{BonehShoupBook})} Suppose we are given a block cipher $(E,D)$ operating
on domain $X$ . We want a block cipher $(E', D')$ that operates on a smaller domain $X'\subseteq X$ . Define $(E', D')$ as follows:

\begin{align*}
   E'(k, x) := \myInd &y \leftarrow E(k, x)\\
    &\textrm{while } y \not\in X' \textrm{ do}: y \leftarrow E(k, y)\\
    &\textrm{output }  y 
\end{align*}

$D'(k, y)$ is defined analogously, applying $D(k, ·)$ until the result falls in $X'$. Clearly $(E', D')$ are defined on domain $X'$.

\begin{enumerate}
    \item With $t := |X|/|X'|$, how many evaluations of $E$ are needed in expectation to evaluate $E'(k, x)$ as a function of $t$? You answer shows that when $t$ is small (e.g., $t \leq 2$) evaluating $E'(k, x)$ can be done efficiently.
    
    \item Show that if $(E,D)$ is a secure block cipher with domain $X$ then $(E, D')$ is a secure block cipher with domain $X'$. Try proving security by induction on $|X |-|X'|$.
\end{enumerate}

\newpage
\input{tweakciphers}
\newpage
\input{symenc}
\newpage
\input{authenc}
\newpage
\input{msgauth}
\newpage
\input{genericcomp}
\newpage
\input{advanced-ae}
\newpage
\input{aepractice}
\newpage
\input{hashfunctions}
\newpage
\input{hashfunctions2}
\input{hash/all}
\newpage
\section{PRFs from Hash functions}

\subsection{Other Hash based PRFs}
The Merkle Damg{\aa}rd construction from the previous section is a type of Prefix MAC (since the key is prepended to the message). Although the construction is a secure PRF in the random oracle model, we saw how it is susceptible to length extension attacks. Several alternatives have been proposed to counter these attacks. We use $\hash^f$ to represent the Merkle Damg{\aa}rd construction with $f$ as the underlying compression function.  

\paragraph{Suffix MAC:}
$\hash^f(M \Vert K)$ - The key is appended to the end instead of the beginning. Now, for a length extension attack to succeed, the adversary must find a collision for hash output of the message prefix until the key is added. To elaborate, if the adversary finds messages $M_0$ and $M_1$ such that $\hash^f(M_0) = \hash^f(M_1)$ by an offline collision attack, then it can break security for the overall construction without ever needing the key. To prove \PRF-security, therefore, we require that $\hash$ be collision resistant.


\paragraph{Envelope MAC:} $\hash^f(K_1 \Vert M \Vert K_2)$ - A key $K_1$ is prepended and another key $K_2$ is appended to the input. The \PRF-security proof is similar to the one before except now collision resistance is no longer required.

\paragraph{Nested MAC:}
$\hash^f(K_1 \Vert \hash^f(K_2 \Vert M))$ - The standard Prefix MAC construction is applied to the message twice with two \textit{independently} chosen keys $K_1$ and $K_2$.


\subsection{The HMAC Construction}

$\HMAC$, a widely used MAC in practice is a close variant of the nested MAC. For $\HMAC$, The keys $K_1$ and $K_2$ are no longer picked independently but rather derived from some base key $K$. The keys are computed as $K_1 = f(\IV, K \oplus \ipad)$ and $K_2 = f(\IV, K \oplus \opad)$ where $\ipad$ (inner pad) and $\opad$ (outer pad) are constants. The usage of a single base key rather than two allows $\HMAC$ to be built using existing SHA hash functions in a black-box way. Figure \ref{fig: HMAC construction} shows both a shortened and an extended version of the $\HMAC$ construction

\begin{figure}[h]
\centering
\subcaptionbox{%
    % 
    \label{fig:HMAC 1}
  }[0.3\linewidth] {
   \begin{tikzpicture}[scale=0.4]
            \begin{scope}[]
                \node [draw,trapezium,trapezium left angle=70,trapezium right angle=70,minimum height=0.75cm,thick,fill=orange!15,shift={(1.15,0)},rotate=-90] 
                {\begin{sideways}\Large $\hash$ \end{sideways}};
                \draw[->,thick] (0,0) node[left] {$K_1 \Vert M $} -- (1.9,0);
                \draw[->,thick] ++(4,0) -- ++(1.5,0) -- ++(0,-2.5);
            \end{scope}
            \begin{scope}[shift={(6.2,-3.25)}]
                \node [draw,trapezium,trapezium left angle=70,trapezium right angle=70,minimum height=0.75cm,thick,fill=orange!15,shift={(1.15,0)},rotate=-90] 
                {\begin{sideways}\Large $\hash$ \end{sideways}};
                \draw[->,thick] (0,0) node[left] {$K_2 \Vert h$} -- (1.9,0);
                \draw[->,thick] ++(4,0) -- ++(2,0) node[right] {$Y$};
            \end{scope}
        \end{tikzpicture}
    }
\hfill
\subcaptionbox{%
   % \label{fig:HMAC 2}
  }[0.5 \linewidth]
  { 
  \begin{tikzpicture}[scale=0.4]
            \begin{scope}[]
                \node [draw,trapezium,trapezium left angle=50,trapezium right angle=90,minimum height=0.5cm,thick,fill=orange!15,shift={(1.15,0.3)},rotate=-90] 
                {\begin{sideways}\Large$f$\end{sideways}};
                \draw[->,thick] ++(0.5,+3) node[above] {$K \oplus \ipad$} -- ++(0,-1) -- ++(1.7,0);
                \draw[->,thick] ++(0,0.5) node[left] {IV} -- ++(2.2,0);
            \end{scope}

            \begin{scope}[shift={(3.5,0)}]
                \node [draw,trapezium,trapezium left angle=50,trapezium right angle=90,minimum height=0.5cm,thick,fill=orange!15,shift={(1.15,0.3)},rotate=-90] 
                {\begin{sideways}\Large$f$\end{sideways}};
                \draw[->,thick] ++(0.5,+3) node[above] {$M_1$} -- ++(0,-1) -- ++(1.7,0);
                \draw[->,thick] ++(0,0.5) -- node[below] {$K_1$} ++(2.2,0);
            \end{scope}

            \begin{scope}[shift={(7,0)}]
                \node [draw,trapezium,trapezium left angle=50,trapezium right angle=90,minimum height=0.5cm,thick,fill=orange!15,shift={(1.15,0.3)},rotate=-90] 
                {\begin{sideways}\Large$f$\end{sideways}};
                \draw[->,thick] ++(0.5,+3) node[above] {$M_2$} -- ++(0,-1) -- ++(1.7,0);
                \draw[->,thick] ++(0,0.5) --  ++(2.2,0);
            \end{scope}
            
            \begin{scope}[shift={(7,-6)}]
                \node [draw,trapezium,trapezium left angle=50,trapezium right angle=90,minimum height=0.5cm,thick,fill=orange!15,shift={(1.15,0.3)},rotate=-90] 
                {\begin{sideways}\Large$f$\end{sideways}};
                \draw[->,thick] ++(0.5,+3) node[above] {$K \oplus \opad$} -- ++(0,-1) -- ++(1.7,0);
                \draw[->,thick] ++(0,0.5) node[left] {IV} --  ++(2.2,0);
            \end{scope}
            
            \begin{scope}[shift={(10.5,-6)}]
                \node [draw,trapezium,trapezium left angle=50,trapezium right angle=90,minimum height=0.5cm,thick,fill=orange!15,shift={(1.15,0.3)},rotate=-90] 
                {\begin{sideways}\Large$f$\end{sideways}};
                \draw[->,thick] ++(0,+6.4) -- ++(0.5, 0) -- ++(0,-4.4) -- ++(1.7,0);
                \draw[->,thick] ++(0,0.5) -- node[below] {$K_2$} ++(2.2,0);
                \draw[->,thick] ++(3.6,0.5) -- ++(2,0) node[right] {$Y$};
            \end{scope}
        \end{tikzpicture} 
    }
    \caption{HMAC construction} \label{fig: HMAC construction}
\end{figure}


\subsection{Analysis of HMAC}
To analyze the PRF security of the $\HMAC$ construction, we can first prove security under the assumption that the hash function is a random oracle $\Horacle$. Notice that now the dependence of keys $K_1$ and $K_2$ no longer matters; the proof will apply for the regular Nested MAC. We leave the formal proof to the reader as \ref{Exercise 1} 

While security in the RO model is a good sanity check, it is usually not sufficient by itself. Attacks on real world hash based constructions can take advantage of the underlying structure to break security.
This naturally begs the question: What assumptions are reasonable to show security of real world constructions? We argue that it might be reasonable to assume that the underlying compression function $f$ is a good PRF to prove security. We defer further details till the next section.

What does it mean exactly for $f$ to be a ``good'' PRF here? A strawman answer would be to make sure that $f$ is secure under the standard PRF security game. Unfortunately, the standard PRF game provides no room for an adversary to use the output under a related key to distinguish between the real and ideal worlds. Recall that for $\HMAC$, $K_1$ and $K_2$ are derived from a single key $K$. Even then, ideally an adversary should not be able to correlate the outputs of $f$ under the two keys. Intuitively, we want to ask whether $K_1$ and $K_2$ are indistinguishable from random bit strings \textit{even when} the adversary can see outputs under related keys. 
We resolve this mismatch by constructing a slightly different game $\RKAPRF$ (Figure \ref{fig:RKA}). Statements modified from the original PRF game are colored blue. Suppose $f$ takes as input a $d$ bit key and an $n$ bit message and compresses them to a single $n$ bit output. That is, $f: \bits^n \times \bits^d \to \bits^n$. The game is now parametrized by a set of functions $\Phi$ and the adversary can now query the oracle using a related key $\phi(K)$ for any $\phi \in \Phi$. The adversarial advantage for the $\RKAPRF$ game is defined in the standard way:
\[
\AdvRKAPRF{f, \Phi}{\advA} = 
\left| 
    \Pr[\LRKAPRFIdeal^\advA_{f,\Phi} \Rightarrow 1] 
    - \Pr[\LRKAPRFReal^\advA_{f,\Phi} \Rightarrow 1]
\right|
\]


\begin{figure}[h]
\centering
\hfpagess{.15}{.25}{
    \underline{$\LRKAPRFIdeal^\advA_{f, \modify{\Phi}}$}\\[1pt]
    $K \getsr \bits^d$\\
    $b' \getsr \advA^{\FnOracle}$\\
    Ret $b'$\medskip

    \underline{$\FnOracle(\modify{\phi}, X)$}\\
    \modify{If $\phi \in \Phi$ then}\\
    \modify{\myInd Ret $\bot$}\\
    Ret $f(X,\modify{\phi(K)})$\medskip
}
{
    \underline{$\LRKAPRFReal^\advA_{f, \modify{\Phi}}$}\\[1pt]
    $K \getsr \bits^d$\\
    $\rho \getsr \Func(\modify{\bits^n \times \bits^d},n)$\\
    $b' \getsr \advA^{\FnOracle}$\\
    Ret $b'$\medskip

    \underline{$\FnOracle(\modify{\phi}, X)$}\\
    \modify{If $\phi \in \Phi$ then}\\
    \modify{\myInd Ret $\bot$}\\
    Ret $\rho(X \modify{, \phi(K)})$\medskip
}
\caption{Related Key Attack Game} \label{fig:RKA}
\end{figure}



We note that this game modification is far more general than what might be needed for HMAC security. For our application, it is reasonable to restrict $\Phi$ to just $\phi_{\ipad}$ and $\phi_{\opad}$ where $\phi_{\ipad}(K) = K \oplus \ipad$ and $\phi_{\opad}(K) = K \oplus \opad$ and then ask whether $f$ is \RKAPRF secure. Choosing such an $f$ function would imply that $K_1$ and $K_2$ are indistinguishable from random bit strings. 

Suppose now that we have ensured that the compression function $f$ used in HMAC is \RKAPRF secure. We still need to show that the overall construction is \PRF-secure. 

\begin{lemma}
(Informal) If $f$ is $\RKAPRF$-secure and the iterated keyed hash function $\hash$ is collision resistant then the $\HMAC$ construction is $\PRF$-secure
\end{lemma}
\begin{proofsketch}
If $f$ is $\RKAPRF$-secure, then the keys $K_1$ and $K_2$ are indistinguishable from random bit strings even for adversaries that use outputs from related key functions $\phi_{\ipad}$ and $\phi_{\opad}$. Now, we have seen that the composition of a collision resistant function and a secure PRF is a secure PRF. Therefore, we can conclude that $\HMAC$ is PRF-secure
\end{proofsketch}

\noindent Some subtleties like block padding have been ignored in the above sketch. A detailed proof can be found in \cite{Bellare1996}. A stronger result that does not rely on the collision resistance of $\hash$ (but instead on some weaker assumptions) is proved in \cite{Bellare2006}

\subsection{The Indifferentiability Framework}
We've seen that security analysis in the RO model is insufficient justification for instantiating constructions with real hash functions. Understanding security of real world constructions would be a lot easier if we had a general framework that accounted for structure-abusing attacks on hash functions. Coron et al. \cite{Coron2005} suggest use of the Indifferentiability franework introduced by Maurer et al. \cite{Maurer2004} in the context of hash functions. At a high level, we model the inner compression function $f$ itself as a random oracle for fixed-length inputs and attempt to prove that the overall construction is indistinguishable from a random oracle. Indifferentiability from a random oracle can be considered approximately as secure as a random oracle. 

Figure \ref{fig:Indiff Diagram} shows the schematic representation of the framework. As usual, $\advD$ tries to distinguish between the ideal world (with a random oracle) and the real world (with the construction $\hash^f$). In the real world, $\advD$ also gets access to the underlying function $f$. In the ideal world, since $\RO$ does not depend on $f$, a simulator $\simoracle$ tries to simulate the function $f$. $\advD$ gets access to this simulator in the ideal world. Figure \ref{fig:Indiff Game} describes the $\INDIFF$ game in detail.


\begin{figure}[h]
\centering
\subcaptionbox{%
    Diagram \label{fig:Indiff Diagram}
    \label{fig:NMAC}
  }[0.5\linewidth] {
    \begin{tikzpicture}[scale=0.4]
        \begin{scope}[]
            \draw[fill=blue!30,thick] (0.5,6.5) rectangle ++(2,2) node[pos=.5] {$\hash^f$};
            \draw[fill=violet!30,thick] (4,6.5) rectangle ++(2,2) node[pos=.5] {$f$};
            \draw[fill=green!30,thick] (0,0) rectangle ++(6.5,5) node[pos=.5]  {$\advD$};
            
            \draw[->, thick] ++(2.5,7.5) -- ++(1.5,0);  % H to f
            \draw[->, thick] ++(1.5,5) -- ++(0,1.5);    % D to H
            \draw[->, thick] ++(5,5) -- ++(0,1.5);      % D to f
            
            \draw[->,thick] ++(3.25,0) -- ++(0,-1) -- ++(1.5,0) node[right] {0/1};
        \end{scope}
    
        \begin{scope}[shift={(10,0)}]
            \draw[fill=violet!30,thick] (0.5,6.5) rectangle ++(2,2) node[pos=.5] {$\RO$};
            \draw[fill=cyan!30,thick] (4,6.5) rectangle ++(2,2) node[pos=.5] {$\simoracle$};
            \draw[fill=green!30,thick] (0,0) rectangle ++(6.5,5) node[pos=.5]  {$\advD$};
            
            \draw[<-, thick] ++(2.5,7.5) -- ++(1.5,0);  % RO to Sim
            \draw[->, thick] ++(1.5,5) -- ++(0,1.5);    % D to RO
            \draw[->, thick] ++(5,5) -- ++(0,1.5);      % D to Sim
            
            \draw[->,thick] ++(3.25,0) -- ++(0,-1) -- ++(1.5,0) node[right] {0/1};
        \end{scope}
    
    \end{tikzpicture}
    }
\hfill
\subcaptionbox{%
   Game \label{fig:Indiff Game}
  }[0.4 \linewidth]
  { 
       \hfpagess{.15}{.15}{
            \underline{$\INDIFF1^\advD_{\hash, f}$}\\[1pt]
            $b' \getsr \advD^{\FnOracle, f}$\\
            Ret $b'$\medskip
        
            \underline{$\FnOracle(M)$}\\
            Ret $\hash^f(M)$\medskip
            
            \underline{$f(X)$}\\
            If $\ftable[X] = \bot$ then\\
            \myInd $\ftable[X] \getsr \bits^n$\\
            Ret $\ftable[X]$
        }
        {
            \underline{$\INDIFF0^\advD_{\Horacle, \simoracle}$}\\[1pt]
            $b' \getsr \advD^{\FnOracle, \simoracle}$\\
            Ret $b'$\medskip
        
            \underline{$\FnOracle(M)$}\\
            If $\Htable[M] = \bot$ then\\
            \myInd $\Htable[M] \getsr \bits^n$\\
            Ret $\Htable[M]$\\
            
            \underline{$\simoracle(X)$}\\
            Ret $\mathcal{S}^\FnOracle[X]$
        }
    }
    \caption{Indifferentiability from a Random Oracle} \label{fig:Indiff}
\end{figure}

\noindent We define a distinguisher $\advD$'s advantage in the $\INDIFF$ game as:
\[
\AdvINDIFF{\hash, f, \mathcal{S}}{\advD} = 
\left| 
    \Pr[\INDIFF1^\advD_{\hash,f} \Rightarrow 1] 
    - \Pr[\INDIFF0^\advD_{\Horacle,\mathcal{S}} \Rightarrow 1]
\right|
\]

\begin{wrapfigure}{r}{1.5in}
    \centering
    \fpage{.2}{
            \underline{$\textbf{adversary } \advD^{\FnOracle, O}$}\\[1pt]
            $m_1, m_2 \getsr \bits^d$\\
            $y_1 \gets \FnOracle(m_1)$\\
            $y_2 \gets O(y_1,  m_2)$\\
            $y'_2 \gets \FnOracle(m_1 \Vert m_2)$\\
            Ret $(y_2 = y'_2)$
    }
    \caption{MD adversary} \label{fig:Indiff distinguisher}
\end{wrapfigure}

\noindent First, we note that a good simulator $\simoracle$ for $f$ in the ideal world may not always exist. In fact, the Merkle Damg{\aa}rd construction is actually not indifferentiable from a random oracle. Consider the $\INDIFF$-adversary $\advD$ defined in Figure~\ref{fig:Indiff distinguisher}.
In the real world, the two outputs $y_2$ and $y'_2$ will always be equal. On the other hand, in the ideal world, the probability that $y_2$ and $y'_2$ are equal is just the probability that a randomly sampled $n$ bit string is equal to the $y_2$ output by the simulator; this is equal to $\frac{1}{2^n}$. Therefore, $\advD$ distinguishes between the real and ideal worlds with high probability.

Variations of the Merkle Damg{\aa}rd construction such as the Chopped MD or the Enveloped MD are often used in practice instead since they are indifferentiable from a random oracle

\bigskip
\noindent How is this indifferentiable framework useful? It's real power is seen through the following composition theorem which allows security analysis of a construction that uses the hash function $\hash^f$ for any arbitrary game $G$

\begin{theorem}
\label{thm: indiff composition}
\textbf{Indifferentiability Composition Theorem} \cite{Maurer2004}
For any security game $G$, construction $\hash^f$ and scheme $C$, if 
\begin{enumerate}
    \item $\hash^f$ is indifferentiable from a random oracle under the assumption that $f$ is a random oracle
    \item C is provably $G$-secure in the random oracle model
\end{enumerate}
then, $C$ is provably $G$-secure using $\hash^f$ under the assumption that $f$ is a random oracle

\end{theorem}


\begin{figure}[h]
    \centering
    
    \begin{tikzpicture}[scale=0.4]
\begin{scope}[]
    \draw[fill=blue!30,thick] (0.5,6.5) rectangle ++(2,2) node[pos=.5] {$\hash^f$};
    \draw[fill=violet!30,thick] (4,6.5) rectangle ++(2,2) node[pos=.5] {$f$};
    \draw[fill=green!30,thick] (0,0) rectangle ++(6.5,5);
    \draw[fill=gray!30] (0.5,0.25) rectangle ++(5.5,1.5) node[pos=.5] {$G$};
    \draw[fill=gray!30] (0.5,2.75) rectangle ++(2,2) node[pos=.5] {$C$};
    \draw[fill=red!30] (4,2.75) rectangle ++(2,2) node[pos=.5] {$\advA$};
    
    \draw[->, thick] ++(2.5,7.5) -- ++(1.5,0);      % H to f
    \draw[->, thick] ++(1.5,4.75) -- ++(0,1.75);    % C to H
    \draw[->, thick] ++(5,4.75) -- ++(0,1.75);      % A to f
    
    \draw[<->, thick] ++(2.5,3.75) -- ++(1.5,0);    % C to A
    \draw[->, thick] ++(1.5,1.75) -- ++(0,1);       % G to C
    \draw[->, thick] ++(5,1.75) -- ++(0,1);         % G to A

    \draw[->,thick] ++(3.25,0.25) -- ++(0,-1) -- ++(1.5,0) node[right] {0/1};

    \draw[<-, thick] ++(6.5,5) -- ++(1.25,1);
    \node[] at (8.25,6) {$\advD$};

\end{scope}

\begin{scope}[shift={(10,0)}]
    \draw[<-, thick] ++(0,5) -- ++(-1.25,1);

    \draw[fill=green!30,thick] (0,0) rectangle ++(6.5,5);
    \draw[fill=red!10] (3.75,2.5) rectangle ++(2.5, 6.25) node[right] {$\advB$};

    \draw[fill=violet!30,thick] (0.5,6.5) rectangle ++(2,2) node[pos=.5] {$\RO$};
    \draw[fill=cyan!30,thick] (4,6.5) rectangle ++(2,2) node[pos=.5] {$\simoracle$};

    \draw[fill=gray!30] (0.5,0.25) rectangle ++(5.5,1.5) node[pos=.5] {$G$};
    \draw[fill=gray!30] (0.5,2.75) rectangle ++(2,2) node[pos=.5] {$C$};
    \draw[fill=red!30] (4,2.75) rectangle ++(2,2) node[pos=.5] {$\advA$};
    
    \draw[<-, thick] ++(2.5,7.5) -- ++(1.5,0);  % Sim to RO
    \draw[->, thick] ++(1.5,4.75) -- ++(0,1.75);    % C to RO
    \draw[->, thick] ++(5,4.75) -- ++(0,1.75);      % A to Sim
    
    \draw[<->, thick] ++(2.5,3.75) -- ++(1.5,0);    % C to A
    \draw[->, thick] ++(1.5,1.75) -- ++(0,1);       % G to C
    \draw[->, thick] ++(5,1.75) -- ++(0,1);         % G to A
    
    \draw[->,thick] ++(3.25,0.25) -- ++(0,-1) -- ++(1.5,0) node[right] {0/1};
\end{scope}

\begin{scope}[shift={(20,0)}]

    \draw[fill=violet!30,thick] (0.5,6.5) rectangle ++(2,2) node[pos=.5] {$\RO$};
    \draw[fill=gray!30] (0.5,0.25) rectangle ++(5.5,1.5) node[pos=.5] {$G$};
    \draw[fill=gray!30] (0.5,2.75) rectangle ++(2,2) node[pos=.5] {$C$};
    \draw[fill=red!10] (4,2.75) rectangle ++(2,2) node[pos=.5] {$\advB$};
    
    \draw[<-, thick] ++(2.5,7.5) -- ++(2.5,0) --  ++(0,-2.75);      % B to RO
    \draw[->, thick] ++(1.5,4.75) -- ++(0,1.75);    % C to RO
    
    \draw[<->, thick] ++(2.5,3.75) -- ++(1.5,0);    % C to A
    \draw[->, thick] ++(1.5,1.75) -- ++(0,1);       % G to C
    \draw[->, thick] ++(5,1.75) -- ++(0,1);         % G to A
    
    \draw[->,thick] ++(3.25,0.25) -- ++(0,-1) -- ++(1.5,0) node[right] {0/1};
\end{scope}

\end{tikzpicture}
  
    \caption{Composition Theorem}
    \label{fig:indiff composition}
\end{figure}


\begin{proofsketch}
Suppose $C$ is $G$-secure in the random oracle model. We can split an adversary $\advB$ into two parts; the part ($\advA$) that interacts with the construction and the part ($\simoracle$) that handles calls to the RO. Now, consider the code of $C, \advA$ and $G$ as a single distinguisher $\advD$ for the $\INDIFF$ game. Since $\hash^f$ is indistinguishable from RO, $\advD$'s advantage for distinguishing between the real world (with $\hash^f, f$) and the ideal world (with $\RO, \simoracle$) is small. This means that the advantage of an adversary $\advA$ for game $G$ and construction $C$ using $\hash^f$ is small as well since otherwise we could construct $\advD$ as the code of $C, \advA$ and $G$ to win the $\INDIFF$ game.
We can now conclude that $C$ is $G$-secure using $\hash^f$.
\end{proofsketch}

A detailed proof can be found in \cite{Maurer2004}. The composition theorem allows for the security proofs to be modular. A construction that is secure in the random oracle model will also be secure in the standard model when instantiated using a hash function that is indifferentiable from RO. Therefore, it is now enough to prove security of the construction in the RO model to use it with a real hash function that has already been proven indifferentiable. This has led to ``indifferentiability from RO'' becoming a core requirement for modern hash function designs like SHA3. An important caveat to note however, is that the composition theorem may not apply to all games as shown by Ristenpart, Shacham and Shrimpton \cite{Ristenpart2011}. 



\subsection*{Exercises}
\begin{enumerate}[label=\textbf{Exercise \thesection.\arabic*}, wide=0pt]
    \item \label{Exercise 1} Prove $\PRF$-security of the HMAC construction under the assumption that the hash function is a random oracle.
\end{enumerate}
\setenumerate[1]{label=\arabic*.}

\newpage
\input{pke}
\newpage
\input{ecc/ecc_main}
\newpage
\input{pke-cca}
%\newpage
%%!TEX root = main.tex
%%%%%%%%%%%%%%%%%%%%%%%%%%%%%%%%%%%%%%%%%%%%%%%%%%%%%%%%%%%%%%%%%%%%%%%%%%%%%%%%
\section{Key Exchange and Public Key Tools}
\label{sec:pke2}

In the symmetric key setting, we assume that two parties exchanging messages have access to a shared secret key. However, how do these parties share such a secret securely? Let us assume a \textit{passive adversary}, meaning that the adversary can listen in on network traffic but cannot alter or inject packets (we note that this assumption is not sufficient for real-world applications). In this section, we will present a key exchange protocol, called Diffie-Hellman anonymous key exchange. We will then go over some important assumptions in public key cryptography that allow us to prove such a protocol secure. Finally, we will end by going over another asymmetric encryption scheme called ElGamal encryption. 

\subsection{Diffie-Hellman Key Exchange}

In this section, we will go over one method to construct a secure key exchange protocol. The Diffie-Hellman key exchange protocol was created by Whitfield Diffie and Martin Hellman. This protocol does not confirm to either party sharing the secret of the identity of the other, therefore making this an \textit{anonymous} protocol. Before we describe the protocol, we will first go over some prerequisite algebra and number theory.

Let $p$ be a large prime number. We fix the group $\group = \Z_p^*=\{1,2,3,\ldots,p-1\}$, with group operation multiplication mod $p$. Recall that $\Z_p^*$ is the set of nonzero elements of $\Z_p$. We then know that $\group$ is \textit{cyclic}. This means that one can provide a group member $g \in \group$, called the \textit{generator}, such that $\group = \{g^0, g^1, g^2, \ldots, g^{p-1}\}$. For generator $g$ of cyclic group $\group$, note that all elements $e\in\Z_{|\group|}$ form a unique element $g^e$ in $\group$. Furthermore, a \textit{subgroup} $H$ of $\group$ under operation $\odot$ is defined as a subset of $\group$ that also forms a group under the operation $\odot$. 

\begin{example}
	Let $p=7$. Is 2 or 3 a generator for $\Z_7^*$?
	
	First let us note that $\Z_7^* = \{1,2,3,4,5,6\}$. If either 2 or 3 is a generator, then exponentiating this value will produce all the elements in $\Z_7^*$. Looking at the table below, we can see that 2 only produces the values $\{1,2,4\}$, while 3 does indeed produce all elements of $\Z_7^*$. We can therefore conclude that 3 is a generator for $\Z_7^*$. 
	\begin{center}
	\begin{tabular}{|c|c|c|c|c|c|c|c|}
		\hline
		$x$ & 0 & 1 & 2 & 3 & 4 & 5 & 6 \\
		\hline \hline
		$2^x \mod 7$ & 1 & 2 & 4 & 1 & 2 & 4 & 1 \\
		\hline
		$3^x \mod 7$ & 1 & 3 & 2 & 6 & 4 & 5 & 1 \\
		\hline
	\end{tabular}
	\end{center}
\end{example}

\paragraph{Diffie-Hellman key exchange.} We now describe the details of the Diffie-Hellman anonymous key exchange protocol. We assume both parties have access to public values group $\group$ and generator $g\in\group$. It executes as follows, as shown in \figref{fig:DHKE}:
\begin{enumerate}
	\item Alice chooses $x$ uniformly at random from $\Z_{|\group|}$, computes $X \gets g^x$, and sends $X$ to Bob.
	
	\item Bob chooses $y$ uniformly at random from $\Z_{|\group|}$, computes $Y \gets g^y$, and sends $Y$ to Alice.
	
	\item Upon receiving $X$, Alice computes $K \gets \hash(Y^x)$, where $\hash$ is a collision-resistant hash function. 
	
	\item Upon receiving $Y$, Bob computes $K \gets \hash(X^y)$. 
\end{enumerate}

The secret key $K$ shared by Alice and Bob must be the same since
\begin{equation*}
	Y^x = g^{yx} = g^{xy} = X^y.
\end{equation*}

\begin{figure}[H]
	\center
	\begin{tikzpicture}
	\node (rect1) [draw]  {
		\begin{minipage}[t][4cm]{2cm}
		{\centering\underline{\textbf{Alice}} \\}
		$x \getsr \Z_{|\group|}$ \\
		$X \gets g^x$
		\end{minipage}
	};
	\node (rect2) [draw, right of=rect1, node distance=6cm] {
		\begin{minipage}[t][4cm]{2cm}
		{\centering\underline{\textbf{Bob}} \\}
		$y \getsr \Z_{|\group|}$ \\
		$Y \gets g^y$
		\end{minipage}
	};
	\node (group1) [above of=rect1, node distance=3cm] {$\group,g$};
	\node (group2) [above of=rect2, node distance=3cm] {$\group,g$};
	\path[->,>=stealth'] (group1) edge (rect1);
	\path[->,>=stealth'] (group2) edge (rect2);
	
	\path[->,>=stealth'] (rect1.10) edge node[anchor=south] {$X$} ( rect2.west|-rect1.10); 
	\path[<-,>=stealth'] (rect1.-40) edge node[anchor=south] {$Y$} (rect2.west|-rect1.-40);
	
	\node (key1) at (0,-1.75) {$K \gets \hash(Y^x)$};
	\node (key2) at (6,-1.75) {$K \gets \hash(X^y)$};
	\end{tikzpicture}
	\caption{The Diffie-Hellman anonymous key exchange protocol for cyclic group $\group$ with generator $g$.}
	\label{fig:DHKE}
\end{figure}

\paragraph{Security of Diffie-Hellman key exchange.} Notice that if an adversary could easily compute $x$ given $X \gets g^x$, then the adversary could certainly compute $K$, rendering this protocol insecure. The function that calculates this value is called the \textbf{discrete logarithm function}. Thus, for Diffie-Hellman key exchange to have any chance of being secure, we at least must find a group in which it is difficult to compute the discrete logarithm. In the next section, we will formalize the discrete logarithm and other related assumptions.

\subsection{The Discrete Logarithm and Related Assumptions}

\paragraph{The Discrete Logarithm.} For group $\group$, generator $g\in\group$, and $X\in\group$, the discrete logarithm function $\dlog(X)$ finds the unique value $x \in \Z_{|\group|}$ such that $g^x = X$. We show game $\DL_{\group, g}$ where given $X = g^x$, adversary $\advA$ must find $x$. 

\begin{center}
	\fpage{.18}{
		\underline{$\DL^\advA_{\group, g}$} \\
		$x \getsr \Z_{|\group|} \ ; \ X \gets g^x$ \\
		$\hat{x} \getsr \advA(X)$ \\
		Return $\hat{x} = x$ 
	}
\end{center}

The DL-advantage of $\advA$ is defined as 
\begin{equation*}
\AdvDL{\group, g}{\advA} = \Prob{\DL^\advA_{\group, g}\Rightarrow\true}.
\end{equation*}

Consider the following adversary $\advA'$ for game $\DL_{\group, g}$:

\begin{center}
	\mpage{.25}{
		\underline{adversary $\advA'(X)$} \\
		For $i=2,\ldots,|G|-1$ do \\
		\ind If $X = g^i$ then \\
		\ind \ind Return $i$
	}
\end{center}

$\advA'$ simply brute-force searches through every possible value to find the correct exponent $x$. While $\AdvDL{\group, g}{\advA'}=1$, the running time of $\advA'$ is $\bigO(|G|)$, which becomes very slow for large groups. Other algorithms have been developed that are more efficient, such as Baby-step giant-step, whose running time is $\bigO(|G|^{0.5})$. However, this is still quite slow, and while certain groups might have more efficient algorithms, nothing faster is known for other groups. For such a group $\group$ in which the discrete logarithm problem is hard, we refer to this as the discrete logarithm assumption for group $\group$. This is formalized below.

\begin{definition}
	The \textbf{discrete logarithm (DL) assumption} holds for $\group$ if $\AdvDL{\group, g}{\advA}$ is negligible for all efficient adversaries $\advA$.
\end{definition}

As one example, for $p$ at least 2048-bits and $q = |\group|$ at least 256-bits, where $p$ and $q$ are both primes, the discrete logarithm function is believed to be hard to compute in the order $q$ subgroup of $Z_p^*$ \cite{BonehShoupBook}.

\paragraph{Computational Diffie-Hellman.} A related problem to the discrete logarithm problem is the computational Diffie-Hellman problem, which says that given $g^x, g^y \in \group$, where $x \getsr \Z_{|\group|}$ and $y \getsr \Z_{|\group|}$, it is hard to compute $g^{xy} \in \group$ \cite{BonehShoupBook}. We formalize this with game $\CDH_{\group, g}$ shown below.  

\begin{center}
	\fpage{.26}{
		\underline{$\CDH^\advA_{\group, g}$} \\
		$x,y \getsr \Z_{|\group|}$ \\
		$X \gets g^x \ ; \ Y \gets g^y \ ; \ Z \gets g^{xy}$ \\
		$\hat{Z} \getsr \advA(X,Y)$ \\
		Return $\hat{Z} = Z$ 
	}
\end{center}

The CDH-advantage of $\advA$ is defined as 
\begin{equation*}
\AdvCDH{\group, g}{\advA} = \Prob{\CDH^\advA_{\group, g}\Rightarrow\true}.
\end{equation*}

Similar to the DL assumption, the computational Diffie-Hellman assumption for group $\group$ tells us that the CDH problem is hard in $\group$.

\begin{definition}
	The \textbf{computational Diffie-Hellman (CDH) assumption} holds for $\group$ if $\AdvCDH{\group, g}{\advA}$ is negligible for all efficient adversaries $\advA$.
\end{definition}

\paragraph{Decisional Diffie-Hellman.} The decisional Diffie-Hellman problem for group $\group$ and generator $g$ says that for $x \getsr \Z_{|\group|}$, $y \getsr \Z_{|\group|}, z \getsr \Z_{|\group|}$, when given $g^x,g^y \in \group$ and either $g^z \in \group$ or $g^{xy} \in \group$, it is hard to distinguish between getting $g^z$ and $g^{xy}$. This is formalized in game $\DDH_{\group,g}$ below.  

\begin{center}
	\fpage{.20}{
		\underline{$\DDH_{\group,g}^\advA$}\\
		$b \getsr \bits$\\
		$x,y,z \getsr \Z_{|G|}$\\
		$Z_0 \gets g^z$\\
		$Z_1 \gets g^{xy}$\\
		$b' \getsr \advA(g^x,g^y,Z_b)$\\
		Return $(b' = b)$
	}

%	\hfpages{.25}{
%		\underline{$\DDH1_{\group, g}^\advA$}\\
%		$x,y,z \getsr \Z_{|\group|}$ \\
%		$X \gets g^x \ ; \ Y \gets g^y \ ; \ Z \gets g^{xy}$ \\
%		$b \getsr \advA(X,Y,Z)$ \\
%		Return $b$ 
%	}{
%		\underline{$\DDH0_{\group, g}^\advA$}\\
%		$x,y,z \getsr \Z_{|\group|}$ \\
%		$X \gets g^x \ ; \ Y \gets g^y \ ; \ Z \gets g^z$ \\
%		$b \getsr \advA(X,Y,Z)$ \\
%		Return $b$
%	}
\end{center}

The DDH-advantage of $\advA$ is defined as 
\begin{equation*}
\AdvDDH{\group, g}{\advA} = 2 \cdot \Prob{\DDH_{\group, g}^\advA\Rightarrow\true} - 1.
\end{equation*}

The decisional Diffie-Hellman assumption for group $\group$ tells us that the DDH problem is hard in $\group$. 

\begin{definition}
	The \textbf{decisional Diffie-Hellman (DDH) assumption} holds for $\group$ if $\AdvDDH{\group, g}{\advA}$ is negligible for all efficient adversaries $\advA$.
\end{definition}

We summarize the three problems mentioned in this section in \figref{fig:DL}. 

\begin{figure}[H]
	\center
	\begin{tabular}{|c|c|c|}
		\hline
		Problem & Given & Compute \\
		\hline \hline
		Discrete logarithm (DL) & $g, g^x$ & $x$ \\
		\hline
		Computational Diffie-Hellman (CDH) & $g,g^x,g^y$ & $g^{xy}$ \\
		\hline
		Decisional Diffie-Hellman (DDH) & $g, g^x, g^y, g^z$ & Is $z \equiv xy\ (\bmod \ |\group|)$? \\
		\hline
	\end{tabular}
	\caption{A summary of the discrete logarithm related problems over a cyclic group $\group$ with generator $g$ from this section. For each row, we state the name of the problem, the values given to the adversary, and what the adversary must provide to solve the problem.}
	\label{fig:DL}
\end{figure}

\paragraph{More on the Security of Diffie-Hellman Key Exchange.} We noted previously that the discrete logarithm assumption must hold for any group $\group$ used in Diffie-Hellman key exchange. However, a group that only meets the discrete logarithm assumption on its own is not sufficient to guarantee the security of the Diffie-Hellman key exchange. In fact, this protocol is only secure if and only if the computational Diffie-Hellman assumption holds.  

\subsection{ElGamal Encryption}

\begin{figure}[H]
	\center
	\hfpages{.2}{
		\underline{$\enc(X,M)$:}\\
		$y \getsr \Z_{|\group|}$ \\
		$C_1 \gets g^y$\\
		$Z \gets X^y$\\
		$C_2 \gets Z \cdot M$\\
		Return $(C_1,C_2)$
	}{
		\underline{$\dec(x,(C_1,C_2))$:}\\
		$M \gets C_2 \cdot C_1^{-x}$\\
		Return $M$
	}
	\caption{The ElGamal encryption scheme.}
	\label{fig:elgamal}
\end{figure}

\begin{wrapfigure}{r}{1.5in}
	\center
	\fpage{.20}{
		\underline{$G_0$ \;\;\; \fbox{$G_1$}}\\
		$b \getsr \bits$\\
		$x \getsr \Z_{|\group|}$\\
		$X \gets g^x$\\
		$b' \getsr \advA^\EncOracle(g,X)$\\
		Ret $(b' = b)$\medskip
		
		\underline{$\EncOracle(M_0,M_1)$}\\
		$C_1 \gets g^y$\\
		$Z \gets g^{xy}$\\
		\fbox{$z \getsr \Z_{|\group|}$ \;;\; $Z \gets g^z$}\\
		$C_2 \gets Z\cdot M_b$\\
		Ret $(C_1,C_2)$
	}
	\caption{Games for the proof of \thref{proof:elgamal}.}
	\label{fig:elgamal-games}
\end{wrapfigure}

We will now go over a well-known public key encryption scheme called \textbf{ElGamal encryption}. Let $\group$ be a cyclic group and let $g$ be a generator for $\group$. During key generation, secret key $x$ is chosen at random from $\Z_{|\group|}$, and the public key is computed as $X \gets g^x$. The encryption algorithm $\enc$ takes in public key $X$ and message $M$. It then chooses $y\getsr \Z_{|\group|}$ and computes $C_1 \gets g^y$ and $Z \gets X^y$. It then returns $(C_1, Z \cdot M)$, which is equal to $(g^y, g^{xy} \cdot M)$. The decryption algorithm $\dec$ takes in this ciphertext value in addition to the secret key $x$. It computes $M \gets C_2 \cdot C_1^{-x}$. Notice that
\begin{equation*}
	C_2 \cdot C_1^{-x} = g^{xy} \cdot M \cdot (g^y)^{-x} = M
\end{equation*}
which gives us the original message back as desired. The pseudocode is provided in \figref{fig:elgamal}.

The ElGamal scheme can be proven IND-CPA secure if DDH holds in $\group$, which we show with the following theorem. 

\begin{theorem}
\label{proof:elgamal}
	Let $\AEnc$ be the ElGamal scheme over group $\group$ with generator $g$. 
	Let $\advA$ be an $\INDCPA_\AEnc$-adversary. Then we give a $\DDH_{\group,g}$ 
	adversary $\advB$ such that 
	\bnm
	\AdvINDCPA{\AEnc}{\advA} \le 2\cdotsm \AdvDDH{\group,g}{\advB} \;.
	\enm
	Adversary $\advB$ runs in time that of $\advA$. 
\end{theorem} 
	
\begin{proof}
	We define the games in \figref{fig:elgamal-games}. Game $G_0$ has an identical output distribution as game $\INDCPA_{\AEnc}$, so $\Prob{G_0\Rightarrow\true} = \Prob{\INDCPA_{\AEnc}^\advA\Rightarrow\true}$. Game $G_1$ is the same as $G_0$ except now instead of assigning $Z \gets g^{xy}$, we choose a random value $z \getsr \Z_{|\group|}$ and assign $Z \gets g^z$. Now consider the following adversary $\advB$ playing game $\DDH_{\group, g}$. 
	
	\begin{center}
		\fpage{.20}{
			\underline{Adversary $\advB(X,Y,Z)$}\\
			$r \getsr \bits$\\
			$r' \getsr \advA^\EncSim(g,X)$\\
			If $(r'=r)$ then return 1 \\
			Else return 0 \medskip
			
			\underline{$\EncSim(M_0,M_1)$}\\
			$C_1 \gets Y$\\
			$C_2 \gets Z\cdot M_r$\\
			Ret $(C_1,C_2)$
		}
	\end{center}
	
	It takes inputs $X,Y,Z$ and must determine which value of $Z$ it was given. Recall that $X = g^x$ and $Y = g^y$. $\advB$ first chooses a bit $r$ at random. It then runs adversary $\advA$ with a simulation of its encryption oracle, called $\EncSim$, and gives $\advA$ both $g$ and $X$, which acts as the public key. If $\advA$ returns the correct bit, then $\advB$ returns 1; otherwise, it returns 0. We now have that
	\begin{align*}
		\AdvDDH{\group,g}{\advB} &= 2 \cdot \Prob{\DDH_{\group, g}^\advB\Rightarrow\true} - 1 \\
		&=
		\begin{aligned}
			2 \cdot (&\Prob{\DDH_{\group, g}^\advB\Rightarrow\true \wedge b=1} + \\
			&\Prob{\DDH_{\group, g}^\advB\Rightarrow\true \wedge b=0}) - 1
		\end{aligned} \\
		&=
		\begin{aligned}
			2 \cdot (&\Prob{\DDH_{\group, g}^\advB\Rightarrow\true \mid b=1}\cdot\Prob{b=1} + \\
			&\Prob{\DDH_{\group, g}^\advB\Rightarrow\true \mid b=0}\cdot\Prob{b=0}) - 1 .
		\end{aligned}
	\end{align*} 
	
	To find $\Prob{\DDH_{\group, g}^\advB\Rightarrow\true \mid b=1}$, first notice that when $b=1$, adversary $\advB$ gets $Z = g^{xy}$. This means encryption during $\EncSim$ has the same output distribution as $\EncOracle$ in game $G_0$. Furthermore, in this case game $\DDH_{\group, g}$ returns $\true$ when $\advB$ returns 1, which occurs when $r'=r$. We know that $r'=r$ when $\advA$ successfully determines the correct message encrypted. This happens with the same probability as $\advA$ winning in game $G_0$ and thus $\Prob{\DDH_{\group, g}^\advB\Rightarrow\true \mid b=1} = \Prob{G_0 \Rightarrow\true}$. 
	
	Likewise, to find $\Prob{\DDH_{\group, g}^\advB\Rightarrow\true \mid b=0}$, we again notice that when $b=0$, adversary $\advB$ gets $Z = g^z$. This now means that $\EncSim$ has the same output distribution as $\EncOracle$ in game $G_1$. $\DDH_{\group, g}$ returns $\true$ when $\advB$ returns 1, which occurs when $r'\neq r$. We know that $r' \neq r$ when $\advA$ fails to determine the correct message encrypted. This happens exactly when $\advA$ does not win in game $G_1$ and thus $\Prob{\DDH_{\group, g}^\advB\Rightarrow\true \mid b=0} = 1 - \Prob{G_1 \Rightarrow\true}$. 
	
	We now put this together to get
	\begin{align*}
		\AdvDDH{\group,g}{\advB} &= 2 \cdot \left(\Prob{G_0 \Rightarrow\true} \cdot \frac{1}{2} + (1 - \Prob{G_1 \Rightarrow\true})\cdot\frac{1}{2}\right) - 1 \\
		&= \Prob{G_0 \Rightarrow\true} - \Prob{G_1 \Rightarrow\true}.
	\end{align*}
	
	Lastly, in game $G_1$ since $Z$ is a random value that is multiplied with the message, $C_2$ is also a random value and thus no information is leaked about $b$. We then know that the success of $\advA$ is that of a random coin flip, so $\Prob{G_1 \Rightarrow\true} = \frac{1}{2}$. We then have that
	\begin{align*}
		\AdvINDCPA{\AEnc}{\advA} 
		&= 2\cdotsm\Prob{\INDCPA_{\AEnc}^\advA\Rightarrow\true} - 1\\
		&= 2\cdotsm\Prob{G_0\Rightarrow\true} - 1\\
		&= 2\cdotsm\left(\Prob{G_1\Rightarrow\true} + \AdvDDH{G,g}{\advB})\right) - 1\\
		&= 2\cdotsm\left(\frac{1}{2}+ \AdvDDH{G,g}{\advB})\right) - 1 \\
		&= 2 \cdot \AdvDDH{G,g}{\advB}.
	\end{align*}
\end{proof}

\paragraph{ElGamal in the group $Z_p^*$.} \thref{proof:elgamal} tells us that we must be careful to use a group for which DDH holds when utilizing ElGamal encryption. Indeed, DDH is easy in some well-known groups. In particular, the group $\Z_p^*$ for prime $p$ is one such example. ElGamal is thus not IND-CPA secure when it uses $\Z_p^*$. Before we show how DDH is easy for group $\Z_p^*$, we will go over some relevant topics in number theory that are exploited in the attack. 

An integer $n$ is called a \textit{quadratic residue} modulo $p$ if there is some integer $x$ such that $n \equiv x^2 \ (\bmod \ p)$. Now let $p$ be a prime, $g$ be a generator for $\Z_p^*$, and $a\in\Z_p^*$. 

The \textit{Legendre symbol} is defined as follows:
\bnm
J_p(a) = \left\{ \begin{array}{rl} 
	1 & \textnormal{if $a$ is a quadratic residue mod $p$}\\
	0 & \textnormal{if $a \bmod p = 0$}\\
	-1 & \textnormal{otherwise}
\end{array}\right.
\enm 
When $J_p(a)=1$, this means $a=g^x$ and $x$ is even. When $J_p(a)=-1$, this means $a=g^x$ and $x$ is instead odd. The case where $J_p(a)=0$ is impossible since $\gcd(a,p)=1$. Furthermore, $J_p(a)$ is easy to compute since
\bnm
	J_p(a) = a^{\frac{p-1}{2}} \bmod p.
\enm
To see the proof of this, we point the interested reader to Euler's criterion. Now consider the following facts: 
\begin{fact}
	$J_p(g^{xy}) = 1 \textrm{ iff } J_p(g^x)=1 \textrm{ or } J_p(g^y)=1$
\end{fact}
\begin{fact}
	The number of quadratic residues modulo $p$ in $\Z_p^*$ is $\frac{p-1}{2}$
\end{fact}

\begin{wrapfigure}{r}{2.5in}
	\center
	\begin{tabular}{|c||c|c|}
		\hline
		& $J_p(g^y)=1$ & $J_p(g^y)=-1$ \\
		\hline \hline
		$J_p(g^x)=1$ & $J_p(g^{xy})=1$ & $J_p(g^{xy})=1$ \\
		\hline
		$J_p(g^x)=-1$ & $J_p(g^{xy})=1$ & $J_p(g^{xy})=-1$ \\
		\hline
	\end{tabular}
	\caption{The values of $J_p(g^{xy})$ for various values of $J_p(g^x)$ and $J_p(g^y)$.}
	\label{fig:legendre-table}
\end{wrapfigure}

Note that the first fact is true because $g^{xy}$ is a quadratic residue modulo $p$ if $xy$ is even, which occurs when either $x$ or $y$ is even. It tells us that $\Prob{J_p(g^{xy})=1}=0.75$, as shown in \figref{fig:legendre-table}. The second fact tells us that exactly half the elements of $\Z_p^*$ are quadratic residues. This means that for any $a \in Z_p^*$, $\Prob{J_p(a)}=0.5$. These facts then allow us to find a way to solve the DDH problem in $\Z_p^*$ with non-negligible probability. In particular, the adversary must distinguish between being given $g^z$ for $z \getsr \Z_{p-1}$ and $g^{xy}$ for $x,y \getsr \Z_{p-1}$. Notice that $\Prob{J_p(g^{xy})=1}=0.75$ while $\Prob{J_p(g^{z})=1}=0.5$. Thus, we can construct an adversary $\advB$ that simply computes the Legendre symbol for $Z$, which is easy to compute. $\advB$ is shown below.
\begin{center}
	\fpage{.20}{
		\underline{Adversary $\advB(X,Y,Z)$}\\
		If $J_p(Z) = 1$ then \\
		\myInd Return 1 \\
		Return 0
	}
\end{center}
The DDH-advantage of $\advB$ is
\begin{align*}
	\AdvDDH{\Z_p^*,g}{\advB} &= 2 \cdot \Prob{\DDH_{\Z_p^*, g}^\advB\Rightarrow\true} - 1 \\
	&=
	\begin{aligned}
		2 \cdot (&\Prob{\DDH_{\group, g}^\advB\Rightarrow\true \ | \ b=1}\cdot\Prob{b=1} + \\
		&\Prob{\DDH_{\group, g}^\advB\Rightarrow\true \ | \ b=0}\cdot\Prob{b=0}) - 1
	\end{aligned} \\
	&= 2 \cdot (0.75 \cdot \frac{1}{2} + 0.5 \cdot \frac{1}{2}) - 1 \\
	&= 0.25.
\end{align*}

While a DDH-advantage of 0.25 is quite high, we can construct an adversary $\advB'$ that gains an even greater advantage with only a slightly longer running time. $\advB'$ checks that the Legendre symbols of $X$, $Y$, and $Z$ all match according to the table in \figref{fig:legendre-table}.

\begin{center}
	\fpage{.25}{
		\underline{Adversary $\advB'(X,Y,Z)$}\\
		If $J_p(X) = 1$ or $J_p(Y) = 1$ then\\
		\myInd $s \gets 1$\\
		Else \\
		\myInd $s \gets -1$\\
		If $J_p(Z) = s$ then\\
		\myInd Return 1 \\
		Return 0
	}
\end{center}

Notice that if $J_p(X) = 1$ or $J_p(Y) = 1$, then $J_p(g^{xy})=1$ always. Otherwise, $J_p(g^{xy})=-1$. In the case that $Z=g^z$ for $z \getsr \Z_{p-1}$, there is probability 0.5 that they match. The DDH-advantage of $\advB'$ is then
\begin{align*}
	\AdvDDH{\Z_p^*,g}{\advB} &= 2 \cdot \Prob{\DDH_{\Z_p^*, g}^\advB\Rightarrow\true} - 1 \\
	&=
	\begin{aligned}
		2 \cdot (&\Prob{\DDH_{\group, g}^\advB\Rightarrow\true \ | \ b=1}\cdot\Prob{b=1} + \\
		&\Prob{\DDH_{\group, g}^\advB\Rightarrow\true \ | \ b=0}\cdot\Prob{b=0}) - 1
	\end{aligned} \\
	&= 2 \cdot (1 \cdot \frac{1}{2} + 0.5 \cdot \frac{1}{2}) - 1 \\
	&= 0.5.
\end{align*}

Now that we have shown $Z_p^*$ is not a DDH-hard group, are DL and CDH also not hard in $Z_p^*$? The current best known attack solving DL in $Z_p^*$ is via a generalized number field sieve which runs in time $\bigO(e^{ (C+o(1))\cdot \ln(p)^{1/3} \cdot (\ln\ln(p))^{2/3 }} )$. If the prime factorization of $p-1$ is known, we can solve DL in the subgroups. Since we need to find generators of the group, the factorization is usually known. This means we need subgroups to be large. It is recommended to pick $p=2q+1$ for a large prime $q$. These are referred to as ``safe primes''. We can also use the subgroup of quadratic residues that has order $q$ in this case. Having the order of the group be prime comes with certain advantages, such as exponents having inverses. Furthermore, DDH is conjectured to be hard in this group. 

\section*{Exercises}

\begin{enumerate}[label=\textbf{Exercise \thesection.\arabic*}, wide=0pt]
	\item Let $\group$ be a cyclic group and let $g$ be a generator of $\group$. Let $\advA$ be an adversary against the DL problem. Show that there exists an adversary $\advB$ such that 
	\begin{equation*}
		\AdvDL{\group,g}{\advA} \leq \AdvCDH{\group,g}{\advB}
	\end{equation*}
	where the running time of $\advB$ is that of $\advA$ plus the time to do one exponentiation in $\group$. 
	
	\item Let $\group$ be a cyclic group and let $g$ be a generator of $\group$. Let $\advB$ be an adversary against the CDH problem. Show that there exists an adversary $\advC$ such that 
	\begin{equation*}
	\AdvCDH{\group,g}{\advB} \leq \AdvDDH{\group,g}{\advC} + \frac{1}{|\group|}
	\end{equation*}
	where the running time of $\advC$ is that of $\advB$.
	
	\item (from \cite{BonehShoupBook}, Problem 11.1) Let $\group$ be a cyclic group of prime order $q$ generated by $g\in \group$. Let $\hash:\msgspace \to \group$ be a hash function, which we shall model as a random oracle. Let $\prf$ be the PRF defined over $(\Z_q,\msgspace,\group)$ as follows:
	\begin{equation*}
		\prf(K,M) = \hash(M)^K \textnormal{ for } K \in \Z_q, M \in \msgspace.
	\end{equation*}
	Show that $\prf$ is a secure PRF in the random oracle model for $\hash$ under the DDH assumption for $\group$. In particular, you should show that for every adversary $\advA$ attacking $\prf$ as a PRF, there exists a DDH adversary $\advB$ such that $\AdvPRF{\prf}{\advA}\leq \AdvDDH{(\group,g)}{\advB} + \frac{1}{q}$. 
	
	
\end{enumerate}
\newpage
\huge
% Graphic for TeX using PGF
% Title: /home/phil/digsigs.dia
% Creator: Dia v0.97+git
% CreationDate: Mon Apr 22 16:47:42 2019
% For: phil
% \usepackage{tikz}
% The following commands are not supported in PSTricks at present
% We define them conditionally, so when they are implemented,
% this pgf file will use them.
\ifx\du\undefined
  \newlength{\du}
\fi
\setlength{\du}{15\unitlength}
\begin{tikzpicture}[even odd rule]
\pgftransformxscale{1.000000}
\pgftransformyscale{-1.000000}
\definecolor{dialinecolor}{rgb}{0.000000, 0.000000, 0.000000}
\pgfsetstrokecolor{dialinecolor}
\pgfsetstrokeopacity{1.000000}
\definecolor{diafillcolor}{rgb}{1.000000, 1.000000, 1.000000}
\pgfsetfillcolor{diafillcolor}
\pgfsetfillopacity{1.000000}
\pgfsetlinewidth{0.100000\du}
\pgfsetdash{}{0pt}
\pgfsetbuttcap
\pgfsetmiterjoin
\pgfsetlinewidth{0.100000\du}
\pgfsetbuttcap
\pgfsetmiterjoin
\pgfsetdash{}{0pt}
{\pgfsetcornersarced{\pgfpoint{0.000000\du}{0.000000\du}}\definecolor{diafillcolor}{rgb}{0.000000, 0.792157, 0.839216}
\pgfsetfillcolor{diafillcolor}
\pgfsetfillopacity{1.000000}
\fill (7.782258\du,3.700000\du)--(7.782258\du,7.750000\du)--(11.701613\du,7.750000\du)--(11.701613\du,3.700000\du)--cycle;
}{\pgfsetcornersarced{\pgfpoint{0.000000\du}{0.000000\du}}\definecolor{dialinecolor}{rgb}{0.000000, 0.000000, 0.000000}
\pgfsetstrokecolor{dialinecolor}
\pgfsetstrokeopacity{1.000000}
\draw (7.782258\du,3.700000\du)--(7.782258\du,7.750000\du)--(11.701613\du,7.750000\du)--(11.701613\du,3.700000\du)--cycle;
}% setfont left to latex
\definecolor{dialinecolor}{rgb}{0.000000, 0.000000, 0.000000}
\pgfsetstrokecolor{dialinecolor}
\pgfsetstrokeopacity{1.000000}
\definecolor{diafillcolor}{rgb}{0.000000, 0.000000, 0.000000}
\pgfsetfillcolor{diafillcolor}
\pgfsetfillopacity{1.000000}
\node[anchor=base west,inner sep=0pt,outer sep=0pt,color=dialinecolor] at (9.000000\du,6.000000\du){kg};
\pgfsetlinewidth{0.100000\du}
\pgfsetdash{}{0pt}
\pgfsetbuttcap
\pgfsetmiterjoin
\pgfsetlinewidth{0.100000\du}
\pgfsetbuttcap
\pgfsetmiterjoin
\pgfsetdash{}{0pt}
{\pgfsetcornersarced{\pgfpoint{0.000000\du}{0.000000\du}}\definecolor{diafillcolor}{rgb}{0.000000, 0.792157, 0.839216}
\pgfsetfillcolor{diafillcolor}
\pgfsetfillopacity{1.000000}
\fill (2.782258\du,11.700000\du)--(2.782258\du,15.750000\du)--(6.701613\du,15.750000\du)--(6.701613\du,11.700000\du)--cycle;
}{\pgfsetcornersarced{\pgfpoint{0.000000\du}{0.000000\du}}\definecolor{dialinecolor}{rgb}{0.000000, 0.000000, 0.000000}
\pgfsetstrokecolor{dialinecolor}
\pgfsetstrokeopacity{1.000000}
\draw (2.782258\du,11.700000\du)--(2.782258\du,15.750000\du)--(6.701613\du,15.750000\du)--(6.701613\du,11.700000\du)--cycle;
}% setfont left to latex
\definecolor{dialinecolor}{rgb}{0.000000, 0.000000, 0.000000}
\pgfsetstrokecolor{dialinecolor}
\pgfsetstrokeopacity{1.000000}
\definecolor{diafillcolor}{rgb}{0.000000, 0.000000, 0.000000}
\pgfsetfillcolor{diafillcolor}
\pgfsetfillopacity{1.000000}
\node[anchor=base west,inner sep=0pt,outer sep=0pt,color=dialinecolor] at (3.596369\du,13.946183\du){sign};
\pgfsetlinewidth{0.100000\du}
\pgfsetdash{}{0pt}
\pgfsetbuttcap
\pgfsetmiterjoin
\pgfsetlinewidth{0.100000\du}
\pgfsetbuttcap
\pgfsetmiterjoin
\pgfsetdash{}{0pt}
{\pgfsetcornersarced{\pgfpoint{0.000000\du}{0.000000\du}}\definecolor{diafillcolor}{rgb}{0.000000, 0.792157, 0.839216}
\pgfsetfillcolor{diafillcolor}
\pgfsetfillopacity{1.000000}
\fill (13.782258\du,11.700000\du)--(13.782258\du,15.750000\du)--(17.701613\du,15.750000\du)--(17.701613\du,11.700000\du)--cycle;
}{\pgfsetcornersarced{\pgfpoint{0.000000\du}{0.000000\du}}\definecolor{dialinecolor}{rgb}{0.000000, 0.000000, 0.000000}
\pgfsetstrokecolor{dialinecolor}
\pgfsetstrokeopacity{1.000000}
\draw (13.782258\du,11.700000\du)--(13.782258\du,15.750000\du)--(17.701613\du,15.750000\du)--(17.701613\du,11.700000\du)--cycle;
}% setfont left to latex
\definecolor{dialinecolor}{rgb}{0.000000, 0.000000, 0.000000}
\pgfsetstrokecolor{dialinecolor}
\pgfsetstrokeopacity{1.000000}
\definecolor{diafillcolor}{rgb}{0.000000, 0.000000, 0.000000}
\pgfsetfillcolor{diafillcolor}
\pgfsetfillopacity{1.000000}
\node[anchor=base west,inner sep=0pt,outer sep=0pt,color=dialinecolor] at (14.838548\du,14.000000\du){ver};
% setfont left to latex
\definecolor{dialinecolor}{rgb}{0.000000, 0.000000, 0.000000}
\pgfsetstrokecolor{dialinecolor}
\pgfsetstrokeopacity{1.000000}
\definecolor{diafillcolor}{rgb}{0.000000, 0.000000, 0.000000}
\pgfsetfillcolor{diafillcolor}
\pgfsetfillopacity{1.000000}
\node[anchor=base west,inner sep=0pt,outer sep=0pt,color=dialinecolor] at (7.000000\du,11.200000\du){sk};
% setfont left to latex
\definecolor{dialinecolor}{rgb}{0.000000, 0.000000, 0.000000}
\pgfsetstrokecolor{dialinecolor}
\pgfsetstrokeopacity{1.000000}
\definecolor{diafillcolor}{rgb}{0.000000, 0.000000, 0.000000}
\pgfsetfillcolor{diafillcolor}
\pgfsetfillopacity{1.000000}
\node[anchor=base west,inner sep=0pt,outer sep=0pt,color=dialinecolor] at (12.100000\du,11.200000\du){pk};
% setfont left to latex
\definecolor{dialinecolor}{rgb}{0.000000, 0.000000, 0.000000}
\pgfsetstrokecolor{dialinecolor}
\pgfsetstrokeopacity{1.000000}
\definecolor{diafillcolor}{rgb}{0.000000, 0.000000, 0.000000}
\pgfsetfillcolor{diafillcolor}
\pgfsetfillopacity{1.000000}
\node[anchor=base west,inner sep=0pt,outer sep=0pt,color=dialinecolor] at (7.000000\du,10.000000\du){};
% setfont left to latex
\definecolor{dialinecolor}{rgb}{0.000000, 0.000000, 0.000000}
\pgfsetstrokecolor{dialinecolor}
\pgfsetstrokeopacity{1.000000}
\definecolor{diafillcolor}{rgb}{0.000000, 0.000000, 0.000000}
\pgfsetfillcolor{diafillcolor}
\pgfsetfillopacity{1.000000}
\node[anchor=base west,inner sep=0pt,outer sep=0pt,color=dialinecolor] at (11.178686\du,12.978978\du){M};
% setfont left to latex
\definecolor{dialinecolor}{rgb}{0.000000, 0.000000, 0.000000}
\pgfsetstrokecolor{dialinecolor}
\pgfsetstrokeopacity{1.000000}
\definecolor{diafillcolor}{rgb}{0.000000, 0.000000, 0.000000}
\pgfsetfillcolor{diafillcolor}
\pgfsetfillopacity{1.000000}
\node[anchor=base west,inner sep=0pt,outer sep=0pt,color=dialinecolor] at (0.414624\du,14.186366\du){M};
% setfont left to latex
\definecolor{dialinecolor}{rgb}{0.000000, 0.000000, 0.000000}
\pgfsetstrokecolor{dialinecolor}
\pgfsetstrokeopacity{1.000000}
\definecolor{diafillcolor}{rgb}{0.000000, 0.000000, 0.000000}
\pgfsetfillcolor{diafillcolor}
\pgfsetfillopacity{1.000000}
\node[anchor=base west,inner sep=0pt,outer sep=0pt,color=dialinecolor] at (11.357372\du,15.147153\du){$\sigma$};
% setfont left to latex
\definecolor{dialinecolor}{rgb}{0.000000, 0.000000, 0.000000}
\pgfsetstrokecolor{dialinecolor}
\pgfsetstrokeopacity{1.000000}
\definecolor{diafillcolor}{rgb}{0.000000, 0.000000, 0.000000}
\pgfsetfillcolor{diafillcolor}
\pgfsetfillopacity{1.000000}
\node[anchor=base west,inner sep=0pt,outer sep=0pt,color=dialinecolor] at (11.000000\du,15.000000\du){};
% setfont left to latex
\definecolor{dialinecolor}{rgb}{0.000000, 0.000000, 0.000000}
\pgfsetstrokecolor{dialinecolor}
\pgfsetstrokeopacity{1.000000}
\definecolor{diafillcolor}{rgb}{0.000000, 0.000000, 0.000000}
\pgfsetfillcolor{diafillcolor}
\pgfsetfillopacity{1.000000}
\node[anchor=base west,inner sep=0pt,outer sep=0pt,color=dialinecolor] at (8.078040\du,14.141215\du){$\sigma$};
% setfont left to latex
\definecolor{dialinecolor}{rgb}{0.000000, 0.000000, 0.000000}
\pgfsetstrokecolor{dialinecolor}
\pgfsetstrokeopacity{1.000000}
\definecolor{diafillcolor}{rgb}{0.000000, 0.000000, 0.000000}
\pgfsetfillcolor{diafillcolor}
\pgfsetfillopacity{1.000000}
\node[anchor=base west,inner sep=0pt,outer sep=0pt,color=dialinecolor] at (19.161144\du,14.098928\du){0/1};
\pgfsetlinewidth{0.100000\du}
\pgfsetdash{}{0pt}
\pgfsetmiterjoin
\pgfsetbuttcap
{
\definecolor{diafillcolor}{rgb}{0.000000, 0.000000, 0.000000}
\pgfsetfillcolor{diafillcolor}
\pgfsetfillopacity{1.000000}
% was here!!!
{\pgfsetcornersarced{\pgfpoint{0.000000\du}{0.000000\du}}\definecolor{dialinecolor}{rgb}{0.000000, 0.000000, 0.000000}
\pgfsetstrokecolor{dialinecolor}
\pgfsetstrokeopacity{1.000000}
\draw (9.703125\du,7.849999\du)--(9.761189\du,7.849999\du)--(9.761189\du,9.499999\du)--(7.561189\du,9.499999\du);
}}
\pgfsetlinewidth{0.100000\du}
\pgfsetdash{}{0pt}
\pgfsetmiterjoin
\pgfsetbuttcap
{
\definecolor{diafillcolor}{rgb}{0.000000, 0.000000, 0.000000}
\pgfsetfillcolor{diafillcolor}
\pgfsetfillopacity{1.000000}
% was here!!!
{\pgfsetcornersarced{\pgfpoint{0.000000\du}{0.000000\du}}\definecolor{dialinecolor}{rgb}{0.000000, 0.000000, 0.000000}
\pgfsetstrokecolor{dialinecolor}
\pgfsetstrokeopacity{1.000000}
\draw (9.850000\du,7.800000\du)--(9.813065\du,7.800000\du)--(9.813065\du,9.500000\du)--(12.850000\du,9.500000\du);
}}
\pgfsetlinewidth{0.100000\du}
\pgfsetdash{}{0pt}
\pgfsetbuttcap
{
\definecolor{diafillcolor}{rgb}{0.000000, 0.000000, 0.000000}
\pgfsetfillcolor{diafillcolor}
\pgfsetfillopacity{1.000000}
% was here!!!
\pgfsetarrowsend{stealth}
\definecolor{dialinecolor}{rgb}{0.000000, 0.000000, 0.000000}
\pgfsetstrokecolor{dialinecolor}
\pgfsetstrokeopacity{1.000000}
\draw (7.544724\du,9.449766\du)--(7.546334\du,10.279173\du);
}
\pgfsetlinewidth{0.100000\du}
\pgfsetdash{}{0pt}
\pgfsetbuttcap
{
\definecolor{diafillcolor}{rgb}{0.000000, 0.000000, 0.000000}
\pgfsetfillcolor{diafillcolor}
\pgfsetfillopacity{1.000000}
% was here!!!
\pgfsetarrowsend{stealth}
\definecolor{dialinecolor}{rgb}{0.000000, 0.000000, 0.000000}
\pgfsetstrokecolor{dialinecolor}
\pgfsetstrokeopacity{1.000000}
\draw (12.800000\du,9.500000\du)--(12.800000\du,10.300000\du);
}
\pgfsetlinewidth{0.100000\du}
\pgfsetdash{}{0pt}
\pgfsetmiterjoin
\pgfsetbuttcap
{
\definecolor{diafillcolor}{rgb}{0.000000, 0.000000, 0.000000}
\pgfsetfillcolor{diafillcolor}
\pgfsetfillopacity{1.000000}
% was here!!!
\pgfsetarrowsend{stealth}
{\pgfsetcornersarced{\pgfpoint{0.000000\du}{0.000000\du}}\definecolor{dialinecolor}{rgb}{0.000000, 0.000000, 0.000000}
\pgfsetstrokecolor{dialinecolor}
\pgfsetstrokeopacity{1.000000}
\draw (7.000000\du,10.500000\du)--(4.741935\du,10.500000\du)--(4.741935\du,11.700000\du);
}}
\pgfsetlinewidth{0.100000\du}
\pgfsetdash{}{0pt}
\pgfsetmiterjoin
\pgfsetbuttcap
{
\definecolor{diafillcolor}{rgb}{0.000000, 0.000000, 0.000000}
\pgfsetfillcolor{diafillcolor}
\pgfsetfillopacity{1.000000}
% was here!!!
\pgfsetarrowsend{stealth}
{\pgfsetcornersarced{\pgfpoint{0.000000\du}{0.000000\du}}\definecolor{dialinecolor}{rgb}{0.000000, 0.000000, 0.000000}
\pgfsetstrokecolor{dialinecolor}
\pgfsetstrokeopacity{1.000000}
\draw (13.500000\du,10.500000\du)--(15.741935\du,10.500000\du)--(15.741935\du,11.700000\du);
}}
\pgfsetlinewidth{0.100000\du}
\pgfsetdash{}{0pt}
\pgfsetbuttcap
{
\definecolor{diafillcolor}{rgb}{0.000000, 0.000000, 0.000000}
\pgfsetfillcolor{diafillcolor}
\pgfsetfillopacity{1.000000}
% was here!!!
\pgfsetarrowsend{stealth}
\definecolor{dialinecolor}{rgb}{0.000000, 0.000000, 0.000000}
\pgfsetstrokecolor{dialinecolor}
\pgfsetstrokeopacity{1.000000}
\draw (1.461460\du,13.724458\du)--(2.782258\du,13.725000\du);
}
\pgfsetlinewidth{0.100000\du}
\pgfsetdash{}{0pt}
\pgfsetbuttcap
{
\definecolor{diafillcolor}{rgb}{0.000000, 0.000000, 0.000000}
\pgfsetfillcolor{diafillcolor}
\pgfsetfillopacity{1.000000}
% was here!!!
\pgfsetarrowsend{stealth}
\definecolor{dialinecolor}{rgb}{0.000000, 0.000000, 0.000000}
\pgfsetstrokecolor{dialinecolor}
\pgfsetstrokeopacity{1.000000}
\draw (17.701613\du,13.725000\du)--(19.079325\du,13.728557\du);
}
\pgfsetlinewidth{0.100000\du}
\pgfsetdash{}{0pt}
\pgfsetbuttcap
{
\definecolor{diafillcolor}{rgb}{0.000000, 0.000000, 0.000000}
\pgfsetfillcolor{diafillcolor}
\pgfsetfillopacity{1.000000}
% was here!!!
\pgfsetarrowsend{stealth}
\definecolor{dialinecolor}{rgb}{0.000000, 0.000000, 0.000000}
\pgfsetstrokecolor{dialinecolor}
\pgfsetstrokeopacity{1.000000}
\draw (6.701613\du,13.725000\du)--(7.991503\du,13.720273\du);
}
\pgfsetlinewidth{0.100000\du}
\pgfsetdash{}{0pt}
\pgfsetbuttcap
{
\definecolor{diafillcolor}{rgb}{0.000000, 0.000000, 0.000000}
\pgfsetfillcolor{diafillcolor}
\pgfsetfillopacity{1.000000}
% was here!!!
\pgfsetarrowsend{stealth}
\definecolor{dialinecolor}{rgb}{0.000000, 0.000000, 0.000000}
\pgfsetstrokecolor{dialinecolor}
\pgfsetstrokeopacity{1.000000}
\draw (12.509534\du,12.538433\du)--(13.799424\du,12.533705\du);
}
\pgfsetlinewidth{0.100000\du}
\pgfsetdash{}{0pt}
\pgfsetbuttcap
{
\definecolor{diafillcolor}{rgb}{0.000000, 0.000000, 0.000000}
\pgfsetfillcolor{diafillcolor}
\pgfsetfillopacity{1.000000}
% was here!!!
\pgfsetarrowsend{stealth}
\definecolor{dialinecolor}{rgb}{0.000000, 0.000000, 0.000000}
\pgfsetstrokecolor{dialinecolor}
\pgfsetstrokeopacity{1.000000}
\draw (12.414935\du,14.693178\du)--(13.704826\du,14.688451\du);
}
% setfont left to latex
\definecolor{dialinecolor}{rgb}{0.000000, 0.000000, 0.000000}
\pgfsetstrokecolor{dialinecolor}
\pgfsetstrokeopacity{1.000000}
\definecolor{diafillcolor}{rgb}{0.000000, 0.000000, 0.000000}
\pgfsetfillcolor{diafillcolor}
\pgfsetfillopacity{1.000000}
\node[anchor=base west,inner sep=0pt,outer sep=0pt,color=dialinecolor] at (7.311227\du,3.287261\du){\large key generation};
\end{tikzpicture}

\normalsize
\newpage
%%%%%%%%%%%%%%%%%%%%%%%%%%%%%%%%%%%%%%%%%%%%%%%%%%%%%%%%%%%%%%%%%%%%%%%%%%%%%%%%
\section{Identification Protocols and Fiat-Shamir}
\label{sec:idprots}


For our purposes, an identification protocol is a tuple of algorithms
$(\IDkeygen,\IDcom,\IDresp,\IDver)$ and a set $\IDchalset$ called the challenge space. 
The key generation $\IDkeygen$ outputs a public key and secret key pair
$(\pk,\sk)$. The commitment generator $\IDcom$ is randomized and takes the
public key $\pk$ as input. It outputs a commitment. The response generator $\IDresp$ is
deterministic, takes as input the secret key $\sk$, the commitment $R$, and the
challenge $c \in \IDchalset$, and outputs a response~$z$. The verification
algorithm $\IDver$ is deterministic, takes as input a public key $\pk$, 
commitment $R$, challenge $c$, and response $z$, and outputs a bit.  

\bnm
\AdvIDPASS{\IDscheme,q_{id}}{\advA} = \Prob{\IDPASS_{\IDscheme,q_{id}}^\advA\Rightarrow\true}
\enm

\fpage{.40}{
\underline{$\IDPASS_{\IDscheme,q_{id}}^\advA$}\\[1pt]
$(\pk,\sk)\getsr \IDkg$\\
$\queried \gets \false$\\
For $i = 1$ to $q_{id}$ do\\
\myInd $R_i \getsr \IDcom(\pk)$\\
\myInd $c_i \getsr \IDchalset$\\
\myInd $z_i \getsr \IDresp(\sk,R_i,c_i)$\\
$z^* \getsr \advA^{\VerOracle}(\pk,(R_1,c_1,z_1),\ldots,(R_{q_{id}},c_{q_{id}},z_{q_{id}}))$\\
Ret $\IDver(\pk,R^*,c^*,z^*)$\medskip

\underline{$\VerOracle(R^*)$}\\[1pt]
If $\queried = \true$ then Ret $\bot$\\
$\queried \gets \true$\\
$c^*\getsr \IDchalset$\\
Ret $c^*$
}


\fpage{.43}{
\underline{$\advB(X)$}\\[1pt]
$\queried \gets \false$\\
For $i = 1$ to $q_{id}$ do\\
\myInd $z_i \getsr \Z_q$\\
\myInd $c_i \getsr \Z_q$\\
\myInd $R_i \getsr g^{z_i}X^{-c_i}$\\
$\coins \getsr \CoinSpace_\advA$\\
$z^* \gets \advA^{\VerOracle}(\pk,(R_1,c_1,z_1),\ldots,(R_{q_{id}},c_{q_{id}},z_{q_{id}}) \semi \coins)$\\
$z^{**} \gets \advA^{\VerOracle}(\pk,(R_1,c_1,z_1),\ldots,(R_{q_{id}},c_{q_{id}},z_{q_{id}}) \semi \coins)$\\
If $c^* = c^{**}$ then Ret $x \getsr \Z_q$\\
$x \gets (z^* - z^{**}) / (c^* - c^{**})$\\
Ret $x$\medskip

\underline{$\VerOracle(R^*)$}\\[1pt]
If $\queried = \false$ then \\
\myInd $\queried \gets \true$\\
\myInd Ret $c^{*} \getsr \Z_q$\\
Ret $c^{**} \getsr \Z_q$
}


\begin{lemma*}
Let $S$ and $T$ be finite, non-empty sets and let $f\Colon S\times T\rightarrow
\bits$ be a function. Let $\calX,\calY,\calY'$ be independent random variables,
with $\calX$ taking values in $T$ and $\calY,\calY'$ being uniformly chosen from
$T$. Let $\Pr[f(\calX,\calY) = 1] = \epsilon$. Then 
\bnm
  \Prob{f(\calX,\calY) = 1 \land f(\calX,\calY') = 1 \land \calY \ne \calY'} 
      \ge \epsilon^2 - \frac{\epsilon}{|T|}
\enm
\end{lemma*}

\begin{proof}[Sketch]
Let $\epsilon(s) = \Pr[f(s,\calY) = 1]$. 
Then $\Ex[\epsilon(\calX)] = \epsilon$. Let $N_s$ be the number of values
$Y \in T$ such that $f(s,Y) = 1$. Let $\calU_s$ be the event that $f(s,\calY) =
1 \land f(s,\calY') = 1 \land \calY \ne \calY'$. Then
\bnm
  \Prob{\calU_s} = \frac{N_s(N_s - 1)}{|T|^2} = \epsilon(s)^2 - \frac{\epsilon(s)}{|T|} 
\enm
Let $\calU$ be event that $f(\calX,\calY) = 1 \land f(\calX,\calY') = 1 \land
\calY \ne \calY'$. Then 
\begin{align*}
  \Prob{\calU} 
  %& =\sum_{s \in S} \Prob{f(\calX,\calY) = 1 \land f(\calX,\calY') = 1 \land \calY \ne \calY' \land \calX = s}\\
  & = \sum_{s \in S} \Prob{f(s,\calY) = 1 \land f(s,\calY') = 1 \land \calY \ne \calY' \land \calX = s}\\
  & = \sum_{s \in S} \Prob{f(s,\calY) = 1 \land f(s,\calY') = 1 \land \calY \ne \calY'}\Prob{\calX = s}\\
  & = \sum_{s \in S} \Prob{\calU_s} \Prob{\calX = s}\\
  & = \sum_{s \in S} \left(\epsilon(s)^2 - \frac{\epsilon(s)}{|T|}\right) \Prob{\calX = s}\\
  & = \Ex[\epsilon(\calX)^2] - \frac{\Ex[\epsilon(\calX)]}{|T|}\\
  & \ge \Ex[\epsilon(\calX)]^2 - \frac{\Ex[\epsilon(\calX)]}{|T|}\\
  & = \epsilon^2 - \frac{\epsilon}{|T|}
\end{align*}
\end{proof}

\begin{proof}[Sketch]
Let $\calU$ be event that $f(\calX,\calY) = 1 \land f(\calX,\calY') = 1 \land
\calY \ne \calY'$. Then 
\begin{align*}
  \Prob{\calU} 
  %& =\sum_{s \in S} \Prob{f(\calX,\calY) = 1 \land f(\calX,\calY') = 1 \land \calY \ne \calY' \land \calX = s}\\
  & = \sum_{s \in S} \Prob{f(s,\calY) = 1 \land f(s,\calY') = 1 \land \calY \ne \calY' \land \calX = s}\\
  & = \sum_{s \in S} \Prob{f(s,\calY) = 1 \land f(s,\calY') = 1 \land \calY \ne \calY'}\Prob{\calX = s}\\
  & = \sum_{s \in S} \Prob{\calU_s} \Prob{\calX = s}\\
  & = \sum_{s \in S} \left(\epsilon(s)^2 - \frac{\epsilon(s)}{|T|}\right) \Prob{\calX = s}\\
  & = \Ex[\epsilon(\calX)^2] - \frac{\Ex[\epsilon(\calX)]}{|T|}\\
  & \ge \Ex[\epsilon(\calX)]^2 - \frac{\Ex[\epsilon(\calX)]}{|T|}\\
  & = \epsilon^2 - \frac{\epsilon}{|T|}
\end{align*}

Let $\epsilon(s) = \Pr[f(s,\calY) = 1]$. 
Then $\Ex[\epsilon(\calX)] = \epsilon$. Let $N_s$ be the number of values
$Y \in T$ such that $f(s,Y) = 1$. Let $\calU_s$ be the event that $f(s,\calY) =
1 \land f(s,\calY') = 1 \land \calY \ne \calY'$. Then
\bnm
  \Prob{\calU_s} = \frac{N_s(N_s - 1)}{|T|^2} = \epsilon(s)^2 - \frac{\epsilon(s)}{|T|} 
\enm
\end{proof}




\begin{theorem*}
Let $G$ be a cyclic group, $q_{id} \ge 0$, 
and $\advA$ be an $\IDPASS_{\IDscheme,q_{id}}$-adversary for the Schnorr
scheme~$\IDscheme$
built using $G$. Then we give an $\DL_G$-adversary $\advB$ such that
\bnm
  \AdvIDPASS{\IDscheme,q_{id}}{\advA} \le \sqrt{\AdvDL{G}{\advB}} + \frac{1}{|G|} \;.
\enm
Adversary~$\advB$ runs in time at most twice that of $\advA$ plus
$\bigO(q_{id})$.
\end{theorem*}


\begin{theorem*}
Let $G$ be a cyclic group and $\DS$ be the Schnorr digital signature scheme
using random oracle $\Horacle \Colon\msgspace\rightarrow\Z_{|G|}$.  
Let $\advA$ be an $\UFCMA_{\DS}$-adversary making at most $q_h$ RO queries and
$q_s$ signing queries.
Then we give an $\DL_G$-adversary $\advB$ such that
\bnm
  \AdvUFCMA{\DS}{\advA} \le \frac{q_s(q_s+q_h+1)}{|G|} + (q_h+1)\left(\sqrt{\AdvDL{G}{\advB}} + \frac{1}{|G|}\right) \;.
\enm
Adversary~$\advB$ runs in time at most twice that of the sum of the running time
of~$\advA$ and $\bigO(q_h+q_s)$.
\end{theorem*}

\subsection{Sigma Protocol and Schnorr Signatures}

(Variant of) Schnorr Signatures.
We will be proving the security of Schnorr signatures. 

We have a group of prime order $q$. Let $g$ be a generator $sk = x$ chosen randomly from $Z_q$. Fragility issue: re-using same randomness allows us to easily recover the secret key $x$ by solving two linear equations of $M\neq M'$.
The core security analysis intuition is that the forger can be turned into DL solver by getting the forget to twice forge on the same R, different C. Re-run the adversary twice to cause a discrete log solution from any UF-CMA adversary (cf. 2:06PM).

Closely related primitive in identification protocols. Schnorr sigma protocol. The idea is that you have a prover, a client that wants to convince a verifier that they have a secret key $x$. Ostensibly this is named as such because a capital sigma looks like a depiction of a three-round protocol. If the client doesn't know $x$, they shouldn't be able to answer the challenge correctly.

Similar structure between the two. A Fiat-Shamir transform can take an identification protocol (sigma) and turn it into a digital signature scheme (signing a message and outputting a signature). Apply this transform to generate $c$ non-interactively. Class question:
Is there a scheme from discrete-log that is tight?
Recently solved open question (build a different scheme that is tight). \scribenote{homework problem?}

Analysis game plan for Schnorr:
Formalize ID protocols and their security under \emph{passive attacks}. Prove Schnorr ID protocol secure: this requires the so-called \emph{rewinding lemma} (run an adversary twice on related randomness, likely to succeed twice). It also uses the fact that the Schnorr protocol is \emph{honest-verifier zero-knowledge}.

Prove that Schnorr signature UF-CMA implied by Schnorr ID passive attack security. Formalizing more carefully what passive attacks. Three abstract algorithms prover's commitment algorithm (P.com), challenge set C, prover's response algorithm (P.resp), verifier's verification algorithm (V.ver).

Want to prove ID security under passive attacks (IDPASS) for Schnorr. Generate a bunch of transcripts that the adversary gets access to. A transcript is (R - commitment, c - challenge, z). Adversary gets a single attempt to impersonate the prover. 2:20pm. Prover gets the public key and transcripts and $q_{id}$ honest executions. Assume we have an adversary A . Can we recover $x$ from $X$? Adversary B that can use impersonator A to recover $x$ via solving discrete log problem. We need to supply the $q_{id}$ transcripts. Need to simulate these transcripts and somehow need to run A twice intuitively to get two transcripts with different c* values. Can we simulate Ri, ci, zi because Schnorr is zero-knowledge and by our ZK scheme we do not need to possess the secret key to simulate the tuple. Build discrete-log (DL) adversary that runs a twice to get two transcripts that allow extracting sk = x.
How can we create believable transcripts without the secret key? 
Recall that a real transcript requires:
identically distributed triple without knowing $x$?
$g^r$, $c$, $z = cx + r$.
First idea: pick random $g^r$, $c$, $z$. But $z$ is independent of the other two. doesn't satisfy with probability 1 over the entire group. 
The real tuples only have two random values and one determined by $x$. We won't pass public verifier since we don't know $x$.
What else can we do to make a believable transcript? We do know $X$. Pick random $c$, random $z$, then calculate $g^r$ using those, as we have inverted the order that we select things.
$R = g^z * X^{-c}$, so $R(x^c) = g^z$ and $g^z * $

This is without loss given the same distribution and the equations check out, given that we had $X$. Anybody can generate a transcript without knowing $x$. This is an intentional part of the zero-knowledge aspect of this proof. But we can leverage this fact to make our believable transcripts. In actual deployment, however, you can't choose R is a function of c. This ordering in deployment is critical for security. This is why it is called a commitment, to pick R before seeing c. This will still suffice for the purposes of our reduction.

Revisit our attack plan. We need to observe two forgeries -- rewind the adversary to the point after they query R* and re-run from there with a different c* value.
Reduction from IDPASS to Discrete Log (DL).

B(x):
First generate our trick transcript without $x$ by picking $z$, $c$ first and computing $r$. Then run a once to get $z*$, then rewind to get a fresh $z**$ only difference is the different challenge value. Intuitively $c*=c**$ has very low probability. 

How do we lower bound the probability that A succeeds twice? Key analysis captured by rewinding lemma (aka reset lemma):
If the runs were completely independent (IID), epsilon squared, but they are not here (2:40pm, 2:44pm).
Same coins (w) on both runs, made A deterministic up until it queries Ver.
Lemma. Let S and T be finite, non-empty sets and let f: S x t -> {0, 1} be a function. Let X, Y, Y' be independent random variables, with X taking values in S and Y, Y' being uniformly chosen from T. Let Pr[f(X,Y) = 1] = epsilon.

f is the success predicate over the set of coins/commitment value (2:45pm).
Then Pr[f(X,Y) = 1 (this is c*) \^ f(X,Y') = 1 (c**) \^ Y neq Y'] ge epsilon sq - (epsilon over absolute val of T).
\scribenote{Prove the lemma as a homework problem?}

See IDPASS security of Schnorr from slides (2:50pm).
Theorem. Let G by a cyclic group, $q_{id} \ge 0$, and A be an IDPASS - adversary for the Schnorr scheme ID built using G. Then we give an $DL_G$-adversary B such that
...
lower bounded by $\epsilon^2 - \frac{\epsilon}{|G|} \ge ()$.
Exercise for the readers to do algebra. 
We've proven Schnorr ID protocols are secure under DL passive attack security. Still need (2:55pm).
Prove security of fiat-shamir's output which is this security ... under UF-CMA.
We want to show any adversary UF-CMA against Schnorr we can convert the passive attack adversary against Schnorr protocols.
Going to use the fact that it's RO to generate c values for the adversary A. UF-CMA to IDPASS reduction. The idea is:
IDPASS adversary B that runs A:
\begin{enumerate}
    \item Guess RO query i* corresponds to forgery, query ver to set H[Mi*] = c*. Can embed challenge point in RO output.
    \item use ci values as response to other RO queries.
    \item Use (Ri, ci, zi) to simulate signing queries
\end{enumerate}

Simulates A's environment perfectly if i* guess correct and no commitments R collide with previous H queries.
We succeed if i* is correct.
Putting it all together: UF-CMA to DL.
Theorem statement: let $G$ be a cyclic group and DS be the Schnorr digital signature scheme using random oracle... (3:02pm - ``subtle detail'' 3:03pm.)
Group size should be roughly twice that of bits of security needed. 256 bit group for 128 bit security...
\newpage
%%%%%%%%%%%%%%%%%%%%%%%%%%%%%%%%%%%%%%%%%%%%%%%%%%%%%%%%%%%%%%%%%%%%%%%%%%%%%%%%
\section{Zero-Knowledge and Proofs of Knowledge}
\label{sec:zknowledge}
Identity protocols provide proof about the ownership of a private key, without revealing that private key.
We can generalize this notion to proving knowledge of an arbitrary secret without revealing the secret as \emph{zero-knowledge (ZK) proofs}.

For the formal definition of a ZK proof, we follow Boneh-Shoup's treatment, an effective relation is a binary relation $\calR \subseteq \calX \times \calY$  where $\calR, \calX,\calY$ are sets.
(We assume the are efficiently recognizable, meaning one can determine if a value is a member of the set efficiently.)
Elements of $\calY$ are called \emph{statements} and elements of $\calX$ are called \emph{witnesses}.

\fpage{.25}{
\underline{$\NIZK_{\Phi,\Simu}^\advA$}\\[1pt]
$b \getsr \bits$\\
$b' \getsr \advA^{\ProofOracle}$\\
Return $(b = b')$\medskip

\underline{$\ProofOracle(x,Y)$}\\[1pt]
If $b = 1$ then\\
\myInd $\pi \getsr \Gen(x,Y)$\\
If $b = 0$ then \\
\myInd $\pi \getsr \Simu(Y)$\\
Return $\pi$
}

\fpage{.15}{
\underline{$\NIZK1_{\Phi,\Simu}^\advA$}\\[1pt]
$b \getsr \bits$\\
$b' \getsr \advA^{\ProofOracle,\Horacle}$
Ret $(b = b')$\medskip

\underline{$\ProofOracle(x,Y)$}\\[1pt]
$\pi \getsr \Gen^\Horacle(x,Y)$\\
Ret $\pi$\medskip

\underline{$\Horacle(M)$}\\[1pt]
If $\TabH[M] = \bot$ then\\
\myInd $\TabH[M] \getsr \bits^n$\\
Ret $\TabH[M]$
}
\fpage{.15}{
\underline{$\NIZK0_{\Phi,\Simu}^\advA$}\\[1pt]
$b \getsr \bits$\\
$b' \getsr \advA^{\ProofOracle,\HashSim}$\\
Return $(b = b')$\medskip

\underline{$\ProofOracle(x,Y)$}\\[1pt]
$\pi \getsr \Simu_P(Y)$\\
Return $\pi$\medskip

\underline{$\HashSim(M)$}\\[1pt]
Ret $\Simu_H(M)$
}



\newpage
%%%%%%%%%%%%%%%%%%%%%%%%%%%%%%%%%%%%%%%%%%%%%%%%%%%%%%%%%%%%%%%%%%%%%%%%%%%%%%%%
\section{Pairings-Based Cryptography}
\label{sec:bilinear}

Consider a triple of groups $\G_1,\G_2,\G_T$ each of prime order $p$ and let
$g_1,g_2$
be generators of $\G_1,\G_2$, respectively. The group
$\G_T$ is traditionally called the target
group, for reasons that will be made clear shortly. A bilinear pairing for these
groups is a function $\pair\Colon\G_1\times\G_2\rightarrow\G_T$ that satisfies
the following properties:
\begin{newitemize}
\item Bilinear: For any $a,b \in \Z_p^2$, we have that
$\pair(g_1^a,g_2^b) = = g_T^{ab}$. 
%
\item Non-degenerate: $\pair(g_1,g_2)$ is a generator of $\G_T$. Let $g_T =
\pair(g_1,g_2)$.
%
\item Efficiently computable: The pairing $\pair$ is efficient to compute.
\end{newitemize}

There are a number of different groups/pairing construction. A classic one is
the Weil pairing, which has $\G_1$ be an elliptic curve over a finite field
$F_p$. In this case we think of $\G_1$ as being supersingular, letting us take
$\G_2 = \G_1$ and $\G_T = F_{p^k}^*$ for some $k$ (multiplicative group modulo
$p^k$). We call $k$ the embedding degree. For example, Boneh and Franklin used
$k = 2$. There are other pairings, such as the Tate pairing, but we will not
discuss the mathematics in detail. 



\fpage{.20}{
\underline{$\sign(\sk,M)$}\\[1pt]
$\sigma \gets H(M)^{\sk}$\\
Return $\sigma$\medskip

\underline{$\ver(\pk,M,\sigma)$}\\[1pt]
If $\pair(H(M),\pk) = \pair(\sigma,\pk)$ then\\
\myInd Ret 1\\
Ret 0
}

\fpage{.25}{
\underline{$\UFCMA_{\textnormal{BLS}}$}\\
$\sk \getsr \Z_p$\\
$\pk \gets h^\sk$\\
$(M^*,\sigma^*) \getsr \advA^{\Horacle,\SignOracle}(\pk)$\\
If $M^* \in \calM$ then Ret $\false$\\
Ret $\left(\pair(\TabH[M^*],\pk) = \pair(\sigma^*,h) \right)$\\

\underline{$\Horacle(M)$}\\
If $\TabH[M] = \bot$ then\\
\myInd $\TabH[M] \getsr \G_1$\\
Ret $\TabH[M]$\medskip

\underline{$\SignOracle(M)$}\\
$\calM \gets \calM \cup \{M\}$\\
$X \getsr \Horacle(M)$\\
Ret $X^\sk$
}


\fpage{.15}{
\underline{$\IBEpg$}\\[1pt]
$\msk \getsr \Z_p$\\
$\mpk \gets h^{\msk}$\\
Ret $(\mpk,\msk)$
}

\fpage{.15}{
\underline{$\IBEkg(\msk,\id)$}\\[1pt]
$\sk \getsr H(\id)^\msk$\\
Ret $(\sk)$
}

\fpage{.19}{
\underline{$\IBEenc(\mpk,\id,M)$}\\[1pt]
$r \getsr \Z_p$\\
$K \gets \pair(H(\id)^r,\mpk)$\\
$C \gets H'(K) \oplus M$\\
Ret $(h^r,C)$
}

\fpage{.19}{
\underline{$\IBEdec(\sk,(R,C))$}\\[1pt]
$K \gets \pair(\sk,R)$\\
$M \gets H'(K) \oplus C$\\
Ret $M$
}

\fpage{.19}{
\underline{$\INDIDCPA_{\IBE}^\advA$}\\[1pt]
$(\mpk,\msk) \gets \IBEpg$\\
$b \getsr \bits$\\
$b' \getsr \advA^{\KgOracle,\EncOracle}(\mpk)$\\
Ret $(b = b')$\medskip

\underline{$\KgOracle(\id)$}\\
If $\id \in \calI$ then Ret $\bot$\\
$\calI \gets \calI \cup \{\id\}$\\
$\sk \getsr \IBEkg(\msk,\id)$\\
Ret $\sk$\medskip

\underline{$\EncOracle(\id^*,M_0,M_1)$}\\
If $\id^* \in \calI$ then Ret $\bot$\\
$C \getsr \IBEenc(\mpk,\id^*,M_b)$\\
Ret $C$
}

\fpage{.19}{
\underline{$\INDIDCPA_{\IBE}^\advA$}\\[1pt]
$\msk \getsr \Z_p$\\
$\mpk \gets h^{\msk}$\\
$b \getsr \bits$\\
$b' \getsr \advA^{\KgOracle,\EncOracle,\Horacle,\Horacle'}(\mpk)$\\
Ret $(b = b')$\medskip

\underline{$\KgOracle(\id)$}\\
If $\id = \id^*$ then Ret $\bot$\\
$\calI \gets \calI \cup \{\id\}$\\
$\sk \gets \Horacle(\id)^{\msk}$\\
Ret $\sk$\medskip

\underline{$\EncOracle(\id^*,M_0,M_1)$}\\
If $\id^* \in \calI$ then Ret $\bot$\\
$r \getsr \Z_p$\\
$K \gets \pair(\Horacle(\id^*)^r,\mpk)$\\
$C \gets M_b \oplus H'(K)$\\
Ret $(h^r,C)$\medskip

\underline{$\Horacle(\id)$}\\
If $\TabH[\id] = \bot$ then\\
\myInd $\TabH[\id] \getsr \G_1$\\
Ret $\TabH[\id]$\medskip

\underline{$\Horacle'(Z)$}\\
If $\TabH'[Z] = \bot$ then\\
\myInd $\TabH'[Z] \getsr \bits^\ell$\\
Ret $\TabH[Z]$
}




\bigskip

A digital signature scheme $\DS = (\kg,\sign,\ver)$ is a triple of
algorithms. Key generation is randomized and outputs a key pair $(\pk,\sk)$,
where $\pk$ is called the public or verification key and $\sk$ is called the secret or
signing key.
Signing takes as input a secret key $\sk$ and a message $M$ and outputs a
signature, usually denoted $\sigma$.
Verification takes as input a public key $\pk$, message $M$, and signature
$\sigma$ and outputs a
bit.



\fpage{.20}{
		\underline{$\UFCMA_\DS^\advA$}\\
    $(\pk,\sk) \getsr \kg$\\
    $(M^*,\sigma^*) \getsr \advA^\SignOracle(\pk)$\\
    If $M^* \in \calM$ then \\
    \myInd Ret $\false$\\
		Ret $\ver(\pk,M^*,\sigma^*)$\medskip

    \underline{$\SignOracle(M)$}\\
    $\calM \gets \calM \cup \{M\}$\\
    $\sigma \getsr \sign(\sk,M)$\\
    Ret $\sigma$
	}

\bnm
  \AdvUFCMA{\DS}{\advA} = \Prob{\UFCMA_{\DS}^\advA\Rightarrow\true}
\enm



Let $\DS$ be the full domain hash (FDH) digital signature scheme. 


\begin{theorem*}
Let $\DS$ be the FDH scheme using $\RSAk$ and $\Horacle\Colon\msgspace\rightarrow\Z_N^*$ modeled as
a RO. Let $\advA$ be any $\UFCMA_\DS$-adversary making at most $q_h$ queries to
$\Horacle$ and $q_s$ queries to its signing oracle. 
Then we give an $\RSAk$-adversary $\advB$ such that
\bnm
    \AdvUFCMA{\DS}{\advA} \le (q_h + q_s + 1) \cdotsm\AdvOWF{\RSAk}{\advB}
\enm
Adversary~$\advB$ runs in time that of $\advA$ plus $\bigO(q_s+q_h)$. 
\end{theorem*}


\begin{proof}
To start we make some simplifying assumptions about $\advA$. First, whenever it
makes a query to $\SignOracle$ on a message $M$ it has previously made a query
$\Horacle(M)$. Second, when it outputs a forgery attmept $(M^*,\sigma^*)$ it has
previously queried $M^*$. And, as we normally assume, it does not repeat a query
to $\Horacle$.  Should $\advA$ not abide by these restrictions, we can easily
build an adversary $\advA'$ from $\advA$ that does so, at cost at most an extra
$q_s + 1$ queries to $\Horacle$.

Now, to build some intuition, let's first consider the case that $q_s = 0$, that is, 
we are arguing about unforgeability in a no-message attack. 
%Assume that $\advA$
%queries $\Horacle(M^*)$ where $M^*$ is the output message for its forgery. This
%is without loss. 
Then in this case we can 
guess which random oracle query corresponds to the solution, and program its
output to be equal be the OWF challenge $Y$.  In more detail, we set our forgery
adversary $\advB$ to be:
%Intuitively, to forge against FDH one needs to invert RSA on $\Horacle(M)$ for
%some $M$. Then we can build an inverter $\advB$ that works as follows. 

\fpage{.25}{
\underline{$\UFCMA_{\DS}$}\\
$((N,e),(N,d)) \getsr \kg$\\
$(M^*,\sigma^*) \getsr \advA^{\Horacle,\SignOracle}((N,e))$\\
If $M^* \in \calM$ then Ret $\false$\\
Ret $\left(\TabH[M^*] = (\sigma^*)^e \bmod N\right)$\medskip

\underline{$\Horacle(M)$}\\
If $\TabH[M] = \bot$ then\\
\myInd $\TabH[M] \getsr \Z_N^*$\\
Ret $\TabH[M]$\medskip

\underline{$\SignOracle(M)$}\\
$\calM \gets \calM \cup \{M\}$\\
$X \getsr \Horacle(M)$\\
Ret $X^d \bmod N$
}

\fpage{.25}{
\underline{$\G_0$ \;\;\; \fbox{$\G_1$}}\\
$((N,e),(N,d)) \getsr \kg$\\
$i^* \getsr [1,q]$\\
$i \gets 0$\\
$(M^*,\sigma^*) \getsr \advA^{\HashSim}((N,e))$\\
If $(M^* \ne M_{i^*})$ then \\
\myInd $\badtrue$\\
\myInd \fbox{Ret $\false$}\\
Ret $\left(\TabH[M^*] = (\sigma^*)^e \bmod N\right)$\medskip

\underline{$\HashSim(M)$}\\
$i \gets i+1$\\
$M_i \gets M$\\
If $i = i^*$ then\\
\myInd Ret $\TabH[M_i] \getsr \Z_N^*$\\
$\TabH[M_i] \getsr \Z_N^*$\\
Ret $\TabH[M_i]$
}
\fpage{.25}{
\underline{$\G_2$}\\
$((N,e),(N,d)) \getsr \kg$\\
$i^* \getsr [1,q]$\\
$i \gets 0$\\
$(M^*,\sigma^*) \getsr \advA^{\HashSim}((N,e))$\\
If $(M^* \ne M_{i^*})$ then \\
\myInd $\badtrue$\\
\myInd \fbox{Ret $\false$}\\
Ret $\left(\TabH[M^*] = (\sigma^*)^e \bmod N\right)$\medskip

\underline{$\HashSim(M)$}\\
$i \gets i+1$\\
$M_i \gets M$\\
If $i = i^*$ then\\
\myInd Ret $\TabH[M_i] \gets Y$\\
$\TabH[M_i] \getsr \Z_N^*$\\
Ret $\TabH[M_i]$
}

\fpage{.25}{
\underline{$\advB_{toy}((N,e),Y)$}\\
$(M^*,\sigma^*) \getsr \advA^{\HashSim}((N,e))$\\
Ret $\sigma^*$\medskip

\underline{$\HashSim(M)$}\\
If $i = i^*$ then\\
\myInd Ret $Y$\\
}


\fpage{.30}{
\underline{$\advB((N,e),Y)$}\\
$i^* \getsr [1,q]$\\
$i \gets 0$\\
$(M^*,\sigma^*) \getsr \advA^{\HashSim,\SignSim}((N,e))$\\
If $(M^* \ne M_{i^*})$ then $X' \getsr \Z_N^*$\\
Else $X' \gets \sigma^*$\\
Ret $X'$\medskip

\underline{$\HashSim(M)$}\\
$i \gets i+1$\\
$M_i \gets M$\\
If $i = i^*$ then\\
\myInd Ret $Y$\\
$\sigma_i \getsr \Z_N^*$\\
$\TabH[M_i] \gets (\sigma_i)^e \bmod N$\\
Ret $\TabH[M_i]$\medskip

\underline{$\SignSim(M)$}\\
Let $i$ be s.t.~$M_i = M$\\
If $i = i^*$ then\\
\myInd Ret $\sigma \getsr \Z_N^*$\\
Ret $\sigma_i$
}
\fpage{.30}{
\underline{$\G_0$}\\
$((N,e),(N,d) \getsr \kg$\\
$X \getsr \Z_N^*$\\
$Y \gets X^e \bmod N$\\
$i^* \getsr [1,q]$\\
$i \gets 0$\\
$(M^*,\sigma^*) \getsr \advA^{\HashSim,\SignSim}((N,e))$\\
If $(M^* \ne M_{i^*})$ then \\
\myInd $\badtrue$\\
\myInd $X' \getsr \Z_N^*$\\
$X' \gets \sigma^*$\\
Ret $(X = X')$\medskip

\underline{$\HashSim(M)$}\\
$i \gets i+1$\\
$M_i \gets M$\\
If $i = i^*$ then\\
\myInd Ret $Y$\\
$\sigma_i \getsr \Z_N^*$\\
$\TabH[M_i] \gets (\sigma_i)^e \bmod N$\\
Ret $\TabH[M_i]$\medskip

\underline{$\SignSim(M)$}\\
Let $i$ be s.t.~$M_i = M$\\
If $i = i^*$ then\\
\myInd $\badtrue$\\
\myInd Ret $\sigma \getsr \Z_N^*$\\
Ret $\sigma_i$
}
\fpage{.30}{
\underline{$\G_1$}\\
$((N,e),(N,d) \getsr \kg$\\
$X \getsr \Z_N^*$\\
$Y \gets X^e \bmod N$\\
$i^* \getsr [1,q]$\\
$i \gets 0$\\
$(M^*,\sigma^*) \getsr \advA^{\HashSim,\SignSim}((N,e))$\\
If $(M^* \ne M_{i^*})$ then \\
\myInd $\badtrue$\\
\myInd $X' \gets \sigma^*$\\
$X' \gets \sigma^*$\\
Ret $(X = X')$\medskip

\underline{$\HashSim(M)$}\\
$i \gets i+1$\\
$M_i \gets M$\\
If $i = i^*$ then\\
\myInd Ret $Y$\\
$\sigma_i \getsr \Z_N^*$\\
$\TabH[M_i] \gets (\sigma_i)^e \bmod N$\\
Ret $\TabH[M_i]$\medskip

\underline{$\SignSim(M)$}\\
Let $i$ be s.t.~$M_i = M$\\
If $i = i^*$ then\\
\myInd $\badtrue$\\
\myInd Ret $\sigma \gets X$\\
Ret $\sigma_i$
}


\begin{align*}
  \AdvUFCMA{\DS}{\advA} &= \Prob{\G_1\Rightarrow\true}\\
      &\le \Prob{\G_0\Rightarrow\true} + \Prob{\bad_0}\\
      &= \AdvOWF{\RSAk}{\advB} + \Prob{\bad_0}
\end{align*}

\begin{align*}
\AdvOWF{\RSAk}{\advB_{toy}} = \AdvUFCMA{\DS}{\advA}
\end{align*}

Intuitively $\advB$ wins as long as $\advA$ wins and $i^*$ is the correct guess
of which hash query by $\advA$ corresponds to the winning $M^*$. (Recall that we
are assuming that $\advA$ always queries $\Horacle$ for the forgery message
$M^*$.) To analyze this, let game $\G_0$ be equal to $\UFCMA_{\DS}^\advA$ except
that it additionally includes a random choice of $i^* \getsr [1,q]$ and sets a
flag $\bad$ to true should $M^* \ne M_{i^*}$. Let
game $\G_1$ be the same as $\G_0$ except that  it chooses a random index $i^*
\getsr [1,q]$ and checks if $M^*$ is equal to the $i\thh$ random oracle
query. If not, it sets a flag $\bad$ to true and outputs $\false$, 
Otherwise it checks if $\advA$'s output is a winning forgery, outputing
$\true$ if so and $\false$ otherwise. Let $\good$ be the event that $\bad$ is
not set in game $\G_0$, and we let $\good$ be the same event for $\G_1$ (a
slight abuse of notation). Then we have that both games are
identical-until-$\bad$ and a variant of the fundamental lemma of game playing
says that:
\bnm
  \Prob{\G_0 \Rightarrow\true \land\good} = \Prob{\G_1\Rightarrow\true \land
  \good} \;.
\enm
Then using this, we have that:
\begin{align*}
\AdvOWF{\RSAk}{\advB} 
  &= \Prob{\G_0\Rightarrow\true}
  &\ge \Prob{\G_0\Rightarrow\true\land\good}\\
  &= \Prob{\G_1\Rightarrow\true\land\good}\\
  &= \Prob{\G_1\Rightarrow\true}\cdot\Prob{\good}\\
  &= \AdvUFCMA{\DS}{\advA}\cdotsm\frac{1}{q}
\end{align*}
%Given that $\Horacle$ is modeled as a random oracle, this would seem
%to correspond to having to 
%We must create an adversary $\advB$ that inverts RSA .

\end{proof}



\newpage
%%%%%%%%%%%%%%%%%%%%%%%%%%%%%%%%%%%%%%%%%%%%%%%%%%%%%%%%%%%%%%%%%%%%%%%%%%%%%%%%
\section{Lattice-based Crypto and LWE}
\label{sec:lattices}


Given lattice basis $\textbf{B} = [\vecb_1,\ldots,\vecb_n] \in \mathbb{Z}^n$,
and $r \in \mathbb{Q}$, determine whether $\lambda_1(\calL(\textbf{B})) \le r$
or $\lambda_1(\calL(\textbf{B})) > \gamma \cdot r$.


\begin{align*}
  m' &= c_2 - c_1\vecs\\
     &= \vecu^T\textnormal{pk} + m\lceil q/2\rceil -   \vecu^T\textbf{A}\vecs \\
     &= \vecu^T\textbf{A}\vecs + \vecu^T\vece + m\lceil q/2\rceil -   \vecu^T\textbf{A}\vecs \\
     &= \vecu^T\vece + m\lceil q/2\rceil
\end{align*}




\printbibliography

%\bibliographystyle{plain}
%\bibliography{notes}


\end{document}
